\chapter{Fazit} \label{sec:conclusion}

\section{Evaluation der Verfahren}

Die implementierten Verfahren haben gezeigt, dass mit dem Ansatz der Tiefenbild Überdeckung von \citet{wloka1995resolving} eine Echtzeit Überdeckung virtueller Objekte auf der mobilen Project Tango Hardware erfolgreich umgesetzt werden kann. Dabei wird im Folgenden auf jedes Verfahren sowie ihrer Vor und Nachteile im Kontext der anderen Verfahren und auf Basis der durchgeführten Tests eingegangen. \\

Die Überlagerung durch die Pointcloud Projektion bietet gegenüber den anderen Verfahren den Vorteil, dass sie zu jeder Zeit eine dynamische und aktualisierte Repräsentation der Tiefe der Szene liefert und somit auch Änderungen in der Szene sofort berücksichtigt. Außerdem es das Verfahren nicht auf Clustergrößen beschränkt und kann dadurch auch



Verfahren
* Pointcloud Projektion
	* Ungenau, nach Eingangsdaten
	* Eingangsdaten könnten aufgewertet werden
	* Flüchtig
	* Rauschen
* Ebenen Rekonstruktion
	* Ungenaue Daten werden auf Ebenen gebannt
	* Augmentierend
	* Lücken zwischen Ebenen 
* TSDF Rekonstruktion (CHISEL)
	* Reduktion von Ungenauigkeit
	* Augmentierend durch Space Carving
	* Konflikt zwischen SubChungSize und Performance
	* => Aktuell Cluster zu groß, führt zu fehlern beim Marching Cubes
* Guided Filter
	* Sehr langsam
	* Führt zu erstaunlichen Optimierungen
	* Weist jedoch je nach Einstellung auch Gradienten auf glatten Tiefenbildern auf


\section{Ausblick}

* Lenovo deployed
* Bilateral Filter in API während Arbeit erschienen => Guided Filter wäre der Performantere Ansatz
* Light-Field Cameras als Kombination für AR!!


	
	