\documentclass[12pt]{support/thcolognethesis} 

\usepackage{amssymb}

\title{Optimierung von Augmented Reality Anwendungen durch die Berücksichtigung von Tiefeninformationen mit Googles Project Tango}

\degree{Masterthesis}

\author{Steffen Tröster}

\college{
	Technischen Hochschule Köln\\
    Ingenieurwissenschaftliches Zentrum\\
    Fakultät für Informations-,\\
    Medien- und Elektrotechnik}

\course{Technische Informatik (Master)}

\company{inovex GmbH}  

\firstExaminer{Prof. Dr. Hubert Randerath}
\firstExaminerLocation{Technische Hochschule Köln}
\secondExaminer{Prof. Dr. Martin Eisemann}
\secondExaminerLocation{Technische Hochschule Köln}

\degreedate{Köln, im \monthyeardate\today}
	
\begin{document}

\baselineskip=18pt plus1pt

\setcounter{secnumdepth}{3}
\setcounter{tocdepth}{3}

\maketitle                 

\begin{abstract}
\setlength{\parskip}{1em}

Project Tango ist eine neue mobile Plattform des Google Advanced Technology and Project (ATAP) Teams, die in der Lage ist, Bewegungsverfolgung, Tiefenwahrnehmung und Umgebungswiedererkennung auf Smartphones und Tablets anbieten zu können. Durch die kontinuierliche Bestimmung der relativen Geräteposition eignet sich die Plattform besonders für dreidimensionale Augmented Reality (AR) Anwendungen. Die Illusion dieser AR Anwendungen wird besonders dann gestört, wenn sich reale Objekte in einer Szene räumlich vor virtuellen Objekten befindet und diese virtuellen Objekte nicht entsprechend ausgespart werden. 

Diese Arbeit stellt daher drei Überdeckungsverfahren vor, mit denen diese Überlagerung der virtuellen Objekte mit Hilfe der Tiefenwahrnehmung von Project Tango und des Z-Buffer Algorithmus realisiert werden kann. Die Tiefeninformationen für den Z-Buffer werden hierfür zum einen direkt aus den Sensordaten und alternativ mit einer TSDF Rekonstruktion und einer selbst zusammengestellten Ebenenrekonstruktion bestimmt. Außerdem wird auf einen zusätzlichen Ansatz eingegangen, der zur Verbesserung dieser Tiefeninformationen die Bildinformationen der Farbkamera durch den Guided Filter berücksichtigt. Diese Mechanismen werden im Laufe der Arbeit prototypisch umgesetzt und gegenübergestellt. 

\setlength{\parskip}{0em}
\end{abstract}
\selectlanguage{english}
\begin{abstract}
\setlength{\parskip}{1em}

Project Tango is a new mobile platform by Google’s Advanced Technology and Projects (ATAP) Teams, which brings Motion Tracking, Depth Perception, and Area Learning to smartphone and tablet devices. With its Motion Tracking technology, Project Tango is suitable for precise three dimensional augmented reality (AR) applications. The illusion of the model projection in these AR applications is often disrupted when some real objects in the scene standing in front of virtual projections, which are not getting occluded.

This thesis is comparing three occlusion mechanisms, which can solve the virtual object occlusion with Project Tangos depth perception by applying the Z-Buffer algorithm. The Z-Buffer is filled either by the direct sensor data, by a TSDF reconstruction method or by a self combined and implemented plane based reconstruction. Additionally a guided image filtering approach is applied to the depth map to interpolate according to the edges of the RGB image frame. These mechanisms are going to be implemented and compared.

\setlength{\parskip}{0em}
\end{abstract}
\selectlanguage{ngerman}


         

\begin{romanpages}         
\tableofcontents            
\end{romanpages}          

% Einleitung
\chapter{Einleitung}

Project Tango ist eine neue mobile Plattform des Google Advanced Technology and Projects (ATAP) Teams, welche Bewegungsverfolgung, Tiefenwahrnehmung und Umgebungswiedererkennung auf mobilen Endgeräten realisiert.

\begin{quotation}
\enquote{Project Tango combines 3D motion tracking with depth sensing to give your mobile device the ability to know where it is and how it moves through space.}  \citep{Proje19:online}
\end{quotation}

Diese Verfügbarkeit ermöglicht viele verschiedene neue Einsatzmöglichkeiten auf mobilen Endgeräten wie Smartphones und Tablets. Typische Einsatzszenarien dieser Plattform sind die Indoor Navigation, die Vermessung der Umgebung sowie andere typische Anwendungen für Virtual und Augmented Reality. Der Fokus dieser Forschungsarbeit liegt hier in dem Anwendungsbereich Augmented Reality (AR). \\

Um eine erfolgreiche AR Anwendung umsetzen zu können, müssen die Kameraeigenschaften, wie Brennweite, Verzerrung und die Position der Kamera zu jeder Zeit bekannt sein. Sensoren wie Kompass, INS (Trägheitsnavigationssystem) oder GPS können zwar eine grobe Lokalisierung ohne bekannte Merkmale im Raum ermöglichen, führen aber langfristig zu Fehlern, wenn keine optischen Referenzen gegeben sind. Mit Hilfe von der Bewegungsverfolgung durch Project Tango kann diese Lokalisierung der Kamera und somit die korrekte Positionierung von virtuellen Objekten im Raum deutlich zuverlässiger und ohne vordefinierte Merkmale im Raum realisiert werden. Project Tango eignet sich daher sehr gut für die Umsetzung und den Einsatz von AR Anwendungen.\\

Ein sinnvoller Einsatz von Augmented Reality besteht darin, virtuelle Objekte in eine echte Szene zu projizieren. Dabei überlagert die Projektion des virtuellen Objekts das aktuelle Kamerabild oder den aktuellen Sichtbereich und erwirkt dadurch den Anschein, dass sich das virtuelle Objekt wirklich in der Szene befindet. Dieser Effekt funktioniert solange erfolgreich, bis ein reales Objekt sich räumlich vor das virtuelle Objekt bewegt und die zu erwartende Überlagerung des virtuellen Objekts nicht erfolgt. \\

Die Project Tango Plattform bietet die Möglichkeit Tiefeninformationen mit Hilfe eines Tiefensensors für den aktuellen Sichtausschnitt zu bestimmen. Hierdurch können Interaktionen oder Darstellungen in Augmented Reality Anwendung näher an die echten räumlichen Gegebenheiten angepasst werden. Es existieren zum Beispiel prototypische Anwendungen, in denen virtuelle Markierungen passend an echten Objekten im virtuellen Raum positioniert werden können, indem sie auf die aktuellen Tiefeninformation des Sichtbereichs zurückgreifen.\\

Diese Arbeit versucht die Fragestellung zu beantworten, durch welche Verfahren mit Hilfe der Tieninformationen von Project Tango, automatisch und in Echtzeit Überdeckung virtueller Objekte mit realen Objekten in einer Augmented Reality Szene realisiert werden können. Dabei soll Project Tango als Autonomes System betrachtet werden, welches diese Problemstellung selbstständig und mit den eingeschränkten Ressourcen dieser mobilen Plattform lösen soll.\\


\section{Vorgehen}




%Die erste Fragestellung richtet sich dem Thema, wie man performant und automatisiert geometrische Primitiven in einer Szene finden kann. Dazu soll zunächst eine Literaturrecherche bezüglich bekannter Methoden und Algorithmen durchgeführt werden, welche darauf folgend anhand gestellter Kriterien entsprechend evaluiert werden. Für diese Evaluation ist auch die Erstellung einer einheitlichen, zum AR Anwendungsfall passenden, Testumgebung denkbar. Letztendlich soll eine prototypische Implementierung dieser Primitiven Detektion erstellt werden, auf der im späteren Verlauf aufgebaut werden kann.\\

%Später soll bestimmt werden ob und wie sich diese gefundenen Primitiven im Nachhinein oder im Verlauf einer Anwendung selbstständig verbessern oder optimieren lassen, oder ob die Basisdaten (Pointcloud) entsprechend verbessert werden können. Hierzu soll näher untersucht werden ob die Bildinformationen aus der Project Tango Kamera dabei helfen können durch zum Beispiel Kantenerkennung eine Optimierung vorzunehmen. Die hieraus gewonnen Erkenntnisse sollen genutzt werden, um den zuvor implementierten Prototypen weiter zu verbessern.\\




% Thematische Vorbemerkung
\include{content/chapter_one}

% Optimierung von AR
\chapter{Verfahren zur echtzeit Optimierung von Augmented Reality durch Tiefeninformationen}

\section{Verdeckung durch Depth Maps}

Der erste weniger aufwändige Weg eine Überlagerung in Augmented Reality zu realisieren ist das Einbringen der Depth Map in das Rendering. \citet{kanbara2000stereoscopic} haben diese Methode für die Anwendung mit einer Stereokamera und einer video see-through Displaytechnologie in Form eines Head-Mounted Display umgesetzt. Das Positionstracking erfolgte dabei durch drei optische Marker im Raum. \\

\begin{figure}[h]
  \centering
	\includegraphics[width=1.0\textwidth]{content/images/methods/stereo-depth-map.png} 
  \caption{Visualisierung des Methode zur Vedeckung durch Depth Maps. Übernommen von \citet{kanbara2000stereoscopic}}
  \label{fig:stereo-depth-map}
\end{figure}

Als Erstes werden in diesem Verfahren 2D Bounding Boxen der zu rendernden Objekte auf dem Viewport aus beiden Sichtfeldern der Stereokamera bestimmt. Diese wird durch eine Projektion der 3D Bounding Box eines Objektes erzeugt. Die Regionen wurden zunächst ermittelt, um die Bestimmung von Tiefeninformationen aus Performancegründen auf diese Bereiche zu beschränken. Als nächstes wird in den ermittelten Regionen ein Stereomatching mit Ecken aus der Anwendung des Sobel Filter durchgeführt. Hieraus können darauf folgend Tiefeninformationen der reellen Umgebung gewonnen werden. \citep{kanbara2000stereoscopic} Da die Projekt Tango Hardware direkt Tiefeninformationen durch den Infrarot Laser liefert, wird das Stereomatching hier nicht weiter erläutert.   \\

Um den Ausschluss der Pixel eines virtuellen Objektes, welches sich hinter einem reellen Objekt befinden, zu verhindern, wird der Z-Buffer Algorithmus nach \citet{greene1993hierarchical} angewendet. Dabei werden zunächst die ermittelten realen Tiefeninformationen in den Z-Buffer gefüllt. Das führt dazu, dass ein virtueller Bildpunkt des zu rendernden Objektes, welcher einen höheren Z-Wert als der bereits vorhandene Wert im Z-Buffer hat, also weiter vom Betrachter entfernt ist, nicht in den Framebuffer gelangt wird. In Abbildung \ref{fig:stereo-depth-map} ist das Vorgehen dieses Verfahrens noch einmal visualisiert. \\




\section{Polygon Rekonstruktion}

\subsection{Marching Cubes als Echtzeit Umsetzung}



\subsection{Possion Reconstruction}

\subsection{Greedy Projection Triangulation}



\section{Planare Rekonstruktion}

Wie bereits in Absatz \ref{sec:polygon_reconstruction} beschrieben, lässt sich das Problem der Optimierung von Augmented Reality mit Hilfe von Tiefeninformationen auf eine Echtzeit Rekonstruktion zurückführen. Im Gegensatz zur Rekonstruktion komplexer Oberflächen, mit dem vorgestellten TSDF Verfahren, soll hier eine Idee näher erläutert werden, die eine Rekonstruktion allein auf planaren Primitiven ermöglicht. \citet{yang2010plane} erwähnt hierzu, dass Ebenen in fast allen künstlichen Umgebungen zu finden sind und auf Grund ihrer vorteilhaften geometrischen Eingenschaften in verschiedensten Computer Vision Verfahren verwendet werden. Daher gibt es viele Forschungsarbeiten, Methoden und Algorithmen um aus verschiedensten Informationsquellen ein Ebenenmodell zu extrahieren.\\

Wie in dem \enquote{Simultaneous Localization and Mapping} (SLAM) Verfahren von \citet{trevor2012planar} wird hier zunächst eine Ebene in der Pointcloud mit Hilfe des RANSAC Algorithmus gesucht. RANSAC bietet gegenüber anderen Algorithmen zur Ebenen Detektion den Vorteil, ein Modell auch bei vielen Ausreißern performant zu ermitteln. Agglomeratives Clustering und Region Growing wie von \citet{feng2014fast} beschreiben, eignet sich auf Grund des Ausgabeformats aus Project Tango nicht, da es keine organisierte Point Cloud ausgibt. \\

Die Repräsentation der Ebene \(P\) wird, angelehnt an das Vorgehen von \citet{trevor2012planar}, wie in Gleichung \ref{eq:plane} festgehalten. Dabei handelt es sich um den Normalenvektor \(\vec{n}\) und der Distanz zum Ursprung \(d\) der Hesse Normalform einer Ebene, sowie der Punkte der konvexen Hülle \(H\). Um die konvexe Hülle der Ebene zu bestimmen, wird der Graham Scan Algorithmus verwendet. Wie auch von \citet{trevor2012planar} beschreiben wird die konvexe in der Repräsentation festgehalten, um eine sukzessive Verbesserung einer Ebene nach mehreren Messdurchläufen zu ermöglichen. So werden die Punkte der konvexen Hülle pro Messvorgang kombiniert, damit die Ebenenausbreitung auch außerhalb des Sichtfeldes beibehalten werden kann.

\begin{equation} \label{eq:plane}
P=\left[\vec{n}, d, H\right] \qquad H=\vec{h_1}, \vec{h_2}, \ldots  \vec{h_n}
\end{equation}

Um wiederum aus dieser Repräsentation eine Triangulation zu erhalten, wird hier die zweidimensionale Sweep-Line Delaunay Triangulation durchgeführt. Die Schritte aus dem Algorithmus \ref{lst:planeReconstruction} werden in den folgenden Kapiteln näher beschreiben.

\begin{lstlisting}[mathescape,caption=Planaren Echtzeit Rekonstruktion, label=lst:planeReconstruction]

Eingabe: Octree $O$
Ausgabe: Polygonpunkte $T_{Gesamt}$

für jedes Cluster $C$ aus $O$
    bestimme Ebene [$\vec{n}$, $d$, $P$] mit RANSAC aus $C_{Punkte}$
    wenn keine Ebene mit genügend $P$ in $C_{Punkte}$ gefunden wurde
    		nächstes Cluster (continue)
    wenn Ebene mit [$\vec{n}$, $d$, $H_{alt}$] in $C_{Ebenen}$ existiert	
        füge die konvexe Hülle $H_{alt}$ zu $P$ hinzu	
    bestimme die konvexe Hülle $H_{neu}$
    bestimme die Tringulation $T_{Ebene}$ aus $H_{neu}$
    $T_{Gesamt}$ += $T_{Ebene}$
    $C_{Ebenen}$ += [$\vec{n}$, $d$, $H_{neu}$]
    $C_{Punkte}$ - $P$
		

\end{lstlisting}

\subsection{RANSAC zur Ebenendetektion}

Der \enquote{RAndom SAmple Consensus} Algorithmus (RANSAC), vorgestellt von \citet{fischler1981random}, ist in der Lage, aus einer Menge von Daten mit vielen Ausreißern, die Parameter für ein passendes Modell zu schätzen. Anders als andere Schätzverfahren wie \enquote{Least-Median} oder \enquote{M-Schätzer}, welche aus der Statistik Literatur entnommen und entsprechend angepasst wurden, wurde RANSAC speziell für die Anwendung in der Computer Graphik entwickelt. Der Kern dieses Algorithmus ist das wiederholte Bestimmen eines Modells aus zufälligen und für das Modell ausreichenden Stichproben. Listing \ref{lst:ransac} zeigt den Verlauf des RANSAC Algorithmus. Die Anzahl der Iterationen \(N\) hängt dabei allein von dem Anteil der Ausreißer in den Messwerten ab. Daher sollte sie entsprechend gewählt werden, um die Wahrscheinlichkeit zu verringern, dass Ausreißer in den Stichproben enthalten sind. \citep{derpanis2010overview} \\

\begin{lstlisting}[mathescape,caption=Der RANSAC Algorithmus, label=lst:ransac]
Eingabe: Messwerte $P$, Modelltoleranz $e$, maximale Iterationen $N$
Ausgabe: Modell $m$, Unterstützende Messwerte $P_m$

1. Wähle zufällig so viele Stichproben aus den Messwerten $P$,
   wie nötig sind, um das Modell zu bestimmen
2. Bestimme aus den gewählten Stichproben das Modell $m$
3. Ermittle die Anzahl der Messwerte $P$, die mit einer 
   entsprechenden Toleranz $e$ das ermittelte Modell $m$ 
   unterstützen
4. Wenn prozentual genügend Messwerte aus $P$ das Modell $m$ 
   unterstützen, ermittle aus den unterstützenden Messwerten 
   $P_m$ durch lineare Regression erneut das finale Modell 
   $m$ und terminiere
5. Wiederhole die Schritte 1-4 $N$ mal
\end{lstlisting} 

Um mit dem RANSAC Algorithmus Ebenen in einer Punktewolke bestimmen zu können, werden pro Iteration drei Stichproben \(A\), \(B\) und \(C\) gewählt. Das Ebenenmodell, hier in der Hesse Normalform mit dem Normalenvektor \(\vec{n}\) und dem Abstand zum Koordinatenursprung \(d\), lässt sich dabei durch die Gleichung \ref{eq:normalform} bestimmen.

\begin{equation}\label{eq:normalform}
\vec{n} =\left|\left| \vec{AB} \times \vec{AC}\right|\right|
\qquad
\vec{D} = \vec{A} \cdot \vec{n}
\qquad
d = D_1 + D_2 + D_3
\end{equation}

Um zu ermitteln ob ein Punkt \(P\) aus einer Messreihe die gefundene Ebene \(\left[\vec{n_E}, d_E\right]\) unterstützt, wird die kürzeste Distanz \(d_P\) zwischen Punkt und Ebene wie in Gleichung \ref{eq:plane-distance} ermittelt.  Ein entsprechender Toleranzwert für die Distanz \(d_{min}\), im gezeigten RANSAC Algorithmus \(e\) genannt, wird später bei der Umsetzung abhängig vom Rauschen des Tiefensensors gewählt. 

\begin{equation} \label{eq:plane-distance}
d_P = n_1*P_1+n_2*P_2+n_3*P_3-d_E \qquad support_{d_P} = d_P < d_{min}
\end{equation}

Um das finale Modell der Ebene zu ermitteln, und somit die Varianz des Abstands der Punkte zur Ebene zu minimieren, wird mit Hilfe der unterstützenden Punkte \(P_{support}=\left[x,y,z\right]\) eine lineare Regression durchgeführt. Diese Mittelt ein Ebenenmodell \(E=\left[a,b,c\right]\) aus den zuvor ermittelten Punkten mit Hilfe des \enquote{least squares} Schätzverfahren. Für eine Ebene versucht man die Funktion \(G(a,b,c)\) aus Gleichung \ref{eq:least-squares} zu minimieren. Hierzu muss das lineare Gleichungssystem in Gleichung \ref{eq:least-squares-solution}  gelöst werden. \citep{Regre94:online}

\begin{equation} \label{eq:least-squares}
z = ax + by + c   \qquad G(a, b, c) = \sum {\left(z_i - ax_i - by_i - c\right)}^2
\end{equation}

\begin{equation} \label{eq:least-squares-solution}
\begin{bmatrix}
\sum x_i^2 & \sum x_iy_i & \sum x_i\\ 
\sum x_iy_i  & \sum y_i^2  & \sum y_i \\ 
\sum x_i & \sum y_i  & n
\end{bmatrix}
*
\begin{bmatrix}
a\\ 
b\\ 
c
\end{bmatrix}
=
\begin{bmatrix}
\sum x_iz_i\\ 
\sum y_iz_i\\ 
\sum z_i
\end{bmatrix}
\end{equation}

\subsection{Bestimmung der Ebenenausbreitung}

Nachdem die Ebene und die korrespondierenden Punkte zur Ebene gefunden wurden, muss noch die Ausbreitung der Fläche bestimmt werden, da die Ebene in Hesse Normalform lediglich die Position \(\vec{n} * d\) und Ausrichtung \(\vec{n}\) festhält. \citet{PlanarSurfaceMapping} nutzt hierfür die konvexe Hülle der korrespondierenden Punkte und trianguliert diese. Um das performant umzusetzen, kann man sich hier die Eigenschaft der Ebene zu Nutzen machen und die dreidimensionalen Punkte durch Parallelprojektion als zweidimensionale Punkte auf die Ebene projizieren. Denn die Algorithmen für die Triangulation haben im zweidimensionalen eine deutlich besseres Laufzeitverhalten. \\

Nach der Triangulation können die Ecken der gefundenen Polygone jeweils zurück projiziert werden. Die Gleichungen \ref{eq:projection2d} und \ref{eq:projection3d} bilden die Projektion der Punkte wobei \(R_{\vec{n}to\vec{z}}\) der Rotationsmatrix zwischen dem Normalenvektor \(\vec{n}\) und der Z-Achse \(\vec{z}\) entspricht.\\

\begin{equation} \label{eq:projection2d}
p_{2d} = (p_{3d} - (\vec{n}*d)) * R_{\vec{n}to\vec{z}}
\end{equation}
\begin{equation} \label{eq:projection3d}
p_{3d} = (p_{2d} * R_{\vec{n}to\vec{z}}^{-1}) + (\vec{n}*d)
\end{equation}

\subsubsection{Convex Hull Algorithmus}

Für die Berechnung der konvexen Hülle wird der Graham Scan nach \citet{graham1972efficient} genutzt. Dieser Algorithmus besitzt eine Laufzeit von \(O(n \log n)\) und gilt als einer der populärsten Algorithmen für die Berechnung der konvexen Hülle. Andere Ansätze besitzen dabei ein ähnliches oder schlechteres Laufzeitverhalten. Gestartet wird der Algorithmus mit der Menge aller Punkte \(P\) der Ebene und mit einem Startpunkt \(P_0\), der Bestandteil der konvexen Hülle ist. Hierzu wird meist der Punkt mit dem niedrigsten \(y\) Faktor gewählt. (\(P_0=P_{min(y)}\)) Listing \ref{lst:graham-scan} zeigt den Verlauf des Algorithmus des Graham Scans. Dabei wird in Abbildung \ref{fig:convexhull} nochmals die erste Sortierung und das Unterscheidungskriterium für die Sortierung als auch für die Aussortierung der Punkte verdeutlicht. \citep{convexHull} \\

\begin{figure}
  \centering
	\includegraphics[width=0.9\textwidth]{content/images/methods/convexhull.png} 
  \caption{Sortierung der Punkte nach Winkel zum Startpunkt (links) mit dem Unterscheidungskriterium (rechts). Übernommen von \citet{convexHull}}
  \label{fig:convexhull}
\end{figure}

\begin{lstlisting}[mathescape,caption=Graham Scan Algorithmus, label=lst:graham-scan]
Eingabe: Menge der Punkte $P$, außen liegender Punkt $P_0$
Ausgabe: Punkte der konvexen Hülle

$i$ = 0
sortiere nach dem Winkel zu $P_0$
solange $i$ <= $|P|$
    wenn $\measuredangle P_{i-1} P_{i}$ > $\measuredangle P_{i-1} P_{i+1}$, also $P_i$ rechts von  $\vec{P_{i-1} P_{i+1}}$ liegt
        inkrementiere $i$
    ansonsten
        entferne $P_i$ aus $P$
        dekrementiere $i$
    
\end{lstlisting} 


\subsubsection{Triangulation}

Nachdem die konvexe Hülle bestimmt wurde, müssen die Punkte dieses Pfades noch trianguliert werden. Hierfür wird eine Delaunay-Triangulation mit Hilfe des Sweep-Line Algorithmus von \citet{domiter2008sweep} verwendet. Unter einer Delaunay-Triangulation versteht man zunächst einmal eine Art Constraint, der sogenannten Umkreisbedingung, die besagt, dass ein Kreis, der alle drei Punkte eines gefundenen Polygons durchzieht, keine weiteren Punkte beinhalten darf. \\

Die Sweep-Line Methode ist ein generelles geometrisches Vorgehen, bei dem eine vertikale imaginäre Linie eine Menge geometrischer Objekte von links nach rechts passiert. 

 ... to be continued

\subsection{Clustering der aufgenommenen Punkte}

Wie im Algorithmus \ref{lst:planeReconstruction} zu erkennen wird das zuvor beschriebene Vorgehen für die planare Rekonstruktion immer pro Cluster eines Cluster-Pools durchgeführt. Dadurch werden pro Durchgang des Algorithmus nur ein Bruchteil der gesammelten Punkte rekonstruiert, was wiederum eine Rekonstruktion in Echtzeit möglich macht. Außerdem verhindert das Clustering das Bilden von konvexen Hüllen über Ebenen, die in Zwischenbereichen nicht mit genügend Punkten unterstützt werden. Dieses Problem ist in Abbildung \ref{fig:clustering} links zu sehen.\\

\begin{figure}
  \centering
	\includegraphics[width=1.0\textwidth]{content/images/methods/clustering.png} 
  \caption{Links: Ebenenrekonstruktion ohne Clustering. Rechts: Rekonstruktion mit K-Mean Clustering.}
  \label{fig:clustering}
\end{figure}

Getestet wurde hier das K-Mean Clustering, Agglomeratives Clustering und einfaches räumliches Clustern mit Hilfe eines Octrees. Das K-Mean Clustering hat, wie in Abbildung \ref{fig:clustering} rechts zu erkennen, gute Ergebnisse für die Aufteilung einer Ebenen geliefert, benötigt aber zuvor eine feste Anzahl von Clustern. Agglomeratives Clustering, getestet mit dem euklidischen Distanzmaß, würde zwar die Anzahl der Cluster dynamisch bestimmen, ist jedoch zu aufwändig für eine echtzeit Rekonstruktion. Gute Ergebnisse liefert wiederum ein einfaches räumliches Clustern mit einem Octree. Das bietet zudem den Vorteil, dass diese Datenstruktur direkt als Speicherort der Aufgenommenen Punkte und Ebenen dienen kann. \\

Ein Octree ist zunächst eine Datenstruktur, die wie ein Baum mit beliebiger Tiefe aufgebaut ist und pro Knoten Acht Kinder besitzt. Dabei repräsentiert ein Knoten ein Würfel, der durch seine Kinder in Acht Kind-Würfel aufgeteilt wird. Durch diese räumliche Aufteilung ergeben sich verschiedene Vorteile gegenüber linearen Datenstrukturen. So müssen Bereiche zum festhalten räumlicher Informationen im Octree nur dann allokiert werden, wenn diese Bereiche auch verwendet werden. Speichert man Punkte in den untersten Knoten eines Octrees kann man durch eine Tiefenbegrenzung beim Zugriff auf den Baum ein sehr effektives Downsampling der Punkte vornehmen. Zuletzt entstehen durch die Knoten einer bestimmten Tiefe ein Cluster, zu denen in diesem Fall Punkte bei einer Aufnahme hinzugefügt werden und für eine weitere Verarbeitung extrahiert werden können.\\




% Umsetzung
\chapter{Umsetzung der Verfahren}

Dieses Kapitel widmet sich der Umsetzung der in der praktischen Umsetzung der beschriebenen Verfahren zur Echtzeit Planaren oder Polygon Rekonstruktion. Außerdem werden zunächst die allgemeinen technischen Problemstellungen bezüglich Project Tango beschrieben und eine Umsetzung vorgestellt, wie man mit Tiefeninformationen in Augmented Reality interagieren kann.



\section{Project Tango}

Project Tango ist eine Technologie Plattform für Android Tablets und Smartphones von Google’s Advanced Technology and Projects Group (ATAP). Das Ziel dieser Plattform ist es Motion Tracking (Positionierung), Depth Perception (Tiefeninformation/Pointcloud) und Area Learning (Lokalisierung) auf mobile Endgeräte zu bringen, um verschiedenste Anwendungs-Szenarien abzudecken. Typische Szenarien sind Indoor Navigation, Virtual Reality Anwendungen, Vermessungs- und Rekonstruktions Software und Augmented Reality Anwendungen.

Es ermöglicht in erster Linie ein Tracking von Positionsänderungen des Geräts im Raum und bietet somit eine genaue relative Lokalisierung. Mit Hilfe dieser Lokalisierung und der Hinzunahme von visuellen Merkmalen im Raum, ist das Gerät in der Lage, seine Umgebung kennenzulernen und gegebenenfalls die Lokalisierung zu korrigieren oder aber in einer bereits erlernten Umgebung zu ermitteln. Zusätzlich bietet Project Tango die Möglichkeit mit Hilfe eines Tiefensensors eine Pointcloud der Tiefeninformation pro Bildausschnitt zu liefern, um Anwendungen auch Räumliche Informationen bereitzustellen.  \citep{Proje19:online}

\subsection{Geräte und Hardware}

Da das Project Tango zum Zeitpunkt der Verfassung dieser Thesis noch unter Entwicklung steht, gibt es von Google die Entwickler Prototypen. Das Erste Gerät im Smartphone Format, welches in Abbildung \ref{fig:tango-device} rechts unten zu erkennen ist, wurde bereits durch eine neue Generation rechts oben ersetzt. Dieses 7\dq Tablet verfügt, wie in Abbildung \ref{fig:tango-device} links zu erkennen, über einen Infrarot Laser Projektor, eine Fisheye Camera und eine normale 4 Megapixel Kamera auf der Rückseite. Zudem sind, wie von aktuellen Smartphones bekannt, ein Beschleunigungssensor, Umgebungslichtsensor, Barometer, Kompass, GPS und ein Gyroskop verbaut. Das Gerät wird von einem NVIDIA Tegra K1 Prozessor betrieben und verfügt über 4GB Arbeitsspeicher. \citep{Proje19:online} Mit diesem Gerät wurden die später beschriebenen Techniken umgesetzt und evaluiert. 

\begin{figure}[h]
  \centering
	\includegraphics[width=1.0\textwidth]{content/images/theory/tango-device.png} 
  \caption{Links: schematischer Aufbau der Google Project Tango Hardware. Rechts: Das aktuelle Entwickler Gerät im Tablet Faktor (oben) und das alte Entwickler Gerät im Smartphone Faktor (unten). Übernommen von \citet{GoogleDevelopers:online}}
  \label{fig:tango-device}
\end{figure}

\subsection{Konzepte und Schnittstellen}

Generell betrachtet ist das Project Tango eine Plattform, die Computer Vision nutzt, um dem Gerät die Möglichkeit bietet seine relative Positionierung in der umgebenen Szene live zu bestimmen. Auf den Geräten kommt Googles Android zum Einsatz, weshalb zu beachten ist, dass es sich bei der Platform nur Bedingt um eine Echtzeit Umgebung handelt. Das liegt daran, dass der Linux Kernel keine Garantien für die zeitlich präzise Ausführung von Instruktionen  auf Grund von Scheduling geben kann. Google weist daher darauf hin, dass das System als \enquote{soft-realtime} betrachtet werden sollte. Daher sollten Messergebnisse verschiedener Sensoren unter Berücksichtigung ihrer Aufnahme Zeitstempel verwendet werden. \citep{GoogleDevelopersConcepts:online}

\subsubsection{Motion Tracking}

Um die relative Bewegung vom Start des Project Tango Systems bestimmen zu können nutzt es \enquote{visual-inertial odometry}. \citep{GoogleDevelopersConcepts:online}
Dabei handelt es sich um eine erweiterte Variante von Visual Odometry. 
Das von \citet{nister2004visual} veröffentlichte Verfahren Visual Odometry ist in der Lage aus einfachen Video Inhalten in Echtzeit die Bewegung der Kamera zu bestimmen. 
Hierzu werden zunächst übergreifende Features, zum Beispiel Punkte aus der \citet{harris1988combined} Kantenerkennung, über mehrere Bilder des Videos bestimmt, woraus mit Hilfe des 5-point Algorthmus und durch Answendung von RANSAC eine bestmögliche Approximation der Kamera Transformation bestimmt wird. \citep{nister2004visual}

Project Tango lässt an dieser Stelle die internen Sensoren zur Rotation, Orientierung und Bewegung mit in die Bestimmung der Kamera Transformation einfließen um so ein akurateres Ergebnis erziehlen zu können. Über eine längere Messzeit oder eine größere Entfernung vom Ursprung kann es jedoch zu kleinen Abweichungen kommen. Außerdem existiert zum aktullen Zeitpunkt noch ein \enquote{drift} Problem, was zu großen Messfehlern führen kann. Es wird jedoch versucht diese Probleme mit dem Konzept \enquote{Area Learning} aus Kapitel\ref{subsec:area-learning} zu lösen. \citep{GoogleDevelopersConcepts:online}
Wie genau das Verfahren aussieht, welche Techniken zur Feature Detection oder Feature Matching genutzt wird und welche Features hierfür erkannt werden ist nicht bekannt. 

\subsubsection{Deph Perception}

\subsubsection{Area Learning}


\subsection{Einordnung zu Augmented Reality}

Um das Project Tango in den zuvor erwähnten technologischen Charakteristika einordnen zu können, wird zunächst einmal auf die technischen Details der Plattform eingegangen. 


\section{Tiefe aus Pointcloud Projektion}

Wie im Kapitel \ref{sec:pc-projection} erwähnt müssen die Punkte der Project Tango Pointcloud auf die Bildebene projiziert werden und mit einer entsprechenden Tiefenfarbe und einem Radius auf den Tiefenpuffer gezeichnet werden. Dieser Schritt wurde auch bereits in Prototoypen mit den angegebenen Gleichungen umgesetzt. Jedoch kann man sich hierfür das OpenGL Rendering zu Nutze machen und die Projektion OpenGL übernehmen lassen. Denn OpenGL unterstützt für das Rendering neben Polygonen auch primitiven wie Punkte und Linien. 




\input{content/chapter_three/marching_cubes}

\section{Planare Rekonstruktion}

Wie bereits in Absatz \ref{sec:polygon_reconstruction} beschrieben, lässt sich das Problem der Optimierung von Augmented Reality mit Hilfe von Tiefeninformationen auf eine Echtzeit Rekonstruktion zurückführen. Im Gegensatz zur Rekonstruktion komplexer Oberflächen, mit dem vorgestellten TSDF Verfahren, soll hier eine Idee näher erläutert werden, die eine Rekonstruktion allein auf planaren Primitiven ermöglicht. \citet{yang2010plane} erwähnt hierzu, dass Ebenen in fast allen künstlichen Umgebungen zu finden sind und auf Grund ihrer vorteilhaften geometrischen Eingenschaften in verschiedensten Computer Vision Verfahren verwendet werden. Daher gibt es viele Forschungsarbeiten, Methoden und Algorithmen um aus verschiedensten Informationsquellen ein Ebenenmodell zu extrahieren.\\

Wie in dem \enquote{Simultaneous Localization and Mapping} (SLAM) Verfahren von \citet{trevor2012planar} wird hier zunächst eine Ebene in der Pointcloud mit Hilfe des RANSAC Algorithmus gesucht. RANSAC bietet gegenüber anderen Algorithmen zur Ebenen Detektion den Vorteil, ein Modell auch bei vielen Ausreißern performant zu ermitteln. Agglomeratives Clustering und Region Growing wie von \citet{feng2014fast} beschreiben, eignet sich auf Grund des Ausgabeformats aus Project Tango nicht, da es keine organisierte Point Cloud ausgibt. \\

Die Repräsentation der Ebene \(P\) wird, angelehnt an das Vorgehen von \citet{trevor2012planar}, wie in Gleichung \ref{eq:plane} festgehalten. Dabei handelt es sich um den Normalenvektor \(\vec{n}\) und der Distanz zum Ursprung \(d\) der Hesse Normalform einer Ebene, sowie der Punkte der konvexen Hülle \(H\). Um die konvexe Hülle der Ebene zu bestimmen, wird der Graham Scan Algorithmus verwendet. Wie auch von \citet{trevor2012planar} beschreiben wird die konvexe in der Repräsentation festgehalten, um eine sukzessive Verbesserung einer Ebene nach mehreren Messdurchläufen zu ermöglichen. So werden die Punkte der konvexen Hülle pro Messvorgang kombiniert, damit die Ebenenausbreitung auch außerhalb des Sichtfeldes beibehalten werden kann.

\begin{equation} \label{eq:plane}
P=\left[\vec{n}, d, H\right] \qquad H=\vec{h_1}, \vec{h_2}, \ldots  \vec{h_n}
\end{equation}

Um wiederum aus dieser Repräsentation eine Triangulation zu erhalten, wird hier die zweidimensionale Sweep-Line Delaunay Triangulation durchgeführt. Die Schritte aus dem Algorithmus \ref{lst:planeReconstruction} werden in den folgenden Kapiteln näher beschreiben.

\begin{lstlisting}[mathescape,caption=Planaren Echtzeit Rekonstruktion, label=lst:planeReconstruction]

Eingabe: Octree $O$
Ausgabe: Polygonpunkte $T_{Gesamt}$

für jedes Cluster $C$ aus $O$
    bestimme Ebene [$\vec{n}$, $d$, $P$] mit RANSAC aus $C_{Punkte}$
    wenn keine Ebene mit genügend $P$ in $C_{Punkte}$ gefunden wurde
    		nächstes Cluster (continue)
    wenn Ebene mit [$\vec{n}$, $d$, $H_{alt}$] in $C_{Ebenen}$ existiert	
        füge die konvexe Hülle $H_{alt}$ zu $P$ hinzu	
    bestimme die konvexe Hülle $H_{neu}$
    bestimme die Tringulation $T_{Ebene}$ aus $H_{neu}$
    $T_{Gesamt}$ += $T_{Ebene}$
    $C_{Ebenen}$ += [$\vec{n}$, $d$, $H_{neu}$]
    $C_{Punkte}$ - $P$
		

\end{lstlisting}

\subsection{RANSAC zur Ebenendetektion}

Der \enquote{RAndom SAmple Consensus} Algorithmus (RANSAC), vorgestellt von \citet{fischler1981random}, ist in der Lage, aus einer Menge von Daten mit vielen Ausreißern, die Parameter für ein passendes Modell zu schätzen. Anders als andere Schätzverfahren wie \enquote{Least-Median} oder \enquote{M-Schätzer}, welche aus der Statistik Literatur entnommen und entsprechend angepasst wurden, wurde RANSAC speziell für die Anwendung in der Computer Graphik entwickelt. Der Kern dieses Algorithmus ist das wiederholte Bestimmen eines Modells aus zufälligen und für das Modell ausreichenden Stichproben. Listing \ref{lst:ransac} zeigt den Verlauf des RANSAC Algorithmus. Die Anzahl der Iterationen \(N\) hängt dabei allein von dem Anteil der Ausreißer in den Messwerten ab. Daher sollte sie entsprechend gewählt werden, um die Wahrscheinlichkeit zu verringern, dass Ausreißer in den Stichproben enthalten sind. \citep{derpanis2010overview} \\

\begin{lstlisting}[mathescape,caption=Der RANSAC Algorithmus, label=lst:ransac]
Eingabe: Messwerte $P$, Modelltoleranz $e$, maximale Iterationen $N$
Ausgabe: Modell $m$, Unterstützende Messwerte $P_m$

1. Wähle zufällig so viele Stichproben aus den Messwerten $P$,
   wie nötig sind, um das Modell zu bestimmen
2. Bestimme aus den gewählten Stichproben das Modell $m$
3. Ermittle die Anzahl der Messwerte $P$, die mit einer 
   entsprechenden Toleranz $e$ das ermittelte Modell $m$ 
   unterstützen
4. Wenn prozentual genügend Messwerte aus $P$ das Modell $m$ 
   unterstützen, ermittle aus den unterstützenden Messwerten 
   $P_m$ durch lineare Regression erneut das finale Modell 
   $m$ und terminiere
5. Wiederhole die Schritte 1-4 $N$ mal
\end{lstlisting} 

Um mit dem RANSAC Algorithmus Ebenen in einer Punktewolke bestimmen zu können, werden pro Iteration drei Stichproben \(A\), \(B\) und \(C\) gewählt. Das Ebenenmodell, hier in der Hesse Normalform mit dem Normalenvektor \(\vec{n}\) und dem Abstand zum Koordinatenursprung \(d\), lässt sich dabei durch die Gleichung \ref{eq:normalform} bestimmen.

\begin{equation}\label{eq:normalform}
\vec{n} =\left|\left| \vec{AB} \times \vec{AC}\right|\right|
\qquad
\vec{D} = \vec{A} \cdot \vec{n}
\qquad
d = D_1 + D_2 + D_3
\end{equation}

Um zu ermitteln ob ein Punkt \(P\) aus einer Messreihe die gefundene Ebene \(\left[\vec{n_E}, d_E\right]\) unterstützt, wird die kürzeste Distanz \(d_P\) zwischen Punkt und Ebene wie in Gleichung \ref{eq:plane-distance} ermittelt.  Ein entsprechender Toleranzwert für die Distanz \(d_{min}\), im gezeigten RANSAC Algorithmus \(e\) genannt, wird später bei der Umsetzung abhängig vom Rauschen des Tiefensensors gewählt. 

\begin{equation} \label{eq:plane-distance}
d_P = n_1*P_1+n_2*P_2+n_3*P_3-d_E \qquad support_{d_P} = d_P < d_{min}
\end{equation}

Um das finale Modell der Ebene zu ermitteln, und somit die Varianz des Abstands der Punkte zur Ebene zu minimieren, wird mit Hilfe der unterstützenden Punkte \(P_{support}=\left[x,y,z\right]\) eine lineare Regression durchgeführt. Diese Mittelt ein Ebenenmodell \(E=\left[a,b,c\right]\) aus den zuvor ermittelten Punkten mit Hilfe des \enquote{least squares} Schätzverfahren. Für eine Ebene versucht man die Funktion \(G(a,b,c)\) aus Gleichung \ref{eq:least-squares} zu minimieren. Hierzu muss das lineare Gleichungssystem in Gleichung \ref{eq:least-squares-solution}  gelöst werden. \citep{Regre94:online}

\begin{equation} \label{eq:least-squares}
z = ax + by + c   \qquad G(a, b, c) = \sum {\left(z_i - ax_i - by_i - c\right)}^2
\end{equation}

\begin{equation} \label{eq:least-squares-solution}
\begin{bmatrix}
\sum x_i^2 & \sum x_iy_i & \sum x_i\\ 
\sum x_iy_i  & \sum y_i^2  & \sum y_i \\ 
\sum x_i & \sum y_i  & n
\end{bmatrix}
*
\begin{bmatrix}
a\\ 
b\\ 
c
\end{bmatrix}
=
\begin{bmatrix}
\sum x_iz_i\\ 
\sum y_iz_i\\ 
\sum z_i
\end{bmatrix}
\end{equation}

\subsection{Bestimmung der Ebenenausbreitung}

Nachdem die Ebene und die korrespondierenden Punkte zur Ebene gefunden wurden, muss noch die Ausbreitung der Fläche bestimmt werden, da die Ebene in Hesse Normalform lediglich die Position \(\vec{n} * d\) und Ausrichtung \(\vec{n}\) festhält. \citet{PlanarSurfaceMapping} nutzt hierfür die konvexe Hülle der korrespondierenden Punkte und trianguliert diese. Um das performant umzusetzen, kann man sich hier die Eigenschaft der Ebene zu Nutzen machen und die dreidimensionalen Punkte durch Parallelprojektion als zweidimensionale Punkte auf die Ebene projizieren. Denn die Algorithmen für die Triangulation haben im zweidimensionalen eine deutlich besseres Laufzeitverhalten. \\

Nach der Triangulation können die Ecken der gefundenen Polygone jeweils zurück projiziert werden. Die Gleichungen \ref{eq:projection2d} und \ref{eq:projection3d} bilden die Projektion der Punkte wobei \(R_{\vec{n}to\vec{z}}\) der Rotationsmatrix zwischen dem Normalenvektor \(\vec{n}\) und der Z-Achse \(\vec{z}\) entspricht.\\

\begin{equation} \label{eq:projection2d}
p_{2d} = (p_{3d} - (\vec{n}*d)) * R_{\vec{n}to\vec{z}}
\end{equation}
\begin{equation} \label{eq:projection3d}
p_{3d} = (p_{2d} * R_{\vec{n}to\vec{z}}^{-1}) + (\vec{n}*d)
\end{equation}

\subsubsection{Convex Hull Algorithmus}

Für die Berechnung der konvexen Hülle wird der Graham Scan nach \citet{graham1972efficient} genutzt. Dieser Algorithmus besitzt eine Laufzeit von \(O(n \log n)\) und gilt als einer der populärsten Algorithmen für die Berechnung der konvexen Hülle. Andere Ansätze besitzen dabei ein ähnliches oder schlechteres Laufzeitverhalten. Gestartet wird der Algorithmus mit der Menge aller Punkte \(P\) der Ebene und mit einem Startpunkt \(P_0\), der Bestandteil der konvexen Hülle ist. Hierzu wird meist der Punkt mit dem niedrigsten \(y\) Faktor gewählt. (\(P_0=P_{min(y)}\)) Listing \ref{lst:graham-scan} zeigt den Verlauf des Algorithmus des Graham Scans. Dabei wird in Abbildung \ref{fig:convexhull} nochmals die erste Sortierung und das Unterscheidungskriterium für die Sortierung als auch für die Aussortierung der Punkte verdeutlicht. \citep{convexHull} \\

\begin{figure}
  \centering
	\includegraphics[width=0.9\textwidth]{content/images/methods/convexhull.png} 
  \caption{Sortierung der Punkte nach Winkel zum Startpunkt (links) mit dem Unterscheidungskriterium (rechts). Übernommen von \citet{convexHull}}
  \label{fig:convexhull}
\end{figure}

\begin{lstlisting}[mathescape,caption=Graham Scan Algorithmus, label=lst:graham-scan]
Eingabe: Menge der Punkte $P$, außen liegender Punkt $P_0$
Ausgabe: Punkte der konvexen Hülle

$i$ = 0
sortiere nach dem Winkel zu $P_0$
solange $i$ <= $|P|$
    wenn $\measuredangle P_{i-1} P_{i}$ > $\measuredangle P_{i-1} P_{i+1}$, also $P_i$ rechts von  $\vec{P_{i-1} P_{i+1}}$ liegt
        inkrementiere $i$
    ansonsten
        entferne $P_i$ aus $P$
        dekrementiere $i$
    
\end{lstlisting} 


\subsubsection{Triangulation}

Nachdem die konvexe Hülle bestimmt wurde, müssen die Punkte dieses Pfades noch trianguliert werden. Hierfür wird eine Delaunay-Triangulation mit Hilfe des Sweep-Line Algorithmus von \citet{domiter2008sweep} verwendet. Unter einer Delaunay-Triangulation versteht man zunächst einmal eine Art Constraint, der sogenannten Umkreisbedingung, die besagt, dass ein Kreis, der alle drei Punkte eines gefundenen Polygons durchzieht, keine weiteren Punkte beinhalten darf. \\

Die Sweep-Line Methode ist ein generelles geometrisches Vorgehen, bei dem eine vertikale imaginäre Linie eine Menge geometrischer Objekte von links nach rechts passiert. 

 ... to be continued

\subsection{Clustering der aufgenommenen Punkte}

Wie im Algorithmus \ref{lst:planeReconstruction} zu erkennen wird das zuvor beschriebene Vorgehen für die planare Rekonstruktion immer pro Cluster eines Cluster-Pools durchgeführt. Dadurch werden pro Durchgang des Algorithmus nur ein Bruchteil der gesammelten Punkte rekonstruiert, was wiederum eine Rekonstruktion in Echtzeit möglich macht. Außerdem verhindert das Clustering das Bilden von konvexen Hüllen über Ebenen, die in Zwischenbereichen nicht mit genügend Punkten unterstützt werden. Dieses Problem ist in Abbildung \ref{fig:clustering} links zu sehen.\\

\begin{figure}
  \centering
	\includegraphics[width=1.0\textwidth]{content/images/methods/clustering.png} 
  \caption{Links: Ebenenrekonstruktion ohne Clustering. Rechts: Rekonstruktion mit K-Mean Clustering.}
  \label{fig:clustering}
\end{figure}

Getestet wurde hier das K-Mean Clustering, Agglomeratives Clustering und einfaches räumliches Clustern mit Hilfe eines Octrees. Das K-Mean Clustering hat, wie in Abbildung \ref{fig:clustering} rechts zu erkennen, gute Ergebnisse für die Aufteilung einer Ebenen geliefert, benötigt aber zuvor eine feste Anzahl von Clustern. Agglomeratives Clustering, getestet mit dem euklidischen Distanzmaß, würde zwar die Anzahl der Cluster dynamisch bestimmen, ist jedoch zu aufwändig für eine echtzeit Rekonstruktion. Gute Ergebnisse liefert wiederum ein einfaches räumliches Clustern mit einem Octree. Das bietet zudem den Vorteil, dass diese Datenstruktur direkt als Speicherort der Aufgenommenen Punkte und Ebenen dienen kann. \\

Ein Octree ist zunächst eine Datenstruktur, die wie ein Baum mit beliebiger Tiefe aufgebaut ist und pro Knoten Acht Kinder besitzt. Dabei repräsentiert ein Knoten ein Würfel, der durch seine Kinder in Acht Kind-Würfel aufgeteilt wird. Durch diese räumliche Aufteilung ergeben sich verschiedene Vorteile gegenüber linearen Datenstrukturen. So müssen Bereiche zum festhalten räumlicher Informationen im Octree nur dann allokiert werden, wenn diese Bereiche auch verwendet werden. Speichert man Punkte in den untersten Knoten eines Octrees kann man durch eine Tiefenbegrenzung beim Zugriff auf den Baum ein sehr effektives Downsampling der Punkte vornehmen. Zuletzt entstehen durch die Knoten einer bestimmten Tiefe ein Cluster, zu denen in diesem Fall Punkte bei einer Aufnahme hinzugefügt werden und für eine weitere Verarbeitung extrahiert werden können.\\




% Evaluation
\chapter{Evaluation}

In diesem Kapitel sollen die beschriebenen und prototypisch implementierten Verfahren zur Überlagerung gegenübergestellt werden, um anhand eines direkten Vergleichs eine objektive Aussage über die Qualität der Ergebnisse treffen zu können. Hierzu wird im ersten Teil das Vorgehen zum Testen vorgestellt, welches darauf folgend mit allen Verfahren umgesetzt wird. Hiernach werden die daraus resultierenden Ergebnisse gegenübergestellt.

\section{Statische Testszenen}

Zum Vergleich der Verfahren wurden zwei statische Szenen gewählt, in denen das Project Tango Gerät nicht bewegt wird und dadurch allen Kandidaten den selben sensorischen Inhalt bietet. Diese Wahl wurde getroffen, um eine zuverlässige und reproduzierbare Informationsquelle für das Gerät zu schaffen. Denn die Reproduktion eines bewegten und dynamischen Szenarios ist für alle zu vergleichenden Verfahren nur sehr schwer möglich. \\

Eine Idee für ein dynamisches Testszenario war es, alle sensorischen Informationen der Hardware einmal aufzunehmen und eine reproduzierbare simulierte Umgebung dieser Daten zu schaffen. Technologien wie das Robot Operating System (ROS) würden dies ermöglichen, jedoch übersteigt der Aufwand den zeitlichen Rahmen dieser Arbeit. Auch wenn die Firma Bosch eine exemplarische Implementation\footnote{Tango Output to Rosbag Files - https://goo.gl/hhnciZ} für die Aufnahme aller Daten in ROS demonstriert, sind die implementierten Verfahren zu sehr in den API Zyklen der Project Tango Schnittstelle involviert, um diese in kurzer Zeit auf eine Desktop Umgebung zu portieren.\\

Die erste gewählte Szene, welche in Abbildung \ref{fig:static-scene} links zu sehen ist, beinhaltet einen Hocker, in Form eines einfachen  Würfels, und einen Sitzball. Der Sitzball wurde gewählt, um auch runde Formen zur Tiefenaufnahme zu testen, welche gegebenenfalls für die Verfahren schwerer zu rekonstruieren sind. Das Project Tango Gerät ist etwas höher in einem Stativ plaziert. Das virtuelle Objekt wird, wie in Abbildung \ref{fig:static-scene} rechts, zwischen die beiden realen Objekte plaziert, sodass es von beiden Seiten durch die realen Objekte überdeckt wird. \\

\begin{figure}[h]
  \centering
	\includegraphics[width=1.0\textwidth]{content/images/evaluation/static-scene.png} 
  \caption{Links: Erste statische Szene mit einem Hocker und einem Sitzball. Rechts: Platzierung des virtuellen Objekts. }
  \label{fig:static-scene}
\end{figure}

Die zweite gewählte Szene, welche in Abbildung \ref{fig:plant-scene} links zu sehen ist, soll als Herausforderung die Überdeckung von komplexeren Strukturen testen. Sie besteht daher aus einer Pflanze, die sich, wie rechts im Bild zu sehen, vor dem virtuellen Objekt befindet. Auch hier befindet sich das Project Tango Gerät in einem Stativ, damit sich die Position während der Tests durch Motion Tracking nicht ändert. \\

\begin{figure}[h]
  \centering
	\includegraphics[width=1.0\textwidth]{content/images/evaluation/plant-scene.png} 
  \caption{Links: Zweite statische Szene mit einer Pflanze im Vordergrund. Rechts: Platzierung des virtuellen Objekts hinter der Pflanze. }
  \label{fig:plant-scene}
\end{figure}

Für beide Szenen sollen alle Kombinationen der Verfahren getestet werden. Somit ergeben sich sechs verschiedene Kombinationen, in denen die Pointcloud Projektion, die TSDF Rekonstruktion und die Ebenen Rekonstruktion jeweils mit und ohne der Anwendung des Guided Filter auf das Tiefenbild getestet werden. Für alle Kombinationen soll ein gerendertes Bild und ein Tiefenbild mit dem virtuellen Objekt festgehalten werden. Zur Auswertung werden die jeweils gerenderten Ergebnisbilder \(p\) mit einem manuell zugeschnittenem Ergebnisbild  \(q\) für jeden Pixel \(i\) verglichen. Für diese Gegenüberstellung wird die Summe der absoluten Bilddifferenz wie in Gleichung \ref{eq:diff} bestimmt.

\begin{equation} \label{eq:diff}
d = \sum_i |p_i-q_i|
\end{equation}

\section{Durchführung der Tests}

Die beiden Testszenen konnten wie beschrieben aufgebaut und mit allen Verfahren durchgetestet werden. Hierzu wurde mit der \enquote{Android Debug Bridge} (adb\footnote{Android Debug Bridge - http://goo.gl/ffH51x (01.03.16)}) eine Videoaufnahme gestartet, in der im Prototypen für jede Szene alle Verfahren durchgeschaltet wurden. Die Verfahren mussten dabei sehr schnell gewechselt werden, um einen potentiellen Drift von Project Tangos Motion Tracking so minimal wie möglich zu halten. Denn diese Bewegungen würden die Ergebnisse stark beeinträchtigen. \\

Abbildung \ref{fig:static_occlusion} und \ref{fig:plant_occlusion} im Anhang zeigen jeweils die aus dem Video extrahierten Bildausschnitte. Die obere Reihe zeigt die drei tiefengenerierenden Verfahren ohne den Guided Filter und die untere Reihe jeweils mit dem Filter. In der untersten Reihe sind jeweils die Projektion der generierten Primitiven in der Szene zu sehen, um sich eine Vorstellung der Rekonstruktion machen zu können.

\section{Auswertung der Ergebnisse}

Der Vergleich der Ergebnisse, welcher mit dem bereits beschriebenen Bilddifferenz Ansatz aus der Gleichung \ref{eq:diff} durchgeführt werden soll, wurde mit Hilfe der OpenCV Bibliothek durchgeführt und ist in Listing \ref{lst:compare} zu finden. Für jede Szene wurde ein Referenzbild manuell konstruiert, welches dem idealen Ergebnis entsprechen soll. Diese Referenzbilder sind in Abbildung \ref{fig:reference} zu finden. Mit dem Referenzbild wurden alle zuvor passend zugeschnitten Grafiken einer Szene verglichen. \\

\begin{lstlisting}[mathescape,caption=Python Implementierung der Bilddifferenz, label=lst:compare, language=Python]
from cv2 import *
from os import listdir
from os.path import isfile, join

reference_path = "reference.png"
reference = imread(reference_path)
reference = cvtColor(reference, COLOR_BGR2GRAY)
all_images = [f for f in listdir("./") 
    if isfile(join("./", f)) and f.endswith(".png")]
    
for img_path in all_images:
    img = imread(img_path)
    img = cvtColor(img, COLOR_BGR2GRAY)
    result = absdiff(reference, img)
    imwrite("result_" + img_path, result)
    difference = sumElems(result)
    print str(int(result[2])) + "\t" + result[0] + "\t" + result[1]
\end{lstlisting}

\begin{figure}[h]
  \centering
	\includegraphics[width=.8\textwidth]{content/images/evaluation/reference.png} 
  \caption{Manuell konstruierte Referenzbilder der idealen Überlagerung in Szene 1(links) und 2(rechts).}
  \label{fig:reference}
\end{figure}

Der gezeigte Skript zum Verglich der Ergebnisbilder mit den Referenzbildern ergibt neben den Differenzwerten auch die absolute Differenzbilder für jedes Verfahren. Diese Ergebnisbilder sind auch im Anhang in Abbildung \ref{fig:static_occlusion_results} zu finden.

\begin{table}[]
\centering
\begin{tabular}{@{}rrrr@{}}
\toprule
                      & \textbf{\begin{tabular}[c]{@{}r@{}}Pointcloud \\ Projektion\end{tabular}} & \textbf{\begin{tabular}[c]{@{}r@{}}TSDF \\ Rekonstruktion\end{tabular}} & \textbf{\begin{tabular}[c]{@{}r@{}}Ebenen \\ Rekonstruktion\end{tabular}} \\ \midrule
\textbf{Szene 1}      & 495.695                                                                    & 560.210                                                                  & 247.327                                                                    \\
\textbf{Szene 1 + GF} & 463.612                                                                    & 79.642                                                                & 166.589                                                                    \\
\textbf{Szene 2}      & 190.473                                                                   & 462.780 & 295.008 \\
\textbf{Szene 2 + GF} & 22.148                                                                 & 339.086 & 264.154 \\ \bottomrule
\end{tabular}
\caption{Distanzwert zwischen dem Referenzbild und den Ergebnisbildern in der jeweiligen Szene}
\label{my-label}
\end{table}

% Fazit
\chapter{Fazit} \label{sec:conclusion}

\section{Evaluation}

Die implementierten Verfahren haben gezeigt, dass mit dem Ansatz der Tiefenbild Überdeckung von \citet{wloka1995resolving} eine Echtzeit Überdeckung virtueller Objekte auf der mobilen Project Tango Hardware erfolgreich umgesetzt werden kann. Dabei wird im Folgenden auf jedes Verfahren sowie ihrer Vor und Nachteile im Kontext der anderen Verfahren und auf Basis der durchgeführten Tests eingegangen. 

\subsection*{Pointcloud Projektion}

Die Überlagerung durch die Pointcloud Projektion bietet gegenüber den anderen Verfahren den Vorteil, dass sie zu jeder Zeit eine dynamische und aktualisierte Repräsentation der Tiefe der Szene liefert und somit auch Änderungen in der Szene sofort berücksichtigt. Außerdem ist das Verfahren nicht auf Clustergrößen beschränkt und kann dadurch auch auch komplexe Strukturen abbilden. Zu erkennen ist dies bei der Testszene zwei, bei der die Bilddifferenz zum optimalen Ergebnis am geringsten ist, obwohl die Pflanze eine komplexe Struktur besitzt.

Dadurch, dass die Pointcloud von Project Tango Fehler in Form von Ausreißern und einem gewissem Rauschen enthalten kann, spiegeln sich diese Fehler auch in der berechneten Projektion wieder. Das führt dazu, dass zum Beispiel die Kante einer realen Überlagerung durchgehend in Bewegung ist und Ihre Struktur mit jedem neuen Tiefenbild variiert. Ein weiteres Problem dieser Technik ist, dass sie, dadurch dass sie sich nur auf einen Datensatz pro Aufnahme bezieht, die Tiefe nur innerhalb des Messbereichs des Tiefensensors repräsentieren kann. Dadurch können Überlagerungen von realen Objekten innerhalb der ersten 50 Zentimeter und ab vier Metern nicht mehr bestimmt werden.

\subsection*{Ebenen Rekonstruktion}

Die Ebenen Rekonstruktion löst die Schwächen der Pointcloud Projektion in dem Sinne, dass sie die Ungenauigkeit der Tiefeninformation als eine Oberflächen Approximation mit Hilfe von RANSAC in Form von Ebenen abbildet. Hierdurch werden Ausreißer ignoriert und auch das Rauschen wird durch eine lineare Regression gemittelt. Zusätzlich ermöglicht das Vorgehen der Ebenen Rekonstruktion eine kontinuierliche Verbesserung, indem alle bereits aufgenommenen Pointclouds in die aktuelle Rekonstruktion einfließen. Auch wenn dieser Rekonstruktionsansatz durch den Octree eine Rekonstruktion in einer groben Struktur, den Clustern des Octrees, durchführt, erhält das Verfahren durch das Ermitteln der konvexen Hülle pro gefundener Ebene einen gewissen Detailgrad, um auch schwierige planare Strukturen abbilden zu können. Die gemessenen Ergebnisse der Szenen eins und zwei spiegeln diese positive Eigenschaft wieder und zeigen, dass diese Art der Rekonstruktion auch komplexe Szenen für dieses Testszenario gut abbilden kann.

In einem manuellen dynamischeren Test mit Bewegungen weißt dieses Verfahren jedoch einige Schwächen auf. So sind durch die begrenzte Dichte der Pointcloud Lücken zwischen den Ebenen zu sehen, die zwar von Aufnahme zu Aufnahme kleiner werden aber üblicherweise nicht komplett schließen. Das führt dazu, dass zum Beispiel große Oberflächen, die ein virtuelles Objekt überlagern, das Objekt vereinzelt nicht aussparen, da keine Tiefe an den Stellen durch Lücken zwischen den Ebenen vorhanden sind. Außerdem ist das Verfahren nur bedingt in der Lage runde Strukturen wie den Sitzball aus Szene eins zu rekonstruieren. Diese Fehler werden besonders dann sichtbar, wenn man sich um diesen Ball dreht und er eine Überlagerung mit den Ecken und Kanten der Ebenen aus der Rekonstruktion auf ein virtuelles Objekt ausübt. Neben der fehlenden Unterstützung für runde Konturen besitzt dieses Verfahren keine Möglichkeit, Messungen zu revidieren wenn reale Objekte in der Szene verändert wurden oder ein Drift Fehler von Project Tango auftritt.

\subsection*{TSDF Rekonstruktion}

Wie zu erwarten liefert Chisel als eine TSDF Implementierung, aufgrund der großen Voxelgröße, nicht die Qualität, die zum Beispiel ein KinectFusion liefern kann. Dafür ist es performant genug, um als eine CPU Implementierung eine Echtzeit Rekonstruktion auf der mobilen Project Tango Hardware zu ermöglichen. Genau wie die Ebenen Rekonstruktion bietet Chisel den Vorteil eine Rekonstruktion pro Tiefenbild anzureichern und stetig zu verbessern. Dadurch können Überlagerungen auch außerhalb des Messbereichs des Tiefensensors ermöglicht werden. Anders als bei der Ebenen Rekonstruktion generiert die TSDF Rekonstruktion stets eine geschlossene Oberfläche. Außerdem können runde Strukturen festgehalten werden, wodurch der in Szene eins stehende Sitzball abgebildet werden kann. 

Die große Voxelgöße führt jedoch dazu, dass wie in beiden getesteten Szenen zu erkennen, die Strukturen der Rekonstrutkion sehr grob ausfallen und die Differenzergebnisse ohne eine Filterung auf einen hohen Fehler hinweisen. Auch wenn Chisel nicht in der Lage ist so detailierte Kantenabbildungen wie die Ebenen Rekonstruktion zu generieren, besitzt Chisel einen Vorteil: Durch den Space Carving Mechanismus können Rekonstruktionen wieder revidiert werden. Das hilft dabei den Problematiken des Drift Effekts von Project Tango entgegenzuwirken. Außerdem könnte durch eine GPU Implementierung auch eine Echtzeit Rekonstruktion mit deutlich kleinerer Voxelgröße und dadurch höherem Detailgrad realisiert werden.


\subsection*{Guided Filter}

Der Guided Filter war in den Tests häufig in der Lage selbst grobe Fehler im Tiefenbild an die Kanten der Farbbilder anzugleichen und somit auch, wie in den Messergebnissen zu erkennen, den Differenzwert zum Optimum zu reduzieren. Jedoch führte der Einsatz des Filters zu deutlichen Performanceeinbußen, denn der Filterprozess selbst benötigt im Durchschnitt \(220\) ms. Der Einsatz von OpenCV erschwert zudem den Einsatz des Filters für die Echtzeit Umsetzung, da die Bildebene pro Bild aus dem OpenGL Framebuffer raus und wieder rein geladen werden muss. Dieser Prozess benötigt zusätzliche \(80\) ms, was die Wiederholrate der prototypischen Implementierung auf 3 Herz reduziert. 

Zusätzlich sind unter gewissen Umständen, bei denen ein virtuelles Objekt nah an der Oberfläche eines realen Objekts, aber immer noch räumlich hinter dem realen Objekt liegt, variable Artefakte aufgefallen, an denen das eigentlich überlagerte virtuelle Objekt durchschimmert. Dieser Effekt ist in Abbildung \ref{fig:artifacts} zu erkennen. Neben den eigentlichen Kanten für die Überlagerung im Farbbildes werden auch Kanten von flachen Strukturen berücksichtigt. Im Bild zu sehen, beeinflusst das Muster vom Würfel das resultierende Tiefenbild soweit, dass eine fehlerhafter Überlagerung nach der Filterung stattfindet. 
 
\begin{figure}[h]
  \centering
	\includegraphics[width=1.0\textwidth]{content/images/artifacts.png} 
  \caption{Fehlerhafte Überdeckung bei der Anwendung des Guided Filters. Links: Reales Objekt mit Rekonstruktion. Mitte: Tiefenbild. Rechts: Sichtbare Fehler nach Guided Filter.}
  \label{fig:artifacts}
\end{figure}

Angewendet auf die Tiefeninformationen der Pointcloud Projektion konnte der Filter in der komplexeren zweiten Szene nahezu das Optimum der Überlagerung erreichen. Schwieriger war jedoch der Einsatz bei der runden Kontur in der ersten Szene, wo der Filter nicht in der Lage war, den initialen groben Fehler vom Tiefensensor zu revidieren. Das selbe Verhalten ist auch bei der Ebenen Rekonstruktion in Szene eins zu beobachten. Besonders beachtenswert ist die Tatsache, dass bei den zu weit reichenden Tiefeninformationen der TSDF Rekonstruktion, in der Mitte der Messergebnisse von Szene eins, ein etwa gleichgroßer Fehler durch die Filterung behoben werden konnte. Eine weitere Besonderheit, die während der Tests beobachtet werden konnte, ist dass der Filter in der Lage war bei der Anwendung auf die Ebenen Rekonstruktion, die Lücken zwischen den Ebenen unkenntlich zu machen. Somit würde dieser Filter ein Nachteil dieses Ansatzes lösen.

\section{Einsatz der Verfahren}

Grundsätzlich ist festzuhalten, dass sich die Pointcloud Projektion ohne den Guided Filter, trotz des erreichbaren Detailgrads, höchstens in einzelnen Bildaufnahmen für eine Überlagerung in Augmented Reality eignet, da das Rauschen der Eingangsdaten durchgehend sichtbar ist. Außerdem ist die Sichtweite auf die Erreichbarkeit des Tiefensensors des aktuellen Ausschnitts begrenzt. Auch die Ebenen Rekonstruktion ist bedingt geeignet, da Lücken zwischen den Ebenen zu erkennen sind, die die Illusion von AR zerstören würde. Auch wenn die TSDF Rekonstruktion durch Chisel nach den statischen Testszenen oft mit Fehler behaftet sind, existieren entschiedene Vorteile gegenüber den anderen Vorgehensweisen. Denn durch Chisel werden geschlossene Flächen gebildet, welche sich dynamisch der Szene anpassen können. Betrachtet man zudem den Einsatz von Chisel in größeren Flächen, wie in Räumen oder sogar im ganzen Gebäude, fällt der gemessene Fehler weniger erkennbar aus.

Angenommen es gäbe die Möglichkeit den Einsatz des Guided Filters für jedes Verfahren in Echtzeit zu ermöglichen, so würde die Pointcloud Projektion durchaus Anwendung finden. Es könnte zum Beispiel in einer AR Applikation genutzt werden, die sich nur in einem bestimmten Sichtbereich bewegt, und die eine gewisse komplexe und dynamische Szene bedienen muss. Ausgehend von den Testergebnissen als Entscheidung zwischen der Ebenen Rekonstruktion und der TSDF Rekonstruktion mit dem Guided Filter würde, wie auch ohne Filter, Chisel die bessere Alternative sein.


\section{Ausblick}



Technologie
* Vorteile von Polygonen als Ausgangsbasis (Schatten, Interaktion)
* Filter Umsetzung in im Vertex/Fragment Shader
* 

Project Tango
* Bilateral Filter in API während Arbeit erschienen => Guided Filter wäre der
* API Änderungen - Zugang zu optimierten Chisel Umsetzungen
* Consumer Phase mit Lenovo Deployment
* Alternative Tiefensensoren LFC


	
	


\appendix
\listoffigures
\lstlistoflistings        
%\chapter{Ergebnisaufnahmen}
\begin{sidewaysfigure}[h]
  \centering
	\includegraphics[width=1.0\textwidth]{content/images/evaluation/static_occlusion.png} 
  \caption{Ergebnisaufnahmen aus der ersten statischen Szene}
  \label{fig:static_occlusion}
\end{sidewaysfigure}

\begin{sidewaysfigure}[h]
  \centering
	\includegraphics[width=1.0\textwidth]{content/images/evaluation/plant_occlusion.png} 
  \caption{Ergebnisaufnahmen aus der zweiten statischen Szene}
  \label{fig:plant_occlusion}
\end{sidewaysfigure}

\begin{sidewaysfigure}[h]
  \centering
	\includegraphics[width=1.0\textwidth]{content/images/evaluation/static_occlusion_results.png} 
	\includegraphics[width=1.0\textwidth]{content/images/evaluation/spacer.png} 
	\includegraphics[width=1.0\textwidth]{content/images/evaluation/plant_occlusion_results.png} 
  \caption{Differenzbilder der Verfahren in ersten (oben) und zweiten Szene (unten)}
  \label{fig:static_occlusion_results}
\end{sidewaysfigure}



\addcontentsline{toc}{chapter}{Bibliography}

\bibliography{main}
\bibliographystyle{natdin} 

\end{document}

