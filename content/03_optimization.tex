\chapter{Verfahren zur Optimierung von Augmented Reality durch Tiefeninformationen}

Anhand der Klassifizierung von Project Tango bezüglich Augmented Reality aus Abschnitt \ref{sec:classification_project_tango}, kann bereits festgehalten werden, dass sich das Gerät, durch die Verfügbarkeit von intrinsischen und extrinsischen Kameraparametern, für den Einsatz von Augmented Reality gut eignet. Um jedoch eine für den Betrachter eine effektive und optimierte Augmented Reality Anwendung umsetzen zu können, benötigt man laut \citet{azuma2001recent} die Möglichkeit mehr Informationen über relevante Objekte im realen Raum zu ermitteln. Diese Informationen könnten zum Beispiel eine Tiefenüberdeckung von virtuellen Objekten durch reale Objekte ermöglichen, eine Interaktion mit realen Objekten durch die Anwendung von physikalischen Modellen und Kollisionen generieren oder die Funktion anhand semantischer Einordnung der Umgebung variieren. \\

Zur zuletzt erwähnte Kontextsensitivität existieren viele Ansätze, basierend auf optischen Merkmalen der Umgebung. So kann zum Beispiel ein optisches Tracking von realen Objekten, wie von \citet{lee2008hybrid} beschrieben, umgesetzt werden. Project Tango nutzt bereits optische Merkmale um \enquote{Motion Tracking} und \enquote{Area Learning} umzusetzen. Wären diese Merkmale für den Nutzer als Schnittstelle verfügbar, könnte man mit Diesen solche kontextsensitiven Anwendungen umsetzen. Der Fokus soll hier jedoch nicht auf den optischen Merkmalen liegen. Die Information, auf die sich hier fokussiert werden sollen sind die Tiefeninformationen, die Project Tango durch \enquote{Depth Perception} in Form einer Pointcloud liefern kann.\\

Die folgenden Abschnitte widmen sich den Verfahren zur möglichen Realisierung dieser Optimierungen durch Tiefeninformationen. Zunächst wird beschrieben, wie eine mögliche Interaktion mit Oberflächen der realen Welt durch eine Pointcloud ermöglicht werden kann. Hiernach wird ein Verfahren beschrieben, was ermöglicht eine Überdeckung virtueller Objekte durch Depth Maps umzusetzen. Diese Idee kann, wie dort näher erläutert, weitergeführt werden, indem existierende Verfahren zur Echtzeit Rekonstruktion von Oberflächen näher behandelt werden. Außerdem wird eine Idee als Verfahren beschrieben, die im laufe dieser Arbeit entwickelt worden ist, um basierend auf verschiedensten Algorithmen eine planare Echtzeitrekonstruktion zu ermöglichen. Zuletzt soll näher auf die Möglichkeit eingegangen werden, die resultierenden Tiefeninformationen aus der Pointcloud oder aus dem Rendering der Rekonstruktion mit Hilfe der Bildaufnahmen der Farbkamera, zu verbessern. \\

Die hier beschriebenen Verfahren zur Tiefenwahrnehmung für die Anwendung in Augmented Reality werden im Laufe dieser Arbeit umgesetzt und gegenübergestellt. Um einen AR Prototypen implementieren zu können wird noch eine Möglichkeit zur Interaktion mit den Tiefeninformationen benötigt. Das Kapitel \ref{sec:ar-depth-interaction} geht dabei auf eine einfache tiefensensitive Auswahlgeste ein.\\


\section{Verdeckung durch Pointcloud Projektion} \label{sec:pc-projection}

Die erste und weniger aufwändige Idee eine Überlagerung in Augmented Reality zu realisieren, ist die Überführung der Pointcloud in eine Depthmap, die wiederum in den Rendering Prozess mit eingebracht wird. Das Verfahren von \citet{kanbara2000stereoscopic} verfolgt einen ähnlichen Weg mit einer Stereokamera und einer video see-through Displaytechnologie in Form eines Head-Mounted Display. Wie in Abbildung \ref{fig:stereo-depth-map} zu erkennen, bestimmen sie mit Hilfe der Stereokamera Tiefeninformationen, die das gerenderte virtuelle Objekt an den Positionen ausspart, an denen Tiefeninformationen im Vordergrund vorliegen. \\

\begin{figure}[h]
  \centering
	\includegraphics[width=1.0\textwidth]{content/images/methods/stereo-depth-map.png} 
  \caption{Visualisierung des Methode zur Vedeckung durch Depth Maps. Übernommen von \citet{kanbara2000stereoscopic}}
  \label{fig:stereo-depth-map}
\end{figure}

Anders als im oberen Verfahren wird eine Depthmap anhand der vorhandenen Pointcloud aus dem Infrarot Sensor von Project Tango generiert. Dadurch, dass die intrinsischen Kameraeigenschaften der Farbkamera zur Verfügung stehen, welche die Selben der Infrarotkamera sind, können wir einen Punkt \(P = [X, Y, Z]\) der Pointcloud mit der Gleichung \ref{eq:projection} auf die Bildebene überführen. Hier stehen die Variablen \(f_{x/y}\) für Brennweite und \(c_{x/y}\) für den Bildmittelpunkt auf der Bildebene. \citep{Tango90:online}

\begin{equation}\label{eq:projection}
x = X / Z * f_x + c_x
\qquad
y = Y / Z * f_y + c_y
\end{equation}

An dieser Stelle der Depthmap wird nun eine Punkt mit einem Graustufenwert, entsprechend der Entfernung \(|\overrightarrow{PO_{cam}}|\) vom Punkt \(P\) zum Kameraursprung \(O_{cam}\) gezeichnet. Der Farbwert richtet sich dabei nach der Konvention des Rendering Frameworks und den Informationen über die vordere und hintere Clippingebene.\\

Die Auflösung des Tiefensensors der rroject Tango Hardware ist mit \(320x180\) zu \(1280x720\) vier mal kleiner als die Auflösung der Farbkamera ist. Zusätzlich die Dichte der Pointcloud zum eigentlichen Sensor geringer als Ihre eigene Auflösung. So würde man bei einer Auflösung von \(320x180 = 57600\) Tiefenpunkte erwarten. Project Tango liefert jedoch unter guten Bedingungen durchschnittlich \(1700\) Tiefenpunkte. Aufgrund der kleineren Auflösung und der geringeren Informationsdichte werden die gezeichneten Punkte auf der Depthmap hier mit einem Radius von 4 Bildpunkten gezeichnet. \\

Nachdem die Depthmap generiert wurde, kann zum Ausschluss der Pixel der virtuellen Objekte, welches sich hinter einem reellen Objekt befinden, der Z-Buffer Algorithmus aus Kapitel \ref{sec:z-buffer} angewendet werden. Hierzu wird vor dem virtuellen Rendering der Z-Buffer mit den generierten Informationen aus der Depthmap gefüllt. Pixel der virtuellen Objekte werden somit nicht gerendet, wenn nähere Tiefeninformationen realer Objekte and dieser Position vorliegen. 




\section{Polygon Rekonstruktion}

\subsection{Marching Cubes als Echtzeit Umsetzung}



\subsection{Possion Reconstruction}

\subsection{Greedy Projection Triangulation}



\section{Planare Rekonstruktion} \label{sec:plane-reconstruction}

Wie bereits in Kapitel \ref{sec:polygon_reconstruction} beschrieben, lässt sich das Problem der Optimierung von Augmented Reality mit Hilfe von Tiefeninformationen auf eine Echtzeit Rekonstruktion zurückführen. Im Gegensatz zur Rekonstruktion komplexer Oberflächen, mit dem vorgestellten TSDF Verfahren, soll hier eine Idee näher erläutert werden, die eine Rekonstruktion allein auf Ebenen ermöglichen soll. \citet{yang2010plane} erwähnt hierzu, dass Ebenen in fast allen künstlichen Umgebungen zu finden sind und auf Grund ihrer vorteilhaften geometrischen Eingenschaften in verschiedensten Computer Vision Verfahren verwendet werden. Daher gibt es viele Forschungsarbeiten, Methoden und Algorithmen um aus verschiedensten Informationsquellen ein Ebenenmodell zu extrahieren.\\

Das \enquote{Simultaneous Localization and Mapping} (SLAM) Verfahren von \citet{trevor2012planar} detektiert Ebenen mit dem RANSAC Algorithmus. RANSAC bietet gegenüber anderen Algorithmen zur Ebenen Detektion den Vorteil, ein Modell auch bei vielen Ausreißern performant zu ermitteln. Agglomeratives Clustering und Region Growing wie von \citet{feng2014fast} beschreiben, eignet sich auf Grund des Ausgabeformats von Project Tango nicht direkt, da es keine organisierte Point Cloud ausgibt und die Daten durch Reflektionen und Löchern mit Fehlern behaftet sind. \\

Das selbst zusammengestellte Verfahren zur Ebenendetektion besteht daher aus folgenden Komponenten. Wie in dem Ansatz von \citet{yang2010plane} wird der RANSAC Algorithmus auf Würfeln ausgeführt, welche hier durch den Einsatz eines Octrees bestimmt werden. Eine gefundene Ebene in einem Würfel wird wie im SLAM Verfahren von \citet{trevor2012planar} persistiert. Die Repräsentation der Ebene \(P\) wird dort wie in Gleichung \ref{eq:plane} festgehalten. Dabei handelt es sich um den Normalenvektor \(\vec{n}\) und der Distanz zum Ursprung \(d\) der Hesse Normalform einer Ebene, sowie der Punkte der konvexen Hülle \(H\). \citet{trevor2012planar} erläutern, dass die konvexe Hülle in der Repräsentation festgehalten wird, um eine sukzessive Verbesserung einer Ebene nach mehreren Messdurchläufen zu ermöglichen. So werden die Punkte der konvexen Hülle pro Messvorgang kombiniert, damit die Ebenenausbreitung auch außerhalb des Sichtfeldes beibehalten werden kann. \\

\begin{equation} \label{eq:plane}
P=\left[\vec{n}, d, H\right] \qquad H=\vec{h_1}, \vec{h_2}, \ldots  \vec{h_n}
\end{equation}

Die einzelnen Schritte des Vorgehens werden in den folgenden Absätzen näher erläutert. Ein Grober Ablauf des Vorgehens wird in Listing \ref{lst:planeReconstruction} einmal zusammengefasst und in Pseudocode beschreiben. Als Cluster sind im Listing die Nodes eines Octrees gemeint, dessen Anwendung auch in Absatz \ref{sec:cluster} näher beschrieben wird.

\begin{lstlisting}[mathescape,caption=Planare Echtzeit Rekonstruktion, label=lst:planeReconstruction]

Eingabe: Octree $O$, Anzahl der zu suchenden Ebene in Clustern $N$
Ausgabe: Polygonpunkte $T_{Gesamt}$

für jedes Cluster $C$=[$C_{Punkte}$, $C_{Ebenen}$] aus $O$
    führe $N$ mal aus
        bestimme Ebene [$\vec{n}$, $d$, $P$] mit RANSAC aus $C_{Punkte}$
        wenn keine Ebene mit genügend $P$ gefunden wurde
            nächstes Cluster (continue)
        wenn Ebene mit [$\vec{n}$, $d$, $H_{alt}$] in $C_{Ebenen}$ existiert	
            füge die konvexe Hülle $H_{alt}$ zu $P$ hinzu	
        bestimme die konvexe Hülle $H_{neu}$
        trianguliere $H_{neu}$ zu $T_{Ebene}$
        $T_{Gesamt}$ += $T_{Ebene}$
        $C_{Ebenen}$ += [$\vec{n}$, $d$, $H_{neu}$]
        $C_{Punkte}$ - $P$
\end{lstlisting}


\subsection{RANSAC zur Ebenendetektion} \label{sec:ransac}

Um mit dem RANSAC Algorithmus, beschrieben in Kapitel \ref{sec:ransac-theory}, Ebenen in einer Punktewolke bestimmen zu können, werden pro Iteration drei Stichproben \(\vec{A}\), \(\vec{B}\) und \(\vec{C}\) gewählt, die zur Bestimmung einer Ebene ausreichen. Das Ebenenmodell, hier in der Hesse Normalform mit dem Normalenvektor \(\vec{n}\) und dem Abstand zum Koordinatenursprung \(d\), lässt sich dabei durch die Gleichung \ref{eq:normalform} bestimmen.

\begin{equation}\label{eq:normalform}
\vec{n} =\left|\left| \vec{AB} \times \vec{AC}\right|\right|
\qquad
d = \vec{A} \cdot \vec{n}
\end{equation}

Um zu ermitteln ob ein Punkt \(\vec{P}\) aus einer Messreihe die gefundene Ebene \(\left[\vec{n}, d\right]\) unterstützt, wird die kürzeste Distanz \(d_P\) zwischen Punkt und Ebene wie in Gleichung \ref{eq:plane-distance} ermittelt.  Ein entsprechender Toleranzwert für die Distanz \(d_{min}\), im gezeigten RANSAC Algorithmus \(e\) genannt, wird später bei der Umsetzung abhängig von der Ungenauigkeit des Tiefensensors gewählt. 

\begin{equation} \label{eq:plane-distance}
d_P = \vec{n} \cdot \vec{P} - d \qquad support_{d_P} = d_P < d_{min}
\end{equation}

Um das finale Modell der Ebene zu ermitteln, wie im Ursprünglichen RANSAC Algorithmus in Punkt 7. beschrieben, und somit die Varianz des Abstands der Punkte zur Ebene zu minimieren, wird mit Hilfe der unterstützenden Punkte \(P_{s}=\left[x,y,z\right]\) eine lineare Regression durchgeführt. Diese Mittelt ein Ebenenmodell \(E=\left[\vec{n}, d\right]\) aus den zuvor ermittelten Punkten mit Hilfe der Methode der kleinsten Quadrate. \citet{hoppe1992surface} nutzen hierfür für eine ähnliche Problemstellung die Eigenwert Dekomposition der Kovarianzmatrix der Punkte \(P_{s}\) zum Zentrum \(\vec{p_{c}}\). In Gleichung \ref{eq:centroid} wird das Zentrum aus den unterstützenden Punkten \(P_{s}\) bestimmt. Gleichung \ref{eq:covarianz} zeigt die Bestimmung der Kovarianzmatrix \(CV\) im Bezug zum Zentroid, in der \(\otimes\) für das dyadische Produkt\footnote{Wenn \(\vec{a}\) und \(\vec{b}\) die Komponenten \(a_i\) und \(b_j\) beinhalten, resultiert aus \(\vec{a} \otimes \vec{b}\) eine Matrix mit den Komponenten \(a_ib_j\) an der \(ij\) Position. \citep{hoppe1992surface}} steht.

\begin{equation} \label{eq:centroid}
\vec{p_{c}} = \frac{\sum_{n=0}^{|P|} P_{n}}{|P|}
\end{equation}

\begin{equation} \label{eq:covarianz}
CV = \sum_{n=0}^{|P|} ( \vec{p_{n}}- \vec{p_{c}}) \otimes ( \vec{p_{n}}- \vec{p_{c}})
\end{equation}

Wendet man nun auf der Kovarianzmatrix \(CV\) die Eigenwert Dekomposition an, erhält man die Normale \(\vec{n}\) aus dem Eigenvektor \(||\vec{v_i}||\) mit dem kleinsten Eigenwert \(\lambda_i\). Somit würde bei \(\lambda_1 \geqq \lambda_2 \geqq \lambda_3\) die Zuweisung \(\vec{n} = ||\vec{v_3}||\) folgen. Die Distanz zum Ursprung \(d\) entspricht dem Kreuzprodukt aus dem Zentroiden \(\vec{p_c}\) und der neu gewonnen Normalen \(\vec{n}\). \citep{hoppe1992surface} \\

\subsection{Bestimmung der Ebenenausbreitung}

Nachdem die Ebene und die korrespondierenden Punkte zur Ebene gefunden wurden, muss noch die Ausbreitung der Fläche bestimmt werden, da die Ebene in Hesse Normalform lediglich die Position \(\vec{n} * d\) und Ausrichtung \(\vec{n}\) festhält. \citet{PlanarSurfaceMapping} nutzt hierfür die konvexe Hülle der korrespondierenden Punkte und trianguliert diese. Wie diese dort genau bestimmt wurde ist nicht beschrieben. \\

Um diese Bestimmung performant umsetzen zu können, kann man sich hier die Eigenschaft der Ebene zu Nutzen machen und die dreidimensionalen Punkte durch Parallelprojektion als zweidimensionale Punkte auf die Ebene projizieren. Denn die Triangulation ist nach dem Erhalten der zweidimensionalen konvexen Hülle, wie im Listing \ref{lst:triangulation} beschrieben, direkt bestimmbar. Nach der Triangulation können die Ecken der gefundenen Polygone jeweils zurück projiziert werden. Die Gleichungen \ref{eq:projection2d} und \ref{eq:projection3d} bilden die Projektion der Punkte wobei \(R_{\vec{n} \rightarrow \vec{z}}\) der Rotationsmatrix zwischen dem Normalenvektor \(\vec{n}\) und der Z-Achse \(\vec{z}\) entspricht.\\

\begin{equation} \label{eq:projection2d}
p_{2d} = (p_{3d} - (\vec{n}*d)) * R_{\vec{n} \rightarrow \vec{z}}
\end{equation}
\begin{equation} \label{eq:projection3d}
p_{3d} = (p_{2d} * R_{\vec{n} \rightarrow \vec{z}}^{-1}) + (\vec{n}*d)
\end{equation}

\begin{lstlisting}[mathescape,caption=Bestimmung der Ebenenausbreitung und Triangulation, label=lst:triangulation]

Eingabe: Unterstützende Ebenenpunkte aus RANSAC $P$
         Transformation $R_{\vec{n} \rightarrow \vec{z}}$
Ausgabe: Polygone $T_{Ebene}$

    Projiziere alle Punkte aus $P_s$ mit $R_{\vec{n} \rightarrow \vec{z}}$ zu $P_{2d}$
    Bestimme die konvexe Hülle $H$ aus $P_{2d}$ mit Graham Scan
    starte mit leerer Menge $P_{2dmesh}$
    für $i$ von $0$ bis $|H| - 2$
        füge $H_i$ zu $P_{2dmesh}$ hinzu
        füge $H_{i+1}$ zu $P_{2dmesh}$ hinzu
        füge $H_{i+2}$ zu $P_{2dmesh}$ hinzu
    Projiziere alle Punkte aus $P_{2dmesh}$ mit $R_{\vec{n} \rightarrow \vec{z}}^{-1}$ zu $P_{3dmesh}$
   
\end{lstlisting}


\subsection{Clustering der aufgenommenen Punkte} \label{sec:cluster}

Wie im Listing \ref{lst:planeReconstruction} zu erkennen wird das zuvor beschriebene Vorgehen für die planare Rekonstruktion immer pro Cluster eines Cluster-Pools durchgeführt. Dadurch werden pro Durchgang des Algorithmus nur ein Bruchteil der gesammelten Punkte rekonstruiert, was wiederum eine Rekonstruktion in Echtzeit möglich macht. Außerdem verhindert das Clustering das Bilden von konvexen Hüllen über Ebenen, die in Zwischenbereichen nicht mit genügend Punkten unterstützt werden. Dieses Problem ist in Abbildung \ref{fig:clustering} links zu sehen, in welcher eine blaue Ebene Rekonstruiert wird, die sich über einen Durchgang ohne vorhandene Punkte streckt.\\

\begin{figure}[h]
  \centering
	\includegraphics[width=1.0\textwidth]{content/images/methods/clustering.png} 
  \caption{Links: Ebenenrekonstruktion ohne Clustering. Rechts: Rekonstruktion mit K-Mean Clustering.}
  \label{fig:clustering}
\end{figure}

Getestet wurden hier das K-Mean Clustering, Agglomeratives Clustering und einfaches räumliches Clustern mit Hilfe eines Octrees. Das K-Mean Clustering hat, wie in Abbildung \ref{fig:clustering} rechts zu erkennen, gute Ergebnisse für die Aufteilung einer Ebenen geliefert, benötigt aber zuvor eine feste Anzahl von Clustern. Agglomeratives Clustering, getestet mit dem euklidischen Distanzmaß, würde zwar die Anzahl der Cluster dynamisch bestimmen, ist jedoch zu aufwändig für eine Echtzeit Rekonstruktion. \\

Gute Ergebnisse liefert wiederum ein einfaches räumliches Clustern mit einem Octree, welcher die aufgenommenen Punkte direkt in Knoten des Baums zuweist. Das bietet zudem den Vorteil, dass diese Datenstruktur direkt als Speicherort der Aufgenommenen Punkte und Ebenen dienen kann. Außerdem entspricht dies dem Vorgehen für die Anwendung von RANSAC auf Würfeln, welches von \citet{yang2010plane} beschrieben wurde. \\


\section{Tiefenanpassungen durch Farbbilder}

Aus allen zuvor beschriebenen Verfahren werden letztendlich Tiefeninformationen, in Form von geometrischen Primitiven oder Punkten im Raum gewonnen. Diese werden passend zur aktuellen Kameraposition als Tiefenbild gerendert und füllen den Z-Buffer für eine entsprechende Überdeckung der virtuellen Objekte. Auf Grund von Sensorungenauigkeiten und den daraus resultierenden größeren Auflösungen der Rekonstruktionsverfahren können dabei fehlerhafte Tiefeninformationen im Z-Buffer gelangen, die zu Fehlern bei der Bestimmung der Überdeckung führen können. Dieses Problem ist am Beispiel der Pointcloud Projektion aus Kapitel \ref{sec:pc-projection} in Abbildung \ref{fig:pc-noise} zu erkennen. \\

\begin{figure}[h]
  \centering
	\includegraphics[width=1.0\textwidth]{content/images/methods/pc-noise.png} 
  \caption{Überdeckung mit einfacher Pointcloud Projektion. Links: Resultat der Überdeckung. Mitte: Darstellung des Tiefepuffers. Rechts: Darstellung der Pointcloud.}
  \label{fig:pc-noise}
\end{figure}

Die Reduktion von Ungenauigkeiten im Tiefenbild könnte durch einen Gaußschen Weichzeichner erreicht werden. Dieser würde jedoch die Kanten im Farbbild nicht berücksichtigen und somit fehlerhafte Tiefen Gradienten erzeugen. \citet{newcombe2011kinectfusion} wenden einen sogenannten \enquote{Bilateralen Filter} in ihrem KinectFusion System an, bevor sie die Tiefeninformationen in die TSDF Repräsentation einfließen lassen. Dieser Filter von \citet{tomasi1998bilateral} ermöglicht das Weichzeichnen ohne dabei die Kanten im Bild zu übergehen, bezieht sich jedoch nur auf das selbe Bild, auf dem der Filter angewendet wird. \\

\citet{liu2012guided} hingegen wenden einen sogenannten \enquote{Guided Filter} in Ihrem Verfahren zur Optimierung der der Tiefeninformationen für Kinect ähnliche Sensoren auf das Tiefenbild an. Dieser Filter von \citet{he2010guided} ist in der Lage, auf Grundlage eines anderen Leitbildes ein Weichzeichnen durchzuführen, ohne dabei die Kanten des Leitbildes zu überschreiten. 

\section{Interaktion durch Raypicking} \label{sec:ar-depth-interaction}

Wie in Kapitel \ref{sec:ar-interaction} beschrieben, bedarf es bei der Umsetzung von Augmented Reality Systemen ein anderes Interaktionsparadigma. Auch wenn die Entwicklung der neuen Tablet und Smartphone Geräte durch Touchscreens eine neue Interaktionsform eingeführt haben, ist sie in den meisten Fällen auf einer zweidimensionalen Ebene beschränkt. In der Entwicklung von Virtual Reality oder voll virtuellen Anwendungen und Spielen wird oft für die Auswahlgeste der Raypicking Mechanismus verwendet, um eine zweidimensionale Interaktion im dreidimensionalen Raum zu ermöglichen. Darüber hinaus gibt es verbesserte semantische Interaktionsformen basierend auf einer zweidimensionalen Toucheingabe, wie von \citet{elmqvist2008semantic} beschrieben.\\

Hier soll aber zunächst eine Raycasting Variante für Augmented Reality Anwendungen umgesetzt werden, die nicht von einem kompletten Modell in Form von Polygonen oder anderen Primitiven der realen Umgebung ausgeht. Diese AR Interaktion ermöglicht, anhand der Tiefeninformationen, das passende Positionieren von virtuellen Objekten im realen Raum und lässt sich auf weitere Interaktionen erweitern. Voraussetzung für die folgende Umsetzung, ist die entsprechende Kalibrierung und Gleichstellung der intrinsischen Kameraparametern und der Verfügbarkeit der extrinsischen Parameter der realen Kamera. \\

\begin{figure}[h]
  \centering
	\includegraphics[width=1.0\textwidth]{content/images/methods/interaction.jpg} 
  \caption{Raypicking Visualisierung. Übernommen von \citet{gluUn11:online}}
  \label{fig:interaction}
\end{figure}

Als Erstes wird ein Strahl erzeugt, der durch die Position der virtuellen Kamera und durch den jeweils ausgewählten Punkt auf der Viewingplane läuft. Den Ursprung der virtuellen Kamera bestimmt dabei Google Tangos \enquote{Motion Tracking}. Der gewählte beziehungsweise berührte Punkt auf dem Touchscreen wird dabei zunächst von Pixeln in das Verhältnis \(\left[-1,1\right]\) des Punktes umgerechnet. Hiernach wird die Projektion auf die Viewingplane durch eine Multiplikation mit \(T\) aus Gleichung \ref{eq:unprojection} rückgängig gemacht. Die Gerade aus den beiden Punkten kann danach genutzt werden, um den Schnitt von Objekten vor der Kamera zu ermitteln. \citep{OpenG86:online} 

\begin{equation} \label{eq:unprojection}
T  = MV_{ModelView}^{-1} * P_{Projection}^{-1}
\end{equation}

Angewendet auf die Tiefeninformation aus Tangos \enquote{Depth Perception} wird die Punktewolke, wie in Abbildung \ref{fig:interaction} anstelle der Objekte, vor die Kamera projiziert. In den projizierten Punkten wird danach der entsprechende Punkt gesucht, welcher sich am nächsten am zuvor bestimmten Strahl befindet. Durch diese beschriebenen Schritte kann der Nutzer mit einer zweidimensionalen Geste einen Punkt in der Tiefe bestimmen. Diese Methode kann zudem um die Ermittlung einer Ebenennormalen erweitert werden. Hierzu werden um den selektierten Punkte Nachbarn gefunden, mit denen durch RANSAC eine Ebene ermittelt wird. Durch die ermittelte Ebenennormale können virtuelle Objekte dann nicht nur an die ausgewählte Stelle positioniert werden, sondern auch an der realen Oberflächenausrichtung ausgerichtet werden.

