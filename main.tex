\documentclass[12pt]{support/thcolognethesis} 

\usepackage{amssymb}

\title{Optimierung von Augmented Reality Anwendungen durch die Berücksichtigung von Tiefeninformationen mit Googles Project Tango}

\degree{Masterthesis}

\author{Steffen Tröster}

\college{
	Technischen Hochschule Köln\\
    Ingenieurwissenschaftliches Zentrum\\
    Fakultät für Informations-,\\
    Medien- und Elektrotechnik}

\course{Technische Informatik (Master)}

\company{inovex GmbH}  

\firstExaminer{Prof. Dr. Hubert Randerath}
\firstExaminerLocation{Technische Hochschule Köln}
\secondExaminer{Prof. Dr. Martin Eisemann}
\secondExaminerLocation{Technische Hochschule Köln}

\degreedate{Köln, im \monthyeardate\today}
	
\begin{document}

\baselineskip=18pt plus1pt

\setcounter{secnumdepth}{3}
\setcounter{tocdepth}{3}

\maketitle                 

% Die eidesstattliche Erklärung mit Unterschrift

\subsubsection*{Fakultät für Informations-, \\Medien- und Elektrotechnik}

\section*{Masterarbeit}

    
Thema: \\ 
\textbf{Optimierung von Augmented Reality Anwendungen \\
durch die Berücksichtigung von Tiefeninformationen mit \\
Googles Project Tango} \\

\noindent Name, Vorname: Tröster, Steffen \\
Anschrift: Münchener Straße 29 \\
Matrikelnummer: 11075591 \\
Studiengang: Technische Informatik (Master) \\

\noindent Erstprüfer:	Prof. Dr. Hubert Randerath \\
Zweitprüfer: Prof. Dr. Martin Eisemann \\

\noindent Anfertigungszeitraum: 10.11.2015 \\
Fertigstellung/Abgabedatum: 11.04.2016 \\



\section*{Erklärung der Urheberschaft}

Ich erkläre an Eides statt, dass ich die vorgelegte Abschlussarbeit 
selbständig und ohne fremde Hilfe verfasst, andere als die angegebenen 
Quellen und Hilfsmittel nicht benutzt und die den benutzten Quellen 
wörtlich oder inhaltlich entnommenen Stellen als solche kenntlich 
gemacht habe.


\vspace{3cm}
\hspace{2cm} Ort, Datum \hfill Unterschrift \hspace{2cm}
\begin{abstract}
\setlength{\parskip}{1em}

Project Tango ist eine neue mobile Plattform des Google Advanced Technology and Project (ATAP) Teams, die in der Lage ist, Bewegungsverfolgung, Tiefenwahrnehmung und Umgebungswiedererkennung auf Smartphones und Tablets anbieten zu können. Durch die kontinuierliche Bestimmung der relativen Geräteposition eignet sich die Plattform besonders für dreidimensionale Augmented Reality (AR) Anwendungen. Die Illusion dieser AR Anwendungen wird besonders dann gestört, wenn sich reale Objekte in einer Szene räumlich vor virtuellen Objekten befindet und diese virtuellen Objekte nicht entsprechend ausgespart werden. 

Diese Arbeit stellt daher drei Überdeckungsverfahren vor, mit denen diese Überlagerung der virtuellen Objekte mit Hilfe der Tiefenwahrnehmung von Project Tango und des Z-Buffer Algorithmus realisiert werden kann. Die Tiefeninformationen für den Z-Buffer werden hierfür zum einen direkt aus den Sensordaten und alternativ mit einer TSDF Rekonstruktion und einer selbst zusammengestellten Ebenenrekonstruktion bestimmt. Außerdem wird auf einen zusätzlichen Ansatz eingegangen, der zur Verbesserung dieser Tiefeninformationen die Bildinformationen der Farbkamera durch den Guided Filter berücksichtigt. Diese Mechanismen werden im Laufe der Arbeit prototypisch umgesetzt und gegenübergestellt. 

\setlength{\parskip}{0em}
\end{abstract}
\selectlanguage{english}
\begin{abstract}
\setlength{\parskip}{1em}

Project Tango is a new mobile platform by Google’s Advanced Technology and Projects (ATAP) Teams, which brings Motion Tracking, Depth Perception, and Area Learning to smartphone and tablet devices. With its Motion Tracking technology, Project Tango is suitable for precise three dimensional augmented reality (AR) applications. The illusion of the model projection in these AR applications is often disrupted when some real objects in the scene standing in front of virtual projections, which are not getting occluded.

This thesis is comparing three occlusion mechanisms, which can solve the virtual object occlusion with Project Tangos depth perception by applying the Z-Buffer algorithm. The Z-Buffer is filled either by the direct sensor data, by a TSDF reconstruction method or by a self combined and implemented plane based reconstruction. Additionally a guided image filtering approach is applied to the depth map to interpolate according to the edges of the RGB image frame. These mechanisms are going to be implemented and compared.

\setlength{\parskip}{0em}
\end{abstract}
\selectlanguage{ngerman}




\begin{romanpages}         
\tableofcontents            
\end{romanpages}          

\setlength{\parskip}{1em}

% Einleitung
\chapter{Einleitung}

Project Tango ist eine neue mobile Plattform des Google Advanced Technology and Projects (ATAP) Teams, welche Bewegungsverfolgung, Tiefenwahrnehmung und Umgebungswiedererkennung auf mobilen Endgeräten realisiert.

\begin{quotation}
\enquote{Project Tango combines 3D motion tracking with depth sensing to give your mobile device the ability to know where it is and how it moves through space.}  \citep{Proje19:online}
\end{quotation}

Diese Verfügbarkeit ermöglicht viele verschiedene neue Einsatzmöglichkeiten auf mobilen Endgeräten wie Smartphones und Tablets. Typische Einsatzszenarien dieser Plattform sind die Indoor Navigation, die Vermessung der Umgebung sowie andere typische Anwendungen für Virtual und Augmented Reality. Der Fokus dieser Forschungsarbeit liegt hier in dem Anwendungsbereich Augmented Reality (AR). \\

Um eine erfolgreiche AR Anwendung umsetzen zu können, müssen die Kameraeigenschaften, wie Brennweite, Verzerrung und die Position der Kamera zu jeder Zeit bekannt sein. Sensoren wie Kompass, INS (Trägheitsnavigationssystem) oder GPS können zwar eine grobe Lokalisierung ohne bekannte Merkmale im Raum ermöglichen, führen aber langfristig zu Fehlern, wenn keine optischen Referenzen gegeben sind. Mit Hilfe von der Bewegungsverfolgung durch Project Tango kann diese Lokalisierung der Kamera und somit die korrekte Positionierung von virtuellen Objekten im Raum deutlich zuverlässiger und ohne vordefinierte Merkmale im Raum realisiert werden. Project Tango eignet sich daher sehr gut für die Umsetzung und den Einsatz von AR Anwendungen.\\

Ein sinnvoller Einsatz von Augmented Reality besteht darin, virtuelle Objekte in eine echte Szene zu projizieren. Dabei überlagert die Projektion des virtuellen Objekts das aktuelle Kamerabild oder den aktuellen Sichtbereich und erwirkt dadurch den Anschein, dass sich das virtuelle Objekt wirklich in der Szene befindet. Dieser Effekt funktioniert solange erfolgreich, bis ein reales Objekt sich räumlich vor das virtuelle Objekt bewegt und die zu erwartende Überlagerung des virtuellen Objekts nicht erfolgt. \\

Die Project Tango Plattform bietet die Möglichkeit Tiefeninformationen mit Hilfe eines Tiefensensors für den aktuellen Sichtausschnitt zu bestimmen. Hierdurch können Interaktionen oder Darstellungen in Augmented Reality Anwendung näher an die echten räumlichen Gegebenheiten angepasst werden. Es existieren zum Beispiel prototypische Anwendungen, in denen virtuelle Markierungen passend an echten Objekten im virtuellen Raum positioniert werden können, indem sie auf die aktuellen Tiefeninformation des Sichtbereichs zurückgreifen.\\

Diese Arbeit versucht die Fragestellung zu beantworten, durch welche Verfahren mit Hilfe der Tieninformationen von Project Tango, automatisch und in Echtzeit Überdeckung virtueller Objekte mit realen Objekten in einer Augmented Reality Szene realisiert werden können. Dabei soll Project Tango als Autonomes System betrachtet werden, welches diese Problemstellung selbstständig und mit den eingeschränkten Ressourcen dieser mobilen Plattform lösen soll.\\


\section{Vorgehen}




%Die erste Fragestellung richtet sich dem Thema, wie man performant und automatisiert geometrische Primitiven in einer Szene finden kann. Dazu soll zunächst eine Literaturrecherche bezüglich bekannter Methoden und Algorithmen durchgeführt werden, welche darauf folgend anhand gestellter Kriterien entsprechend evaluiert werden. Für diese Evaluation ist auch die Erstellung einer einheitlichen, zum AR Anwendungsfall passenden, Testumgebung denkbar. Letztendlich soll eine prototypische Implementierung dieser Primitiven Detektion erstellt werden, auf der im späteren Verlauf aufgebaut werden kann.\\

%Später soll bestimmt werden ob und wie sich diese gefundenen Primitiven im Nachhinein oder im Verlauf einer Anwendung selbstständig verbessern oder optimieren lassen, oder ob die Basisdaten (Pointcloud) entsprechend verbessert werden können. Hierzu soll näher untersucht werden ob die Bildinformationen aus der Project Tango Kamera dabei helfen können durch zum Beispiel Kantenerkennung eine Optimierung vorzunehmen. Die hieraus gewonnen Erkenntnisse sollen genutzt werden, um den zuvor implementierten Prototypen weiter zu verbessern.\\




% Thematische Vorbemerkung
\chapter{Thematische Vorbemerkung} \label{sec:thema}

Im ersten Teil dieses Kapitels werden zunächst einmal die Grundlagen zu Augmented Reality beschrieben, wie diese Technologie einzuordnen ist, welche technischen Anforderungen ein Augmented Reality System hat und wo typische Einsatzszenarien liegen. Außerdem wird hier auf den aktuellen Stand der Forschung bezüglich der Bestimmung von Überdeckungen in einem Augmented Reality System eingegangen. Hiernach wird näher auf Googles Project Tango eingegangen, welche Konzepte angewendet werden und wie diese Technologie im Bereich Augmented Reality einzuordnen ist. 


\section{Augmented Reality}

Augmented Reality (AR) ist eine Klasse aus dem Realitäts-Virtualitäts-Kontinuum von \cite{milgram1995augmented}, welches in Abbildung \ref{fig:virtual-continuum} abgebildet ist. Diese Klasse beschreibt die Darstellungen von realen und virtuellen Informationen in einer Repräsentationsform, wobei hier reale und virtuelle Objekte in einer realen Umgebung kombiniert dargestellt werden können. Diese virtuellen Objekte sind in der realen Umgebung idealerweise fest lokalisiert und fügen sich somit in das reale Erscheinungsbild ein. Typischerweise sind AR Anwendungen interaktiv und stellen die virtuellen Objekte in Echtzeit und dreidimensional in der realen Welt dar. Für die Definition von AR Anwendungen gibt es zudem keine Limitierung für die Darstellungstechnologie, wie zum Beispiel das Project Tango Tablet oder ein Head-Mounted-Display. AR beschränkt sich zudem nicht auf den angesprochenen Sinn - so sind zum Beispiel AR Anwendungen mit visueller, taktiler oder sogar olfaktorischer Umsetzung möglich.

\begin{figure}
  \centering
	\includegraphics[width=0.85\textwidth]{content/images/theory/virtual-continuum.png} 
  \caption{Vereinfachte Darstellung des Realitäts-Virtualitäts-Kontinuum von \citet*{milgram1995augmented}}
  \label{fig:virtual-continuum}
\end{figure}


Virtual Reality (VR) oder auch Virtual Environment hingegen kapselt sich von der realen Umgebung ab und bietet Interaktionen in reinen virtuellen Umgebungen. Diese rein virtuelle Darstellung konnte sich im Gegensatz zu Augmented Reality deutlich schneller entwickeln, da die technologischen Anforderungen an VR deutlich geringer sind. \citep{van2010survey}

\subsection{Technische Anforderungen}

Dieser Abschnitt widmet sich den technischen Anforderungen an Augmented Reality, indem die potentiellen Display Technologien beschrieben werden, mögliche Trackingverfahren zur Ermittlung der Betrachtungsposition erläutert werden und auch die Systeme behandelt werden, mit denen ein Nutzer mit den virtuellen Darstellungen interagieren kann.

\subsubsection{Display Technologie}

Der erste wichtige Teil der technologischen Anforderungen an AR sind visuelle Anzeigen (visual displays), die neben der Möglichkeit eines dreidimensionalen Renderings, welches hier aus der Virtual Reality vorausgesetzt werden soll, weitere Charakteristika mit sich bringen. Nach \citet{van2010survey} lassen sich diese Technologien zunächst in je drei Arten der Darstellung und Positionierung unterteilen.

Die einfachste und günstigste Art der visuellen Darstellung in AR ist \enquote{video see-through}, wodurch die reale Umgebung durch eine Video Aufnahme ersetzt wird und die virtuellen Objekte digital in die Videoaufnahme gerendert werden. Das bietet die Möglichkeit Objekte aus der realen Umgebung zu entfernen oder zu ändern oder, anhand der Luminanz Information vom Video, das Rendering der virtuellen Objekte entsprechend an die Realität anzupassen. Anwendung findet diese Technologie typischerweise in Tablets, Smartphones oder Head-Mounted-Displays.

Die nächste Möglichkeit zur Darstellung ist \enquote{optical see-through}. Hier werden die virtuellen Objekte durch transparente Spiegel in das Sichtfeld des Betrachters gebracht. Anders als bei \enquote{video see-through} bleibt die reale Auflösung für die visuelle Aufnahme des Betrachters gleich und es können zudem nur Latenzprobleme bei dem Rendering der virtuellen Objekte und nicht bei der Darstellung der realen Umgebung auftreten. Auf der anderen Seite besteht bei dieser Technologie das Problem, dass die Darstellung von virtuellen Objekten nicht kräftig genug ist, um die reale Umgebung auf Grund von der transparenten Darstellungsoberfläche komplett auszublenden. Typische Geräte dieser Technologie sind Headmounted Displays wie Google Glass\footnote{Googel Glass - \url{https://developers.google.com/glass/} (23.02.2016)} oder stationäre Geräte wie der HoloDesk\footnote{HoloDesk - \url{http://research.microsoft.com/en-us/projects/holodesk/} (23.02.2016)}.

Die dritte Möglichkeit ist die projizierte Darstellung, in der die Augmented Reality Überlagerung auf die realen Objekte projiziert werden. Diese Darstellung ermöglicht die Abdeckung vom gesamten Sichtfeld des Betrachters, benötigt aber eine entsprechende Kalibrierung oder eine Strukturwahrnehmung bei Umgebungsänderungen.

Neben der Art der Darstellung können die Display Technologien laut \citet{azuma2001recent} anhand ihrer Positionierung klassifiziert werden. Man unterscheidet zwischen am Kopf befestigten Displays (head-mounted), tragbaren Displays (hand-held) und räumlich positionierten Displays. Zu jeder dieser Displayarten gibt es wiederum unterschiedliche technische Umsetzungen mit ihren spezifischen Vor- und Nachteilen bezüglich ihrer Anwendungsszenarien.

\subsubsection{Tracking Technologien}

Um eine virtuelle Projektion im realen Raum auf nicht stationären Displaytechnologien zu realisieren, müssen die Position und gegebenenfalls relative Positionsänderungen des Displays bestimmt werden, auch \enquote{augmented reality registration} genannt. Man spricht dabei üblicherweise von den \enquote{six degrees of freedom (6DOF)}, der Position im Raum (x, y, z) und der Orientierung (yaw, pitch, roll). Da bei vielen Displaytechnologien auch die verwendete Kamera direkt mitgeführt wird, entsprechen diese Informationen meist auch den extrinsischen Kameraeigenschaften der AR Kamera.

Frühe Techniken für die Registrierung benötigten üblicherweise eine speziell vorbereitete Räumlichkeit, denn sie basierten auf mechanischen, magnetischen oder Ultraschallsensoren um die Position zu bestimmen. Diese Sensoren sind zwar noch im Einsatz und bilden auch den Grundstein für die AR und VR Forschung, sind aber praktisch gesehen zu komplex und aufwändig für die meisten Anwendungsfälle. \citep{van2010survey} 

Für ein grobes Positions-Tracking, vor allem auch außerhalb von Gebäuden wird GPS genutzt. Für großräumige Anwendung ist GPS, mit einer Varianz von 10-15 Metern und in Kombination mit einem Kompass, durchaus praktikabel. Als Beispiel reicht diese Genauigkeit aus, um sichtbare Flugzeuge oder Sterne visuell aufzubereiten. Innerhalb von Gebäuden basiert die grobe Positionierung laut \citet{van2010survey} oft auf verfügbaren Wifi Access Points oder RFID Markern. \citet{lamarca2005place} demonstrieren hierzu auch die Möglichkeit diese Idee für grobe Lokalisation außerhalb von Gebäuden einzusetzen.

Optische Tracking Verfahren, basierend auf Bildverarbeitung, bieten laut \citet{van2010survey} deutlich genauere Resultate als die zuvor beschriebenen Verfahren. Es gibt hier viele verschiedene sensorische Ansätze, ein optisches Tracking zu realisieren. Frühe Verfahren, wie die von \citet{dunston2008identification} oder \citet{narzt2006augmented}, nutzten Passmarker (fiducial marker) oder Licht emittierende Dioden (LED) in einem vordefinierten Modell, um zwischen aufgenommenen Bildern die Marker oder LEDs zu detektieren und zusammengehörige zwischen den Bildern zu finden, um daraus eine Kameratransformation zu berechnen. Neue Verfahren ohne Marker, wie das sogenannte \enquote{visual odometry} von \citet{nister2004visual}, nutzen Techniken zur Feature Detection und Matching, um gemeinsame Punkte zwischen aufgenommenen Bildern zu bestimmen und somit die Bewegungen zu berechnen.

Viele kommerzielle und erfolgreiche Tracking Verfahren beruhen jedoch auf hybriden Ansätzen, in denen die Informationen mehrerer Sensoren kombiniert werden, um potentielle Messfehler eines Sensors oder einer Methodik auszuschließen. So werden zum Beispiel Neigungssensor, Kompass und Gyroskop mit einem optischen Verfahren kombiniert, um ein Tracking der sechs Freiheitsgrade zu optimieren. Diese Erweiterung des optischen Verfahrens wird auch \enquote{visual-inertial odometry} genannt. \citep{van2010survey}

\citet{azuma2001recent} erwähnt an dieser Stelle auch die Kalibrierung der Sensoren, die für ein präzises Registrieren nötig ist. So müssen zum Beispiel die Linseneigenschaften der Kamera für optisches Tracking bekannt sein, damit die Verfahren mit Krümmungen, Verzerrungen und den perspektivischen Eigenschaften der Kamera umgehen können. Diese Informationen sind auch bei video see-through Displays für ein korrektes Projizieren der 3D Objekte wichtig. Zudem wird erwähnt, dass man Messfehlern oder Drifts der Position zum Beispiel unter Zuhilfenahme von Gyroscop Informationen entgegenwirken kann, indem man Ereignisse wie einen Schritt des Nutzers einfließen lässt. \citep{azuma2001recent} 

\subsubsection{Interaktions Technologien} \label{sec:ar-interaction}

Neben den Display und Tracking Technologien ist es notwendig dem Nutzer andere angemessene Interaktionsmöglichkeiten anzubieten, da in der Regel das klassische zweidimensionale WIMP Paradigma (Windows, Icons, Menus and Pointer) im dreidimensionalen Kontext von AR keine ausreichende Gebrauchstauglichkeit bietet. Dennoch müssen die Interaktionstechnologien in Augmented Reality die üblichen Interaktionen, welche unter anderem aus WIMP bekannt sind, unterstützen. Dazu gehören zum Beispiel das Auswählen, Positionieren und Drehen von virtuellen Objekten, das Zeichnen von Pfaden oder Flugbahnen, sowie die Eingabe von quantitativen Werten oder Texten. \citep{van2010survey} 

Frühe Augmented Realilty Systeme nutzten einfache Trackballs, Trackpads, Touchscreens oder Gyroskopmäuse für eine zweidimensionale Interaktion mit dem System. Später wurden dreidimensionale Äquivalente eingeführt, wie 3D Mäuse oder Stifte, die eine dreidimensionale Interaktion ermöglichen. Diese greifbaren Schnittstellen werden auch TUIs genannt (Tangible User Interface) und ermöglichen eine unidirektionale Interaktion mit dem System. Zudem wurden auch TUIs mit haptischen Feedback eingeführt, wie zum Beispiel die 3D Maus PHANTOM. \citep{van2010survey} 

Eine weitere Art der TUIs sind laut \citet{azuma2001recent} Gegenstände, mit denen der Nutzer natürlich interagieren kann und die vom System optisch erfasst werden, um die Positionsänderung der Objekte anhand von Markern oder anderen optischen Merkmalen zu bestimmen. Somit kann ein Nutzer, zum Beispiel, für die virtuelle Einrichtung eines Raums, die virtuellen Möbel mit Hilfe eines echten Gegenstands im Raum verschieben. 

Nicht taktile Systeme verwenden meist optische Aufnahmen, um Gesten der Hände, des gesamten Körpers oder die Blickrichtung des Nutzers erkennen zu können. Für die Realisierung werden Kameras am Körper oder im Raum verwendet. Außerdem ist es möglich, Spracherkennung in die Interaktion mit einfließen zu lassen, um eine möglichst authentische Interaktion zu bieten. Wie auch bei den Tracking Technologien existieren hierbei hybride Systeme, die verschiedene Interaktions Technologien kombinieren. \citep{van2010survey} 

\subsection{Anwendungsbereiche}

Über die Jahre haben Wissenschaftler immer mehr Bereiche identifiziert, die von der Anwendung von Augmented Reality profitieren können. \citet{van2010survey} nennt dazu als erstes Einsatzgebiet die persönliche Assistenz, in der AR Systeme eingesetzt werden können, um zum Beispiel mit Hilfe von Brillen (etwa der Google Glass) Namen der sichtbaren Personen anzuzeigen, die Navigation in unbekannten Regionen einzublenden oder beim Sightseeing kontextrelevante Informationen im Sichtfeld anzuzeigen. 

Neben der persönlichen Assistenz können auch Anwendungen in der Industrie laut \citet{van2010survey} von AR profitieren. Es lassen sich zum Beispiel virtuelle Designumgebungen umsetzen, die es ermöglichen, ein Auto in Lebensgröße zu gestalten. Auch bei der Fertigung und Konstruktion können den Arbeitern unterstützende Informationen angezeigt werden. So werden zum Beispiel zu erledigende Schweißstellen hervorgehoben oder der Plan zur Konstruktion entsprechend eingeblendet. Eine weitere Möglichkeit wäre es, für die Instandhaltung komplexer Maschinen dem Nutzer, über ein AR System eine Art Röntgenblick mit Hinweisen auf potentielle Schwachstellen bereitzustellen. Auch in der Rüstungsindustrie existieren Anwendungsgebiete für Augmented Reality Systeme. So können zum Beispiel Gefechte für eine Kampfausbildung besser simuliert werden. \citep{azuma2001recent} 

Für Anwendungsbereiche in der Medizin ist ein sehr genaues Tracking der Freiheitsgrade erforderlich, da AR in der Chirurgie und Behandlung von Patienten Anwendung findet. Erstellte Röntgenbilder oder Ultraschallbilder können hierdurch, anstatt auf einem separaten Monitor, direkt auf die entsprechende Körperstelle projiziert werden, wodurch gegebenenfalls eine genauere Untersuchung oder Behandlung möglich ist. \citep{van2010survey} 

Augmented Reality wird auch im Entertainment Sektor eingesetzt. Videoübertragungen von Sportereignissen werden heutzutage oft durch zusätzliche Informationen angereichert. So erhalten zum Beispiel American Football Spiele dynamische Spielfeldbegrenzungen. Auch die Werbeeinblendungen am Rand des Spielfelds können entsprechend dem Gebiet der Ausstrahlung ausgetauscht werden. \citep{azuma2001recent} 

Ein weiteres großes Anwendungsgebiet für Augmented Reality sind laut \citet{azuma2001recent} Computerspiele, in denen es möglich ist, in einer beliebigen Umgebung Objekte eines Spiels im Raum zu platzieren und mit ihnen entsprechend zu interagieren. Die natürlichere Interaktion gegenüber herkömmlichen Spielplattformen und die Nutzung in einer persönlichen Spielumgebung führt zu einem intensiveren Spielerlebnis. 

Auch in pädagogischen Bereichen, in Schulen oder Museen können AR Anwendungen eingesetzt werden. Zur Vermittlung von geometrischen oder mathematischen Grundlagen gibt es die Möglichkeit der kollaborativen und interaktiven Visualisierung von Körpern, an denen etwa Parameter manipuliert werden, um danach ihre Eigenschaften besser beobachten und nachvollziehen zu können. \citep{van2010survey} 

\subsection{Einschränkungen und Probleme}

Die frühen Augmented Reality Systeme sind auf Grund ihrer Größe sehr unhandlich und mobil daher nur mit großem Aufwand anwendbar. Durch die Verfügbarkeit mobiler und performanter Endgeräte ist ein mobiler Einsatz wiederum ermöglicht worden. Jedoch besitzen die aktuellen Geräte wie Smartphones oder Tablets nicht die entsprechende Sensorik für ein präzises Tracking der sechs Freiheitsgrade. \citet{van2010survey} weisen zudem darauf hin, dass die Registrierung der Tiefe für die Anwendung von Überdeckungen oder korrekter Positionierung bei einer Interaktion ein komplexes Problem sei. Wie in Kapitel \ref{sec:theory_project_tango} zu finden, geht Project Tango dieses Problem der Sensorik entsprechend an und versucht Schnittstellen zu bieten, um sowohl das Tracking zu ermöglichen als auch Tiefeninformationen über die aktuelle Szene zu liefern. 

Eine weitere erwähnenswerte Problematik ist neben den sensorischen Themen von Augmented Reality die Herangehensweise zur Gestaltung der Nutzeroberflächen für AR. Denn die UI Konzeption gestaltet sich, laut \citet{azuma2001recent}, schwierig. Die Anreicherungen durch Augmented Reality führt schnell dazu, dass das Sichtfeld überladen wirkt. Jedoch sollen dem Nutzer immer die Informationen zur Verfügung gestellt werden, die gegebenenfalls kontextsensitiv und relevant sind. Diese angesprochenen Faktoren führen aktuell bei Augmented Reality Anwendungsgebieten noch zu einer geringen Akzeptanz der Endverbraucher.

\subsection{Realisierung von Augmented Reality Über\-deckungen} \label{sec:ar-occlusion}

Es gibt einige Verfahren und Ansätze, um eine Überlagerung in Augmented Reality Szenen zu realisieren. \citet{wloka1995resolving} bilden hierfür den Grundstein für die verschiedenen existierenden Methoden. Sie stellen in Ihrer Arbeit ein Verfahren vor, welches mit Bildern aus einer Stereokamera ein Stereomatching durchführt und dadurch ein Tiefenbild generiert. Dieses Tiefenbild führt in dem Renderingprozess mit Hilfe des Z-Buffer Algorithmus (beschrieben in Kapitel \ref{sec:z-buffer}) zum Ausschluss von Teilen der virtuellen Objekte, die von realen Objekten überlagert werden. Das Ergebnis des Stereomatchings ist in ihrer Arbeit mit gewissen Ungenauigkeiten behaftet und generiert unerwünschte Lücken in der Projektion des virtuellen Objekts. Arbeiten wie von \citet{seo2013direct} mit neuen Tiefensensoren, wie der Microsoft Kinect\footnote{Microsoft Kinect - \url{https://dev.windows.com/en-us/kinect} (04.03.16)}, erhalten durch den selben Mechanismus deutlich bessere Ergebnisse. 

Die Arbeit von \citet{breen1996interactive} nahmen diesen Ansatz von \citet{wloka1995resolving} auf und stellten die Idee vor, neben einer deutlich genaueren Überdeckung, auch eine Interaktion mit realen Objekten zu realisieren. Hierfür werden virtuelle Modelle der realen Objekte in der Szene passend an der echten Umgebung und der Position der realen Objekte ausgerichtet. Dieses Vorgehen setzt jedoch voraus, dass die entsprechenden virtuellen Modelle für die realen Objekte bereits vorliegen. Nach dieser Ausrichtung wird die Tiefe der virtuellen Objekte bestimmt, um daraus, mit dem Verfahren von \citet{wloka1995resolving}, eine Überlagerung durch den Ausschluss der weiteren virtuellen Objekte der Szene zu bewirken. 

Neben den modellbasierten Verfahren existieren auch kantenbasierte Verfahren, wie das von \citet{berger1997resolving}, in dem Objektkanten auf optischer Basis mit Filtern ermittelt werden. Diese Kanten werden über mehrere Bilder verfolgt, um die Tiefeninformation der Kanten durch Epipolargeometrie und Heuristiken zu bestimmen. \citet{berger1997resolving} gewinnt darauf folgend  eine Tiefenmaske, indem er annimmt, dass Konturen, die unter einer gewissen Distanz von einander entfernt sind, zu einem Objekt gehören. \citet{klein2004sensor} erreichen mit Bergers Verfahren und einer auf mobiler Hardware umgesetzten Umgebung sehr überzeugende Ergebnisse in einer vordefinierten Umgebung. Zwar verspricht dieses Vorgehen eine kantengenaue Überdeckung virtueller Objekte, führt aber bei komplexeren Szenen, in denen die Kanten nicht mehr erfolgreich verfolgt werden oder nicht zu einem Objekt zugeordnet werden können, zu Fehldarstellungen. Außerdem werden Ausbreitungen innerhalb des Objekts, welche nicht als Kante erkannt werden können, nicht berücksichtigt.

Die letzte Variante wurde von \citet{breen1996interactive} bereits erwähnt und ist die Ermittlung der Überdeckung durch eine Rekonstruktion der Szene. Dieser Ansatz verschiebt durch das Verfahren von \citet{wloka1995resolving} die Problemstellung der AR Überdeckung in den Bereich der Echtzeit Rekonstruktionsproblematik. Bekannte Verfahren hierfür sind zum Beispiel KinectFusion von \citet{newcombe2011kinectfusion} oder die Echtzeitrekonstruktion von \citet{niessner2013real}. Diese sehr komplexen Verfahren sind meist auf der Grafikhardware von Desktopsystemen umgesetzt und generieren eine detaillierte Rekonstruktion aus den Tiefeninformationen in Echtzeit. Diese Rekonstruktionen können laut \citet{newcombe2011kinectfusion} für eine Echtzeitüberdeckung in Augmented Reality Systemen dienen. Vorteilhaft bei rekonstuktionsbasierter Überdeckung ist zudem, dass auch Interaktionen mit echten Objekten, wie bei \citet{breen1996interactive}, erfolgreich umgesetzt werden können.





\section{Project Tango} \label{sec:theory_project_tango}

Project Tango ist eine Technologie Plattform für Android Tablets und Smartphones von Google’s Advanced Technology and Projects Group (ATAP). Das Ziel dieser Plattform ist es Motion Tracking (Positionierung), Depth Perception (Tiefeninformation/Pointcloud) und Area Learning (Lokalisierung) auf mobile Endgeräte zu bringen, um verschiedenste Anwendungs-Szenarien abzudecken. Typische Szenarien sind Indoor Navigation, Virtual Reality Anwendungen, Vermessungs- und Rekonstruktions Software und Augmented Reality Anwendungen.

Das System ermöglicht in erster Linie ein Tracking von Positionsänderungen des Geräts im Raum und bietet somit eine genaue relative Lokalisierung. Mit Hilfe dieser Lokalisierung und der Hinzunahme von visuellen Merkmalen im Raum, ist das Gerät in der Lage, seine Umgebung kennenzulernen und gegebenenfalls die Lokalisierung zu korrigieren oder aber Diese in einer bereits erlernten Umgebung zu bestimmen. Zusätzlich bietet Project Tango die Möglichkeit, mit Hilfe eines Tiefensensors, eine Pointcloud der Tiefeninformation pro Bildausschnitt zu ermitteln, um Anwendungen auch räumliche Informationen bereitzustellen.  \citep{Proje19:online} 

\subsection{Geräte und Hardware}

Da das Project Tango zum Zeitpunkt der Verfassung dieser Thesis noch unter Entwicklung steht, gibt es von Google erste Entwickler Prototypen. Das Erste Gerät \enquote{Peanut Phone} im Smartphone Format, welches in Abbildung \ref{fig:tango-device} rechts unten zu erkennen ist, wurde Anfang 2014 veröffentlicht und ein halbes Jahr später bereits durch eine neue Generation, dem \enquote{Yellowstone Tablet} ersetzt. Dieses 7\dq Tablet, zu sehen rechts oben im Bild \ref{fig:tango-device}, verfügt, wie in der Abbildung links zu erkennen, über einen Infrarot Laser Projektor, eine Fisheye Kamera und eine normale 4 Megapixel Kamera auf der Rückseite. Zudem sind, wie in aktuellen Smartphones und Tables üblich, ein Beschleunigungssensor, Umgebungslichtsensor, Barometer, Kompass, GPS und ein Gyroskop verbaut. Das Gerät wird von einem NVIDIA Tegra K1 Prozessor betrieben und verfügt über 4GB Arbeitsspeicher. \citep{Proje19:online} Mit diesem Gerät wurden die später beschriebenen Techniken umgesetzt und evaluiert. 

\begin{figure}[h]
  \centering
	\includegraphics[width=1.0\textwidth]{content/images/theory/tango-device.png} 
  \caption{Links: schematischer Aufbau der Google Project Tango Hardware. Rechts: Das aktuelle Entwickler Gerät im Tablet Format (oben) und das alte Entwickler Gerät im Smartphone Format (unten). Übernommen von \citet{GoogleDevelopers:online}}
  \label{fig:tango-device}
\end{figure}

\subsection{Konzepte und Schnittstellen}

Generell betrachtet ist das Project Tango eine Plattform, die Computer Vision nutzt, um dem Gerät die Möglichkeit bietet seine relative Positionierung in der umgebenen Szene Echtzeit zu bestimmen. Auf den Geräten kommt Googles Android Betriebsystem zum Einsatz, weshalb zu beachten ist, dass es sich bei der Platform nur bedingt um eine Echtzeitumgebung handelt. Das liegt daran, dass der Linux Kernel keine Garantien für die zeitlich präzise Ausführung von Instruktionen auf Grund von Scheduling geben kann. Google weist daher darauf hin, dass das System als \enquote{soft-realtime} betrachtet werden sollte. Daher sollten Messergebnisse verschiedener Sensoren unter Berücksichtigung ihrer Aufnahmezeitpunkte verwendet werden. \citep{GoogleDevelopersConcepts:online} 

\subsubsection{Motion Tracking}

Um die relative Bewegung vom Start des Project Tango Systems bestimmen zu können, nutzt es \enquote{visual-inertial odometry}. \citep{GoogleDevelopersConcepts:online}
Dabei handelt es sich um eine erweiterte Variante von Visual Odometry. 
Das von \citet{nister2004visual} veröffentlichte Verfahren Visual Odometry ist in der Lage aus einfachen Videoinhalten in Echtzeit die Bewegung der Kamera zu bestimmen. 
Hierzu werden zunächst übergreifende Features, zum Beispiel Punkte aus der \citet{harris1988combined} Kantenerkennung, aus mehreren Bildern bestimmt. Um daraus eine Transformation zwischen den Bildern ermitteln zu können, wird der 5-point Algorithmus von \citet{nister2004efficient} angewendet. Dieser Algorithmus ist in der Lage das Problem zu lösen, eine relative Transformation zwischen zwei Bildern mit gegebenen 5 Punktübereinstimmungen zu ermitteln. Außerdem wird erwähnt, dass mit Hilfe des Schätzverfahrens RANSAC (beschrieben in Absatz \ref{sec:ransac}) bei einer Überbestimmung des Modells, ein potentieller Fehler deutlich minimiert werden kann. 

Project Tango lässt an dieser Stelle die internen Sensoren zur Rotation, Orientierung und Bewegung mit in die Bestimmung der Kameratransformation einfließen, um so ein präziseres Ergebnis erzielen zu können. Außerdem wird versucht mit Hilfe des Kalman Filters, nach dem gleichnamigen Autor \citet{kalman1960new}, die Fehler der Sensoren bei dieser Echtzeitmessung zu reduzieren. Über eine längere Messzeit oder eine größere Entfernung vom Ursprung kann es jedoch zu kleinen Abweichungen kommen. Außerdem existiert zum aktullen Zeitpunkt noch ein \enquote{Drift} Problem, was zu großen Messfehlern führen kann. Es wird jedoch versucht diese Probleme mit dem Konzept \enquote{Area Learning}, beschrieben in Kapitel \ref{subsec:area-learning}, zu lösen. \citep{GoogleDevelopersConcepts:online}

Wie genau das Verfahren aussieht, welche Techniken zur Feature Detection oder Feature Matching genutzt werden und welche Features hierfür erkannt werden, ist nicht bekannt. \citet{Klingensmith_2015_7924}, als Mitglieder Googles Advanced Technologies and Projects Abteilung ATAP, erwähnen jedoch, dass nähere Informationen über das Verfahren von \citet{kottas2013consistency} und \citet{mourikis2007multi} beschrieben werden. Sie erläutern in Ihren Arbeiten, welche Mechanismen eingesetzt werden können, um eine Migration aller Sensorinformationen für ein zuverlässiges hybrides optisches Tracking zu realisieren.

\subsubsection{Deph Perception}

Zur Tiefenmessung ist die Project Tango Hardware mit einem kalibrierten Infrarot Laserprojektor ausgestattet. Dieser streut Infrarot Punkte mit einer Auflösung von 320 x 180 Punkten in den Raum, um dann mithilfe von Aufnahmen der RGB Kamera, eine Punktewolke der Tiefeninformation zu bestimmen. Aufgrund einer ausgewogenen Konfiguration zwischen Messbereich, Messfehlern und dem Energieverbrauch, liegt der Messbereich der Sensorkombination, laut \citet{GoogleDevelopersConcepts:online}, zwischen einem halben und vier Metern. 

Dadurch, dass diese Technologie auf der Aufnahme von projiziertem Infrarot Licht basiert, ist ein Einsatz der Tiefenmessung außerhalb geschlossener Räume nicht möglich. \citep{GoogleDevelopersConcepts:online} Außerdem entstehen Messfehler durch reflektierende,  lichtabsorbierende oder zu komplex strukturierte Oberflächen, wie zum Beispiel Metalle, LCD Monitore oder Hochflor Teppiche. 

Die zuvor erwähnten Punktewolken werden in dem eigens definierten XYZij Format von der Entwicklungsschnittstelle zurück gegeben. Dabei wird jeder Punkt mit den \(X\),\(Y\) und \(Z\) Koordinaten im Weltkoordinatensystem und den beiden Indizes \(i \) und \(j \) für die Spalte und Zeile der projizierten Punkte auf der Bildebene angegeben \citep{GoogleDevelopersConcepts:online}. Man spricht dabei von einer organisierten Punktewolke, da durch die \(i\) und \(j\) Koordinaten die direkten Nachbarn, ausgehend von dem Aufnahmeblickwinkel, eines Punktes bestimmt werden können. Hieraus ist es möglich Tiefenbilder, die sogenannten \enquote{Depth Maps}, zu bestimmen, für die es viele verschiedene Computer Vision Verfahren zur Bestimmung von Objekten, Strukturen und Fluchtpunkten gibt. Die Schnittstellen liefern jedoch zum aktuellen Entwicklungsstand die beschriebenen Informationen über die Spalten \(i\) und Zeilen \(j\), laut \citet{GoogleDevelopersKnownIssues:online}, noch nicht.

\citet{GoogleDevelopersConcepts:online} weist darauf hin, dass das Generieren von Polygon basierten Rekonstruktionen noch nicht in den Schnittstellen enthalten ist. Es gibt jedoch freie Dritt-Bibliotheken und -Systeme, wie das Robot Operating System \footnote{\url{http://ros.org/} (23.02.2016)} oder die Point Cloud Library\footnote{\url{http://pointclouds.org/} (23.02.2016)}, die für eine weitere Verarbeitung genutzt werden können.

\subsubsection{Area Learning} \label{subsec:area-learning}

Area Learning bezeichnet den Prozess in dem Project Tango Geräte in der Lage sind durch visuelle Hinweise die umgebenen Welt kennenzulernen und auf die Position des Gerätes zu schließen. 
Es ermöglicht somit eine Unterstützung für Motion Tracking und löst das Problem das Gerät in einer bereits bekannten Umgebung zu lokalisieren, wie in Abbildung \ref{fig:area-learning} links zu erkennen.
Project Tango bietet außerdem die Möglichkeit diese visuellen Hinweise und Ihre Position im Raum in sogenannten \enquote{Area Description Files} zu speichern und wiederzuverwenden. \citep{GoogleDevelopersConcepts:online}

\begin{figure}[h]
  \centering
	\includegraphics[width=1.0\textwidth]{content/images/theory/tango-area-learning.png} 
  \caption{Links: Lokalisierungsprozess durch Area Learning. Rechts: Korrigierung von Motion Tracking anhand gelernter Merkmale . Übernommen von \citet{GoogleDevelopers:online}}
  \label{fig:area-learning}
\end{figure}

Wie bereits erwähnt entstehen bei Motion Tracking über eine längere Strecke Messfehler. 
Während diese Strecke mit einem Project Tango Gerät abgelaufen wird, ermittelt es fortlaufend die Position und den Pfad, den der Nutzer im Raum gegangen ist. 
Erkennt es während der Strecke visuelle Merkmale aus Area Learning, wird der Pfad anhand der Positionen der Merkmale entsprechend angepasst. 
Project Tango unterscheidet hier zwischen zwei Manipulationen, \enquote{loop closures}, zur Zusammenführung des Pfads wenn ein Kreis gelaufen wurde, und \enquote{drift corrections}, um den erwähnten Drift-Effekt bei zu wenigen optischen Features im visual-inertial odometry. 
Die drift correction ist in Abbildung \ref{fig:area-learning} rechts zu erkennen. \citep{GoogleDevelopersConcepts:online} 

Auch bei diesem Prozess werden die genauen Details nicht näher erläutert und es ist nicht bekannt wie die Area Desciptions definiert sind oder was sie enthalten. \citep{GoogleDevelopersConcepts:online} weist jedoch darauf hin, dass auch wenn die Area Desciptions Files keine direkten Bilder enthalten, es möglich sei, Rückschlüsse auf die gelernte Umgebung ziehen zu können. 

\subsection{Einordnung zu Augmented Reality} \label{sec:classification_project_tango}

Da sowohl die Grundlagen aus dem Bereich Augmented Reality und die technische Basis von Project Tango bekannt ist, kann die Project Tango Hardware bezüglich Augmented Reality näher eingeordnet werden. Bei der Hardware handelt es sich um ein hand-held Gerät, welches mit einer video see-through Display Technologie Augmented Reality Anwendungen ermöglicht. Hierfür kann wahlweise die normale RGB Kamera oder die Graustufen Fish-Eye Kamera verwendet werden. Denn für beide Kameras können die intrinsischen Kameraparameter ausgelesen werden, wodurch die Eigenschaften der realen Kamera durch die virtuelle Kamera übernommen werden können. Das führt zu einer parallax freien Überblendung und zu einer guten Tiefenwahrnehmung. 

Als Tracking Technologie wird hier eine hybride optische Variante angewendet. Dabei wird die Visual Odometry mit der Fish-Eye Kamera durch interne Sensoren für die Rotation, Orientierung und Bewegung kombiniert. Außerdem kann das Verfahren gegebenenfalls durch Googles Area Learning Mechanismen angereichert werden, um Messfehlern entgegenzuwirken. Die Eingabe erfolgt durch den Touchscreen des Tablets. Eine Interaktion mithilfe von optischer Gestenerkennung oder anhand der Tiefeninformationen ist zudem auch denkbar. 







% Theoretische Grundlagen
\chapter{Theoretische Grundlagen} \label{sec:algorithms}

In diesem Kapitel werden die Theoretischen Grundlagen und Algorithmen näher erläutert, die zur Realisierung der Optimierungsverfahren im nachfolgenden Kapitel \ref{sec:optimization} angewendet werden. 



\section{Z-Buffer Algorithmus} \label{sec:z-buffer}

Der Z-Buffer Algorithmus, welcher fast zur selben Zeit von \citet{straber1974schnelle} und \citet{catmull1974subdivision} vorgestellt wurde, ermöglicht in der Computergrafik eine einfache Berechnung der Überdeckungen von gerenderten Objekten auf der Bildebene. Hierzu wird neben der Bildebene noch ein sogenannter \enquote{Z-Buffer} eingeführt, der für jeden gerenderten Pixel auf der Bildebene eine Tiefeninformation festhält. Initial enthält der gesamte Z-Buffer die Tiefeninformation die der Backclipping-Ebene entspricht. Somit lässt sich für jeden Pixel einer bestimmten Position bestimmen, ob dieser einen kleineren Tiefenwert besitzt und somit auf die Bildebene gerendert wird. 

Heutzutage ist dieser leicht zu implementierende Mechanismus aus Performancegründen in nahezu jeder Grafikkarte in Hardware implementiert. Das Problem bei dieser einfachen Umsetzung ist jedoch, dass jedes Objekt der Szene den gesamten Renderingprozess durchlaufen muss, auch wenn es später auf der Bildebene nicht erscheint. Abhilfe für dieses Problem bietet der Hierarchical Z-Buffer von \citet{greene1993hierarchical}, welcher nicht sichtbare Objekte zuvor aussortiert. Für die Anwendung für die Optimierungsverfahren in Kapitel \ref{sec:optimization} ist jedoch nur die Idee des einfachen Z-Buffers relevant.

\section{RANSAC} \label{sec:ransac-theory}

Der \enquote{RAndom SAmple Consensus} Algorithmus (RANSAC), vorgestellt von \citet{fischler1981random}, ist in der Lage, aus einer Menge von Daten mit vielen Ausreißern, die Parameter für ein passendes Modell zu schätzen. Anders als andere Schätzverfahren wie \enquote{Least-Median} oder \enquote{M-Schätzer}, welche aus der Statistik Literatur entnommen und entsprechend angepasst wurden, wurde RANSAC speziell für die Anwendung in der Computer Graphik entwickelt. Der Kern dieses Algorithmus ist das wiederholte Bestimmen eines Modells aus zufälligen und für das Modell ausreichenden Stichproben. Listing \ref{lst:ransac} zeigt den Verlauf des RANSAC Algorithmus. Die Anzahl der Iterationen \(N\) hängt dabei allein von dem Anteil der Ausreißer in den Messwerten ab. Daher sollte sie entsprechend gewählt werden, um die Wahrscheinlichkeit zu verringern, dass Ausreißer in den Stichproben enthalten sind. \citep{derpanis2010overview} \\

\begin{lstlisting}[mathescape,caption=Der RANSAC Algorithmus, label=lst:ransac, float=htbp]
Eingabe: Messwerte $P$, Modelltoleranz $e$, maximale Iterationen $N$
Ausgabe: Modell $m$, Unterstützende Messwerte $P_m$

1. Wähle zufällig so viele Stichproben aus den Messwerten $P$,
   wie nötig sind, um das Modell zu bestimmen
2. Bestimme aus den gewählten Stichproben das Modell $m$
3. Ermittle die Anzahl der Messwerte $P$, die mit einer 
   entsprechenden Toleranz $e$ das ermittelte Modell $m$ 
   unterstützen
4. Wenn prozentual genügend Messwerte aus $P$ das Modell $m$ 
   unterstützen dann terminiere und gehe zu 7
5. Wenn $|P_m| < |P|$ dann $|P_m| = |P|$
6. Wiederhole die Schritte 1-4 $N$ mal
7. Ermittle aus den unterstützenden Messwerten $P_m$ erneut 
   das finale Modell $m$
   
\end{lstlisting} 

Die in Schritt sieben beschriebene erneute Modellermittlung kann laut \citet{fischler1981random} durch ein ein Regressionsverfahren wie der Methode der kleinsten Quadrate bestimmt werden. Anwendung findet dieser Algorithmus in der Ebenendetektion aus Kapitel \ref{sec:ransac}.

\section{Ermittlung der konvexen Hülle}

Als eine konvexe Hülle wird eine Teilmenge einer euklidischen Gesamtmenge an Punkten bezeichnet, für die folgende Bedingung gilt: Wählt man aus der Gesamtmenge zwei beliebige Punkte, so ist die Strecke zwischen ihnen immer innerhalb der konvexen Hülle. Für die Bestimmung der konvexen Hülle von Punkten \(P\) in \( \mathbb{R}^2\) existieren verschiedenste Algorithmen, wie zum Beispiel der Monotone Chain von \citet{andrew1979another}, QuickHull nach \citet{eddy1977new} oder der Divide-and-Conquer Algorithmus von \citet{preparata1985convex}. Da alle diese Algorithmen ein ähnliches Laufzeitverhalten aufweisen, wird für die spätere Anwendung hier exemplarisch der Graham Scan nach \citet{graham1972efficient} näher erläutert. Dies ist einer der populärsten Algorithmen für die Berechnung der konvexen Hülle und besitzt eine Komplexität von \(O(n \log n)\).

Gestartet wird der Algorithmus mit der Menge aller Punkte \(P\) und mit einem Startpunkt \(P_0\), welcher Bestandteil der konvexen Hülle ist. Hierzu wird meist der Punkt mit dem niedrigsten \(y\) Faktor gewählt (\(P_0=P_{min(y)}\)). Listing \ref{lst:graham-scan} zeigt den Verlauf des Algorithmus des Graham Scans. Außerdem wird in Abbildung \ref{fig:convexhull} zusätzlich die erste Sortierung und das Unterscheidungskriterium für die Sortierung und Aussortierung der Punkte verdeutlicht. \citep{convexHull} 


\begin{lstlisting}[mathescape,caption=Graham Scan Algorithmus, label=lst:graham-scan, float=htbp]
Eingabe: Menge der Punkte $P$, außen liegender Punkt $P_0$
Ausgabe: Punkte der konvexen Hülle in $P$

$i$ = 0
sortiere $P$ nach dem Winkel zu $P_0$
solange $i$ <= $|P|$
    wenn $\measuredangle P_{i-1} P_{i}$ > $\measuredangle P_{i-1} P_{i+1}$, also $P_i$ rechts von  $\vec{P_{i-1} P_{i+1}}$ liegt
        inkrementiere $i$
    ansonsten
        entferne $P_i$ aus $P$
        dekrementiere $i$
    
\end{lstlisting} 

\begin{figure}[h]
  \centering
	\includegraphics[width=0.9\textwidth]{content/images/methods/convexhull.png} 
  \caption{Sortierung der Punkte nach Winkel zum Startpunkt (links). Das Unterscheidungskriterium für die Sortierung (rechts). Übernommen von \citet{convexHull}}
  \label{fig:convexhull}
\end{figure}


\section{Octree}

Ein Octree ist zunächst eine Datenstruktur, die wie ein Baum mit beliebiger Tiefe aufgebaut ist und pro Knoten Acht Kinder besitzt. Die Funktion ist dabei in \(\mathbb{R}^3\) gleich zu einem Binärbaum in \(\mathbb{R}\) oder einem Quadtree in \(\mathbb{R}^2\). Ein Knoten repräsentiert im Octree einen Würfel, der durch seine Kinder in Acht Kind-Würfel aufgeteilt wird. Durch diese räumliche Aufteilung ergeben sich verschiedene Vorteile gegenüber linearen Datenstrukturen. So müssen Bereiche zum festhalten räumlicher Informationen im Octree nur dann allokiert werden, wenn diese Bereiche auch verwendet werden. Speichert man Punkte in den untersten Knoten eines Octrees kann man durch eine Tiefenbegrenzung beim Zugriff auf den Baum ein sehr effektives Downsampling der Punkte vornehmen. Zuletzt entstehen durch die Knoten einer bestimmten Tiefe ein Cluster, zu denen in diesem Fall Punkte bei einer Aufnahme hinzugefügt werden und für eine weitere Verarbeitung extrahiert werden können.\\

\section{Interaktion durch Raypicking} \label{sec:ar-depth-interaction}

Wie in Kapitel \ref{sec:ar-interaction} beschrieben, bedarf es bei der Umsetzung von Augmented Reality Systemen ein anderes Interaktionsparadigma. Auch wenn die Entwicklung der neuen Tablet und Smartphone Geräte durch Touchscreens eine neue Interaktionsform eingeführt haben, ist sie in den meisten Fällen auf einer zweidimensionalen Ebene beschränkt. In der Entwicklung von Virtual Reality oder voll virtuellen Anwendungen und Spielen wird oft für die Auswahlgeste der Raypicking Mechanismus verwendet, um eine zweidimensionale Interaktion im dreidimensionalen Raum zu ermöglichen. Darüber hinaus gibt es verbesserte semantische Interaktionsformen basierend auf einer zweidimensionalen Toucheingabe, wie von \citet{elmqvist2008semantic} beschrieben.\\

Hier soll aber zunächst eine Raycasting Variante für Augmented Reality Anwendungen umgesetzt werden, die nicht von einem kompletten Modell in Form von Polygonen oder anderen Primitiven der realen Umgebung ausgeht. Diese AR Interaktion ermöglicht, anhand der Tiefeninformationen, das passende Positionieren von virtuellen Objekten im realen Raum und lässt sich auf weitere Interaktionen erweitern. Voraussetzung für die folgende Umsetzung, ist die entsprechende Kalibrierung und Gleichstellung der intrinsischen Kameraparametern und der Verfügbarkeit der extrinsischen Parameter der realen Kamera. \\

\begin{figure}[h]
  \centering
	\includegraphics[width=1.0\textwidth]{content/images/methods/interaction.jpg} 
  \caption{Raypicking Visualisierung. Übernommen von \citet{gluUn11:online}}
  \label{fig:interaction}
\end{figure}

Als Erstes wird ein Strahl erzeugt, der durch die Position der virtuellen Kamera und durch den jeweils ausgewählten Punkt auf der Viewingplane läuft. Den Ursprung der virtuellen Kamera bestimmt dabei Google Tangos \enquote{Motion Tracking}. Der gewählte beziehungsweise berührte Punkt auf dem Touchscreen wird dabei zunächst von Pixeln in das Verhältnis \(\left[-1,1\right]\) des Punktes umgerechnet. Hiernach wird die Projektion auf die Viewingplane durch eine Multiplikation mit \(T\) aus Gleichung \ref{eq:unprojection} rückgängig gemacht. Die Gerade aus den beiden Punkten kann danach genutzt werden, um den Schnitt von Objekten vor der Kamera zu ermitteln. \citep{OpenG86:online} 

\begin{equation} \label{eq:unprojection}
T  = MV_{ModelView}^{-1} * P_{Projection}^{-1}
\end{equation}

Angewendet auf die Tiefeninformation aus Tangos \enquote{Depth Perception} wird die Punktewolke, wie in Abbildung \ref{fig:interaction} anstelle der Objekte, vor die Kamera projiziert. In den projizierten Punkten wird danach der entsprechende Punkt gesucht, welcher sich am nächsten am zuvor bestimmten Strahl befindet. Durch diese beschriebenen Schritte kann der Nutzer mit einer zweidimensionalen Geste einen Punkt in der Tiefe bestimmen. Diese Methode kann zudem um die Ermittlung einer Ebenennormalen erweitert werden. Hierzu werden um den selektierten Punkte Nachbarn gefunden, mit denen durch RANSAC eine Ebene ermittelt wird. Durch die ermittelte Ebenennormale können virtuelle Objekte dann nicht nur an die ausgewählte Stelle positioniert werden, sondern auch an der realen Oberflächenausrichtung ausgerichtet werden.



% Optimierung von AR
\chapter{Verfahren zur Realisierung von Überdeckungen in Augmented Reality durch Tiefen\-informationen} \label{sec:optimization}


Die folgenden Abschnitte widmen sich den Verfahren zur möglichen Realisierung von Überlagerungen durch Tiefen- und Bildinformationen. Nach der Recherche zu möglichen Verfahren soll erst einmal der grundlegende Ansatz von \citet{wloka1995resolving} zur einfachen Überlagerung durch Tiefeninformationen mithilfe der Projektion der von Project Tango gelieferten Pointcloud realisiert werden, da dieses Ausschlussverfahren das grundlegende Vorgehensmodell zur Überlagerung darstellt. 

Die Kanten- und Modellbasierten Verfahren zur AR Überdeckung aus Kapitel \ref{sec:ar-occlusion} werden hier nicht weiter berücksichtigt, da sie offensichtliche Nachteile gegenüber anderen Ansätzen bergen. So muss beim Modell basierten Verfahren bereits ein Modell der echten Umgebung existieren und das Kanten basierte Verfahren schränkt den Einsatz auf eine weniger komplexe Szene ein. Außerdem sind beide Verfahrensarten so konzipiert, dass sie keine direkten Tiefeninformationen benötigen, die aber von Project Tango generiert werden können und hier auch ausgehend von der anfangs beschriebenen Zielsetzung genutzt werden. Aus diesem Grund widmen sich die darauf folgend beschriebenen Umsetzung den rekonstuktionsbasierten Verfahren.

Während dieser Arbeit wurde zunächst ein eigenes Echtzeit Rekonstuktionsverfahren basierend auf einer Ebenenerkennung zusammengestellt und entwickelt, welches auch auf der mobilen Project Tango Hardware realisierbar ist. Daraufhin wurde nach weiteren Recherchen ein neues Rekonstruktionsverfahren gefunden, welches im Zusammenhang mit Project Tango veröffentlicht, und somit für den Einsatz auf mobiler Hardware konzipiert wurde. Auch dieser Ansatz wird hier näher beschrieben, um ihn später zu implementieren und zu testen. Zuletzt soll näher auf die Möglichkeit eingegangen werden, die resultierenden Tiefeninformationen aus der Pointcloud oder aus dem Rendering der Rekonstruktion mit Hilfe der Bildaufnahmen der Farbkamera, zu verbessern. 


\section{Verdeckung durch Pointcloud Projektion} \label{sec:pc-projection}

Die erste und weniger aufwändige Idee eine Überlagerung in Augmented Reality zu realisieren, ist die Überführung der Pointcloud in eine Depthmap, die wiederum in den Renderingprozess mit eingebracht wird. Das Verfahren von \citet{kanbara2000stereoscopic} verfolgt einen ähnlichen Weg mit einer Stereokamera und einer video see-through Displaytechnologie in Form eines Head-Mounted Display. Wie in Abbildung \ref{fig:stereo-depth-map} zu erkennen, bestimmen sie mit Hilfe der Stereokamera Tiefeninformationen, die das gerenderte virtuelle Objekt an den Positionen ausspart, an denen Tiefeninformationen im Vordergrund vorliegen. Aufgrund von weiteren Forschungsergebnissen aus dem Bereich des Stereomatchings, erreichen sie bessere Ergebnisse als \citet{wloka1995resolving}, die das Vorgehen als Erste vorgestellt haben.

\begin{figure}[h]
  \centering
	\includegraphics[width=1.0\textwidth]{content/images/methods/stereo-depth-map.png} 
  \caption{Visualisierung der Methode zur Vedeckung durch Depth Maps. Übernommen von \citet{kanbara2000stereoscopic}}
  \label{fig:stereo-depth-map}
\end{figure}

Anders als im Verfahren von \citet{kanbara2000stereoscopic} wird hier eine Depthmap anhand der vorhandenen Pointcloud aus dem Infrarotsensor von Project Tango gewonnen. Dadurch, dass die intrinsischen Kameraeigenschaften der Farbkamera zur Verfügung stehen, welche auch die Aufnahmequelle der Infrarotpunkte ist, kann ein Punkt \(P = [X, Y, Z]\) der Pointcloud mit der Gleichung \ref{eq:projection} auf die Bildebene überführt werden. Hier stehen die Variablen \(f_{x/y}\) für die Brennweite und \(c_{x/y}\) für den Bildmittelpunkt auf der Bildebene. \citep{Tango90:online}

\begin{equation}\label{eq:projection}
x = \frac{X* f_x * \frac{r_d}{r_u}}{Z}  + c_x
\qquad
y = \frac{Y* f_y * \frac{r_d}{r_u}}{Z}  + c_y
\end{equation}

Da die Linsen einer Kamera nie perfekte Brechungseigenschaften besitzen, muss an dieser Stelle auch die Verzerrung der Linse mit berücksichtigt werden. In den Gleichungen aus \ref{eq:projection} ist diese Verzerrung in den Parametern \(\frac{r_d}{r_u}\) enthalten. Zur Kalibrierung der Linse wird bei Project Tango für die normale Farbkamera und somit auch für die Aufnahme der Infrarotpunkte die Technik von \citet{tsai1987versatile} verwendet  \citep{Tango90:online}. Das hier beschriebene Verzerrungsmodell basiert dabei auf den drei Parametern \(k_{1} \downarrow  k_{3}\), welche die radiale Verzerrung ausgehend vom Linsenmittelpunkt modellieren kann. Hierzu wird für jeden Punkt, wie in Gleichung \ref{eq:distortion} gezeigt, die radiale Distanz \(r_u\) zum Linsenmittelpunkt und die durch die Linseneigenschaft verzerrte radiale Distanz \(r_d\) ermittelt. Dieses Verhältnis streckt oder staucht die Position radial auf der Bildebene.


\begin{equation} \label{eq:distortion}
r_u = \sqrt{\frac{X^2 + Y^2}{ Z^2}} 
\qquad
r_d = r_u + k_1 * r_u^3 + k_2 * r_u^5 + k_3 * r_u^7
\end{equation}

An dieser berechneten Position auf der Bildebene bzw. Depthmap wird nun ein Punkt mit einem Graustufenwert, entsprechend der Entfernung \(|\overrightarrow{PO_{cam}}|\) vom Punkt \(P\) zum Kameraursprung \(O_{cam}\) gezeichnet. Der Farbwert richtet sich dabei nach der Konvention des Renderingframeworks und den Informationen über die vordere und hintere Clippingebene.

Die Auflösung des Tiefensensors der Project Tango Hardware ist mit \(320 \times 180\) Pixeln gegenüber der Auflösung der Farbkamera mit \(1280 \times 720\) vier mal kleiner. Zusätzlich ist die Dichte der Pointcloud zum eigentlichen Sensor geringer als ihre eigene Auflösung. So würde man bei einer Auflösung von \(320 \times 180 = 57600\) Tiefenpunkte bei idealen Verhältnissen erwarten. Project Tango liefert jedoch unter guten Bedingungen durchschnittlich \(17000\) Tiefenpunkte. Aufgrund der kleineren Auflösung und der geringeren Informationsdichte werden die gezeichneten Punkte auf der Depthmap hier mit einem Radius von 4 Bildpunkten gezeichnet. 

Nachdem die Depthmap generiert wurde, kann zum Ausschluss der Pixel der virtuellen Objekte, welches sich hinter einem realen Objekt befinden, der Z-Buffer Algorithmus aus Kapitel \ref{sec:z-buffer} wie von \citet{wloka1995resolving} beschrieben, angewendet werden. Hierzu wird vor dem virtuellen Rendering der Z-Buffer mit den generierten Informationen aus der Depthmap gefüllt. Pixel der virtuellen Objekte werden somit nicht gerendert, wenn an dieser Position eine geringere Distanz zu realen Objekten in den Tiefeninformationen vorliegen. 




\section{Planare Rekonstruktion}

\subsection{Verfahren zur Ebenendetektion}

\subsubsection{RANSAC}

Der \enquote{RAndom SAmple Consensus} Algorithmus (RANSAC), vorgestellt von \citet{fischler1981random}, ist in der Lage, aus einer Menge von Daten mit vielen Ausreißern, die Parameter für ein passendes Modell zu schätzen. Anders als andere Schätzverfahren wie \enquote{Least-Median} oder \enquote{M-Schätzer}, welche aus der Statistik Literatur entnommen und entsprechend angepasst wurden, wurde RANSAC speziell für die Anwendung in der Computer Graphik entwickelt. Der Kern dieses Algorithmus ist das wiederholte Bestimmen eines Modells aus zufälligen und für das Modell ausreichenden Stichproben. Listing \ref{lst:ransac} zeigt den Verlauf des RANSAC Algorithmus. Die Anzahl der Iterationen \(N\) hängt dabei allein von dem Anteil der Ausreißer in den Messwerten ab. Daher sollte sie entsprechend gewählt werden, um die Wahrscheinlichkeit zu verringern, dass Ausreißer in den Stichproben enthalten sind. \citep{derpanis2010overview} \\

\begin{lstlisting}[caption=Der RANSAC Algorithmus, label=lst:ransac]
Eingabe: Messwerte P, Modelltoleranz e, maximale Iterationen N
Ausgabe: Modell m, Unterstützende Messwerte Pm

1. Wähle zufällig so viele Stichproben aus den Messwerten P,
   wie nötig sind, um das Modell zu bestimmen
2. Bestimme aus den gewählten Stichproben das Modell m
3. Ermittle die Anzahl der Messwerte P, die mit einer 
   entsprechenden Toleranz e das ermittelte Modell m unterstützen
4. Wenn prozentual genügend Messwerte aus P das Modell m unterstützen,
   ermittle aus den unterstützenden Messwerten Pm durch lineare 
   Regression erneut das finale Modell m und terminiere
5. Wiederhole die Schritte 1-4 N mal
\end{lstlisting} 

Um mit dem RANSAC Algorithmus Ebenen in einer Punktewolke bestimmen zu können, werden pro Iteration drei Stichproben \(A\), \(B\) und \(C\) gewählt. Das Ebenenmodell, hier in der Hesse Normalform mit dem Normalenvektor \(\vec{n}\) und dem Abstand zum Koordinatenursprung \(d\), lässt sich dabei durch die Gleichung \ref{eq:normalform} bestimmen.

\begin{equation}\label{eq:normalform}
\vec{n} =\left|\left| \vec{AB} \times \vec{AC}\right|\right|
\qquad
\vec{D} = \vec{A} \cdot \vec{n}
\qquad
d = \vec{D}\left[x\right] + \vec{D}\left[y\right] + \vec{D}\left[z\right]
\end{equation}


\subsubsection{3D Hugh Transformation}

\subsubsection{Agglomeratives Clustering}

\subsubsection{Region Growing}

\subsection{Bestimmung der Ebenenausbreitung}

\subsubsection{Convex Hull Algorithmus}

\subsubsection{Triangulation}

\subsection{Planare Rekonstruktion als Echtzeit Prozess}

\subsubsection{Clusteringverfahren}

\subsubsection{...}

\subsection{Kantenverbesserung durch Einbeziehung von Bildmaterial}

\subsubsection{...}

\section{Echtzeit Polygon Rekonstruktion} \label{sec:polygon_reconstruction}

Das zweite rekonstuktionsbasierte Überlagerungsverfahren bezieht sich in diesem Kapitel auf dem aktuellen Forschungsstand der Echtzeitrekonstruktion und soll in den folgenden Absätzen näher beschrieben werden. Die Echtzeit 3D Rekonstruktion ist bereits ein etabliertes Forschungsgebiet in der Computergrafik und gewinnt, aufgrund von kostengünstigen Consumer Tiefensensoren, wie die Microsoft Kinect\footnote{Microsoft Kinect - \url{https://dev.windows.com/en-us/kinect} (13.03.16)}, Xtion\footnote{Ausus Xtion - \url{https://www.asus.com/de/3D-Sensor/Xtion_PRO_LIVE/} (13.03.16)} oder Structure\footnote{Occipital Structure - \url{http://structure.io/} (13.03.16)} , zunehmend an Bedeutung. Dabei liegt der Fokus zunehmend auf der Echtzeitrekonstruktion, da diese Geräte in der Lage sind, Tiefeninformationen, zwar mit leichten Messfehlern, aber in Echtzeit, zu liefern. \citet{niessner2013real} erwähnen an dieser Stelle zudem den möglichen Einsatz für Augmented Reality:

\begin{quote}
\enquote{The ability to obtain reconstructions
in real-time opens up various interactive applications including:
augmented reality (AR) where real-world geometry can be fused
with 3D graphics and rendered live to the user; ...} \citep{niessner2013real}
\end{quote}

Die Herausforderung in der Echtzeitrekonstruktion liegt dabei in der möglichst performanten Fusion von mehreren überlagernden Tiefenbildern, welche aus verschiedenen Betrachtungswinkeln aufgenommen werden. Hieraus soll eine möglichst detaillierte Repräsentation der echten Umgebung generiert werden, welche sich im Idealfall stetig verbessert. Diese Problemstellung unterscheidet sich dabei von herkömmlichen Rekonstruktionsverfahren wie dem von \citet{hoppe1992surface} und der Poission Rekonstruktion von \citet{kazhdan2006poisson}. Aktuelle Verfahren nutzen verschiedenste optimierte Datenstrukturen, welche zudem durch den Einsatz von entsprechenden GPU Implementierungen beschleunigt und in Echtzeit angewendet werden können. Hier spielt die Gegenüberstellung von Detailgrad, der Skalierung und Geschwindigkeit stets eine große Rolle. \citep{niessner2013real} 

Bekannte Verfahren wie KinectFusion \citep{newcombe2011kinectfusion}, ein SLAM Verfahren von \citet{bylow2013real} oder DynamicFusion \citep{newcombe2015dynamicfusion} nutzen die \enquote{Truncated Signed Distance Function}, kurz TSDF, zur Speicherung und Migration der Oberflächeninformation mehrerer Depth Maps. Das Verfahren von \citet{niessner2013real} erweitert diesen Ansatz mit einem effizienten räumlichen Hashingverfahren, um die Zugriffszeiten und Speicherverbrauch zu minimieren. Darüber hinaus nimmt das Verfahren Chisel von \citep{Klingensmith_2015_7924} diese Vorzüge auf und kombiniert TSDF mit \enquote{visual-inertial odometry}, der Trackingtechnologie von Project Tango. In den folgenden Absätzen werden die Mechanismen hinter TSDF, dem räumlichen Hashing und den Vorzügen von Chisel näher erläutert. Außerdem soll auch noch auf das Rendering der TSDF Oberfläche durch Marching Cubes eingegangen werden, welches in Chisel verwendet wird. 


\subsection{Truncated Signed Distance Function}

Bei der von \citet{curless1996volumetric} vorgestellten räumlichen Repräsentation von Oberflächen, Truncated Signed Distance Function (TSDF), wird der Raum in Voxel einer gewünschten Auflösung unterteilt. Anders als bei Occupancy Maps, in denen die Voxel als sichtbar oder unsichtbar binär markiert werden, werden bei TSDF in den Voxeln die jeweiligen Entfernungen zur nächst gelegenen Oberfläche angegeben. Wichtig dabei ist das Vorzeichen, welches angibt, ob sich ein Voxel innerhalb oder außerhalb eines Objektes befindet. Abbildung \ref{fig:tsdf} zeigt unter a) die Ergebnisse mit Occupancy Maps und in b) die Voxel von TSDF. \citep{curless1996volumetric} 

Gefüllt wird die Repräsentation durch die Depth Maps und der entsprechenden Kameraposition, die im Fall von Project Tango durch Motion Tracking bereits gegeben ist. So wird für jede Tiefeninformation ein Strahl ausgehend von der Kameraposition generiert, der die durchgeschnittenen Voxel aktualisiert. Der Stahl ist dabei von der Länge begrenzt, um die zu aktualisierenden Voxel klein zu halten und zudem keine Oberflächen zu aktualisieren, die sich weiter hinter der gefundenen Oberfläche befindet. Dieses Vorgehen ist in Abbildung \ref{fig:tsdf} c) zu erkennen. \citep{Compu66:online} 

\begin{figure}
  \centering
	\includegraphics[width=1.0\textwidth]{content/images/methods/tsdf.png} 
  \caption{a) Beispielhafte Voxel Füllung von Occupancy Maps; b) Beispielhafte Voxel Füllung durch TSDF; c) Exemplarische 2D Darstellung der Oberfläche mit entsprechenden Strahlensatz für die TSDF. Übernommen von \citet{Compu66:online}}
  \label{fig:tsdf}
\end{figure}

Der Vorteil dieser Repräsentation liegt darin, dass die konkreten Oberflächeninformationen, anders als bei der Diskretisierung von Occupancy Maps, nicht verloren gehen. Das heißt, dass trotz einer gröberen Voxel Struktur stets der Nulldurchgang rekonstruiert werden kann. Neben der Entfernung zur nächsten Oberfläche wird zusätzlich noch ein Gewichtungswert in jedem Voxel gespeichert. Das ermöglicht es leichtes Rauschen durch eine einfache Mittelung mehrerer Messerergebnisse der Oberfläche zu unterdrücken. Außerdem kann somit eine stetige Optimierung der Oberfläche vorgenommen werden. \citep{Compu66:online}

\citet{hoppe1992surface} nutzen in Ihrem offline Rekonstruktionsverfahren auch die hier beschriebene TSDF. Jedoch bestimmen sie für jeden festgehaltenen Punkt der Pointcloud die umliegenden Nachbarn, um eine Tangentialebene zu ermitteln, von der aus die auf der Normalen liegenden Voxel mit der entsprechenden Distanz aktualisiert werden. Für die Echtzeitrekonstruktion ist dieses Vorgehen jedoch zu aufwändig. Hier werden die Voxel nicht anhand der exakten euklidischen Distanz aktualisiert, sondern es wird mithilfe des Raycastings, ausgehend von der Tiefenkamera, eine projizierte Distanz als Approximation verwendet. \citep{Compu66:online} Später bei der detaillierten Beschreibung von Chisel (Kapitel \ref{sec:chisel}) wird das Vorgehen auch nochmal Grafisch in Abbildung \ref{fig:tsdf-sketch} erläutert.

\subsection{Spatial Hashing}

Das Problem der Echtzeitrekonstruktion ist wie bereits angesprochen der Kompromiss zwischen dem Detailgrad, der Skalierung der zu rekonstruierenden Szene und der Performance der Rekonstruktion. Auch die TSDF-Repräsentation ist sehr speicherintensiv und benötigt für die zu scannende Szene reservierten Speicher, der auf mobilen Endgeräten nur begrenzt verfügbar ist. Daher muss auch für größere Rekonstruktionen oder Rekonstruktionen unbekannter Größe ein dynamischer Ansatz gefunden werden, Speicher zu verwalten. \citet{Klingensmith_2015_7924} erwähnt dazu, dass einige Verfahren Octrees einsetzen, die zwar äußerst dynamisch sind, jedoch einen deutlichen Nachteil hinsichtlich der Zugriffszeiten auf die Voxel bergen. 

\begin{figure}[h]
  \centering
	\includegraphics[width=1.0\textwidth]{content/images/methods/hashing.png} 
  \caption{a) Voxel Hashing Datenstruktur. Übernommen von \citet{niessner2013real} b) Darstellung der relevanten Voxel Chunks für die Aktualisierung. Übernommen von \citet{Klingensmith_2015_7924}}
  \label{fig:hashing}
\end{figure}

\citet{niessner2013real} führen daher eine zwei-Ebenen Struktur ein, die auf der zweiten Ebene eine Menge von Voxel räumlich zusammenfassen. Diese werden hier Chunks genannt. Auf der ersten Ebene können diese Chunks in einer Hash Tabelle räumlich mit einer Hashfunktion identifiziert werden. Das ermöglicht somit einen nahezu direkten Zugriff auf räumliche Voxel und bietet zudem Chunks dynamisch zu allokieren. Als Hash der Chunkposition \(x\), \(y\) und \(z\) wird die folgende Hashfunktion aus Gleichung \ref{eq:spatial_hash} verwendet. Bei den Variablen \(p_1\), \(p_2\) und \(p_3\) handelt es sich um willkürlich hohe Primzahlen und \(n\) entspricht der Größe der Hash Tabelle. Da Kollisionen in der Hash Tabelle nicht vollkommen ausgeschlossen werden können, werden die Chunks gegebenenfalls in der Hashtabelle verkettet.

\begin{equation}\label{eq:spatial_hash}
H(x,y,z) = (x * p_1 \oplus y * p_2 \oplus z * p_3) \mod n
\end{equation}

\subsection{Marching Cubes}

Die meisten Echtzeitrekonstruktionen durch TSDF wie KinectFusion sind GPU-Umsetzungen. Diese haben daher die Möglichkeit ein hardwarebeschleunigtes Rendering durch Raycasting durchzuführen. Das Verfahren Chisel von \citet{Klingensmith_2015_7924}, welches eine reine CPU Umsetzung ist, nutzt hingegen einen indirekten Weg zum Rendering durch die Marching Cubes Triangulation. 

\begin{figure}[h]
  \centering
	\includegraphics[width=1.0\textwidth]{content/images/methods/marchingcubes.png} 
  \caption{a) Marching Cubes Voxel Repräsentation mit den Ecken und den Kantenschnittpunkten b) Die 15 möglichen 3D Polygon Varianten. Übernommen von \citet{MarchingCubes:online}}
  \label{fig:marchingcubes}
\end{figure}

Marching Cubes nach \citet{lorensen1987marching} ist ein Algorithmus um aus einer, als Voxel repräsentierten, Isofläche Polygone zu bestimmen, die dieser Fläche möglichst nahe kommt. Hierzu werden zu jedem Voxel die Ecken \(v1\) bis \(v8\) anhand der Nachbarvoxel und des Distanzwertes untersucht, ob sie innerhalb oder außerhalb eines Objektes liegen. Zusätzlich werden zu jeder Kante auf dem Voxel, wenn ein Schnitt der Isofläche existiert, die Schnittpunkte auf den Kanten \(e1\) bis \(e12\) bestimmt. Abbildung \ref{fig:marchingcubes} a) zeigt die Ecken und Kantenschnittpunkte eines Voxels. 

Je nach binärer Gewichtung der Ecken können hiernach aus einem Katalog von 255 Varianten die Polygone nachgeschlagen werden. Inhalt des Katalogs sind die Indizes der Kantenschnittpunkte \(p_{e1}\) bis \(p_{e12}\), aus denen Polygone generiert werden können. Alle diese 255 Varianten können auf 15 verschiedene Fälle zurückgeführt werden, die sich nur in Rotation oder Symmetrie unterscheiden. Die 15 Varianten sind in Abbildung \ref{fig:marchingcubes} b) zu finden. \citep{MarchingCubes:online} 

\subsection{Chisel mit Space Carving} \label{sec:chisel}

Das bereits erwähnte Verfahren Chisel von \citet{Klingensmith_2015_7924} verwendet alle zuvor erwähnten Techniken der TSDF, Spatial Hashing und der Marching Cubes Überführung. Sie sprechen dabei von einem \enquote{dynamic spatial-hashed truncated distance field}. Das für den mobilen Einsatz optimierte Verfahren ist in der Lage eine Echtzeitrekonstruktion von Räumen von bis zu \(300 m^2\) mit einem Detailgrad von zwei bis drei Zentimetern zu erstellen. Zudem können neben Tiefeninformationen auch gefärbte Pointclouds verarbeitet werden, wodurch ein gefärbtes Mesh durch Marching Cubes generiert werden kann. Die Farbinformationen sind jedoch auf die Voxel Auflösung des Verfahrens beschränkt. \citep{Klingensmith_2015_7924}

Zusätzlich erweitern sie den TSDF Algorithmus um die \enquote{space carving} Funktionalität. Sie betrachten dabei den Strahl von der Kamera zur Oberfläche als eine Art Bedingung, in der alle durchstoßenden Voxel bis zur Oberfläche eine negativen Wert beinhalten müssen. Ist das nicht der Fall, so wird ein Voxel außerhalb der inneren Begrenzung auf den leeren Ursprungswert gesetzt. Der Pseudocode in Listing \ref{lst:chisel} sowie die Abbildung \ref{fig:tsdf-sketch} erläutern dieses Verhalten. Diese Verbesserung führt dazu, dass die Rekonstruktion bei stark rauschenden Tiefeninformationen, besonders an Objektkanten, deutlich verbessert wird. Außerdem ist das Verfahren hiermit in der Lage dynamische Änderungen in der Umgebung zu detektieren und neue Oberflächen entsprechend zu aktualisieren. So beeinflusst, als Beispiel, sich eine im Bild bewegende Personen nur kurz die Voxel der TSDF. \citep{Klingensmith_2015_7924}


\begin{lstlisting}[mathescape,caption=Chisel TSDF Algorithmus, label=lst:chisel, float=htbp]
Eingabe: Pointcloud $C$, Kameratransformation $P_{cam}$, 
         Strahlenbegrenzung $t$

für jeden Tiefenwert $\vec{p}$ aus $C$
    bestimme die Oberflächenposition $\vec{z}$ aus $\vec{p}$ und $P_{cam}$
    bestimme einen Strahl $\vec{r}$ aus $\vec{z}$ und $P_{cam}$
    bestimme den Begrenzungsbereich $t_{vor}$, $t_{nach}$ mit $t$ um $\vec{z}$ auf $\vec{r}$
    # space carving
    für jeden Voxel $v$ zwischen Kamera und $t_{vor}$
        wenn die Distanz im Voxel negativ ist
            setze Voxel zurück
    # normale TSDF Bestimmung
    für jeden Voxel $v$ zwischen $t_{vor}$ und $t_{nach}$
        bestimme die Voxeldistanz zu $z$
        setze das Gewicht w des Voxels $v$
\end{lstlisting}

\begin{figure}[h]
  \centering
	\includegraphics[width=1.0\textwidth]{content/images/methods/tsdf-sketch.png} 
  \caption{Exemplarische Visualisierung des Strahls $\protect\vec{r}$ und der zu aktualisierenden Voxel $\protect v$ im TSDF Algorithmus. Links ist hier das Project Tango Gerät zu erkennen, welches Tiefeninformationen der Oberfläche eines Objekts rechts aufgenommen hat.}
  \label{fig:tsdf-sketch}
\end{figure}

Neben space carving wurden zudem eine variable Strahlenbegrenzungen und Gewichtungen der Voxel, abhänig von der jeweils aufgenommenen Tiefe, implementiert. Diese Funktion berücksichtigt die Messungenauigkeiten des Tiefensensors, die bei größerer Entfernung der Oberfläche zum Tiefensonsor zunehmen können. \citep{Klingensmith_2015_7924}


\section{Tiefenanpassungen durch Farbbilder}

Aus allen zuvor beschriebenen Verfahren werden letztendlich Tiefeninformationen, in Form von geometrischen Primitiven oder Punkten im Raum gewonnen. Diese werden passend zur aktuellen Kameraposition als Tiefenbild gerendert und füllen den Z-Buffer für entsprechende Aussparungen bei der Überdeckung virtueller Objekte. Auf Grund von Sensorungenauigkeiten und größeren Auflösungen der Rekonstruktionsverfahren können dabei fehlerhafte Tiefeninformationen in den Z-Buffer gelangen, die zu Fehlern bei der Bestimmung der Überdeckung führen können. Dieses Problem ist am Beispiel der Pointcloud Projektion aus Kapitel \ref{sec:pc-projection} in Abbildung \ref{fig:pc-noise} zu erkennen. 

\begin{figure}[h]
  \centering
	\includegraphics[width=1.0\textwidth]{content/images/methods/pc-noise.png} 
  \caption{Überdeckung mit einfacher Pointcloud Projektion. Links: Resultat der Überdeckung. Mitte: Darstellung des Z-Buffers mit dem Ausschluss des virtuellen Objekts. Rechts: Darstellung der Pointcloud.}
  \label{fig:pc-noise}
\end{figure}

Die Reduktion von Ungenauigkeiten im Tiefenbild könnte durch einen einfachen Weichzeichner erreicht werden. Dieser würde jedoch die Kanten im Farbbild nicht berücksichtigen und somit fehlerhafte Tiefengradienten an den Kanten erzeugen und einen durchaus größeren Fehler generieren. \citet{newcombe2011kinectfusion} wenden einen sogenannten \enquote{Bilateralen Filter} in ihrem KinectFusion Rekonstruktionsverfahren an, bevor sie die Tiefeninformationen in die TSDF Repräsentation einfließen lassen. Dieser Filter von \citet{tomasi1998bilateral} ermöglicht das Weichzeichnen ohne dabei die Kanten im Bild zu übergehen, bezieht sich jedoch nur auf das selbe Bild, auf dem der Filter angewendet wird. 

\citet{liu2012guided} hingegen wenden einen sogenannten \enquote{Guided Filter} in ihrem Verfahren zur Optimierung der Tiefeninformationen für Kinect ähnliche Sensoren auf das Tiefenbild an. Dieser Filter von \citet{he2010guided} ist in der Lage, auf Grundlage eines anderen Leitbildes ein Weichzeichnen durchzuführen, ohne dabei die Kanten des Leitbildes zu überschreiten. Auch wenn \citet{petschnigg2004digital} eine Erweiterung, den Joint Bilateral Filter, vorstellen, der auf Basis eines anderen Leitbildes eine Weichzeichnung ohne Kantenüberschreitung ermöglicht, bietet der Guided Filter eine deutlich bessere Performance. Außerdem verhindert der Guided Filter Fehlerartefakte im Resultat, die bei dem Bilateralen Filter an den Kanten auftreten können. \citep{he2010guided} 

Ausgehend von der Eingangsgrafik \(p\), einem Leitbild \(I\) und dem Ergebnisbild \(q\) wird das grundlegende Modell von dieser Art Filter mit der Gleichung \ref{eq:gf-model} beschrieben. Diese Gleichung findet für jeden Pixel \(i\) in \(q\) eine gewichtete Summe über jeden Pixel \(j\) einer vordefinierten Ausschnittgröße. \(W_{ij}\) entspricht dabei dem Gewicht, welches für die jeweiligen Pixel \(p_j\) gilt. Bei dieser Faltung gilt üblicherweise  \(\sum_{j} W_{ij}(I)=1  \text{ für } \forall i \in [1\ldots |p|]\). \citep{he2010guided}

\begin{equation} \label{eq:gf-model}
q_{i} = \sum_j W_{ij}(I)p_j
\end{equation}

Der Filterkern \(W_{ij}(I)\) des Guided Filter, zu finden in Gleichung \ref{eq:gf-W}, ist, wie auch beim bilateralen Filter, abhängig von einem Leitbild \(I\), um die Gewichte entsprechend den Kanten des Leitbildes an der Position ermitteln zu können. Die Variablen \(\mu_k\) und \(\sigma^2_k\) beschreiben jeweils den Mittelwert und die Abweichung des Leitbildes im Bildausschnitt \(w_k\). \(|w|\) entspricht der Pixelgröße des Ausschnitts. \citep{he2010guided}

\begin{equation} \label{eq:gf-W}
W_{ij}(I) = \frac{1}{|w|^2} \sum_{k:(i,j) \in w_k} (1+\frac{(I_i-\mu_k)(I_j-\mu_k)}{\sigma^2_k + \epsilon})
\end{equation}

Dieser Filterprozess wird auch als eine translationsabhängige Faltung bezeichnet, die üblicherweise aufwändig ist und dessen Berechnungsaufwand abhängig zur Filterkern Größe (\(|w|\)) ist. \citet{he2010guided} stellen jedoch noch eine andere Definition des Filters zur Verfügung, in denen alle Summen der Form \(\sum_i\in w_k f_i\) entsprechen und dadurch mit der Bildintegrationstechnik von \citet{crow1984summed} in \(O(N)\) gelöst werden können. Der Guided Filter wird in der letztendlichen Implementierung nach Gleichung \ref{eq:gf-final} implementiert, in der die Koeffizienten \(\overline{a}_i\) und \(\overline{b}_i\) dem Mittelwert über \(a_k\) aus Gleichung \ref{eq:gf-a} und \(b_k\) aus Gleichung \ref{eq:gf-b} für jedes Fenster \(w_k\) entspricht. So wird auch \(\overline{p}_k\) durch \(\frac{1}{|w|} \sum_{i \in w_k} p_i\) berechnet.

\begin{equation} \label{eq:gf-final}
q_i = \overline{a}_iI_i+\overline{b_i}
\end{equation}

\begin{equation} \label{eq:gf-a}
a_k = \frac{\frac{1}{w} \sum_{i \in w_k} I_i p_I - \mu_k \overline{p}_k}{\sigma_k^2+\epsilon}
\end{equation}

\begin{equation} \label{eq:gf-b}
b_k = \overline{p}_k - a_k\mu_k
\end{equation}

Der Parameter \(\epsilon\) reguliert im beschriebenen Filter von \citet{he2010guided} welcher Bildanteil als beizubehaltende Kante im resultierenden Bild gewertet werden soll und somit stärker oder schwächer in der Gewichtung \(W_{ij}\) Einfluss nimmt. Neben diesem Regulierungsfaktor ist auch die Wahl des Radius \(r\) für den Ausschnitt \(w_k\) als Eingabe für diesen Filter wichtig. Der Radius wirkt sich laut \citet{he2010guided} jedoch nicht wie beim bilateralen Filter auf die Laufzeit des Filters aus. 

\begin{quote}
\enquote{One more advantage of the guided filter over the bilateral filter is that it automatically has an \(O(N)\) time exact algorithm. \(O(N)\) time implies that the time complexity is independent of the window radius \(r\), so we are free to use arbitrary kernel sizes in the applications.} \citep{he2010guided}
\end{quote}

\begin{figure}[h]
  \centering
	\includegraphics[width=1.0\textwidth]{content/images/methods/gf-result.png} 
  \caption{Guided Filter Anwendungsbeispiel. Das Tiefenbild links ergibt durch den Guided Filter mit dem Leitbild in der Mitte das Ergebnis im rechten Bild.}
  \label{fig:gf-result}
\end{figure}

Mit einer Komplexität von \(O(N)\) findet dieser Filter erfolgreich Anwendung in verschiedensten Bereichen. Er wird zum Beispiel zur Rauschunterdrückung, dem Weichzeichnen oder Verstärken von Details, zur HDR Kompression, dem Entfernen von matten Bildeigenschaften oder, wie in diesem Fall, zum zusammengeführten Anreichern von Bildinformationen verwendet \citep{he2010guided}. Angewendet auf das ermittelte Tiefenbild kann dieser Guided Filter, mit dem jeweiligen RGB Bild als Leitbild, ein Rauschen eliminieren und die Kanten der Tiefeninformationen durch einen entsprechend groß gewählten Fensterradius \(r\) und Regulierungsfaktor \(\epsilon\), an die Kanten der Kameraaufnahme angleichen \citep{liu2012guided}. Ein Beispiel für eine erfolgreiche Anwendung dieses Filters ist in Abbildung \ref{fig:gf-result} zu sehen.





% Umsetzung
\chapter{Umsetzung der Verfahren}

Dieses Kapitel widmet sich der Umsetzung der in der Theoretischen Vorbemerkung erwähnten Verfahren zur Planaren oder Pointcloud Rekonstruktion sowie den technischen Problemstellungen bezüglich Project Tango.

\section{Project Tango: Technische Grundlagen}

\section{Echtzeit Verdeckung durch Depth Maps}

\input{content/implementation/marching_cubes}

\section{Planare Rekonstruktion}

Wie bereits in Absatz \ref{sec:polygon_reconstruction} beschrieben, lässt sich das Problem der Optimierung von Augmented Reality mit Hilfe von Tiefeninformationen auf eine Echtzeit Rekonstruktion zurückführen. Im Gegensatz zur Rekonstruktion komplexer Oberflächen, mit dem vorgestellten TSDF Verfahren, soll hier eine Idee näher erläutert werden, die eine Rekonstruktion allein auf planaren Primitiven ermöglicht. \citet{yang2010plane} erwähnt hierzu, dass Ebenen in fast allen künstlichen Umgebungen zu finden sind und auf Grund ihrer vorteilhaften geometrischen Eingenschaften in verschiedensten Computer Vision Verfahren verwendet werden. Daher gibt es viele Forschungsarbeiten, Methoden und Algorithmen um aus verschiedensten Informationsquellen ein Ebenenmodell zu extrahieren.\\

Wie in dem \enquote{Simultaneous Localization and Mapping} (SLAM) Verfahren von \citet{trevor2012planar} wird hier zunächst eine Ebene in der Pointcloud mit Hilfe des RANSAC Algorithmus gesucht. RANSAC bietet gegenüber anderen Algorithmen zur Ebenen Detektion den Vorteil, ein Modell auch bei vielen Ausreißern performant zu ermitteln. Agglomeratives Clustering und Region Growing wie von \citet{feng2014fast} beschreiben, eignet sich auf Grund des Ausgabeformats aus Project Tango nicht, da es keine organisierte Point Cloud ausgibt. \\

Die Repräsentation der Ebene \(P\) wird, angelehnt an das Vorgehen von \citet{trevor2012planar}, wie in Gleichung \ref{eq:plane} festgehalten. Dabei handelt es sich um den Normalenvektor \(\vec{n}\) und der Distanz zum Ursprung \(d\) der Hesse Normalform einer Ebene, sowie der Punkte der konvexen Hülle \(H\). Um die konvexe Hülle der Ebene zu bestimmen, wird der Graham Scan Algorithmus verwendet. Wie auch von \citet{trevor2012planar} beschreiben wird die konvexe in der Repräsentation festgehalten, um eine sukzessive Verbesserung einer Ebene nach mehreren Messdurchläufen zu ermöglichen. So werden die Punkte der konvexen Hülle pro Messvorgang kombiniert, damit die Ebenenausbreitung auch außerhalb des Sichtfeldes beibehalten werden kann.

\begin{equation} \label{eq:plane}
P=\left[\vec{n}, d, H\right] \qquad H=\vec{h_1}, \vec{h_2}, \ldots  \vec{h_n}
\end{equation}

Um wiederum aus dieser Repräsentation eine Triangulation zu erhalten, wird hier die zweidimensionale Sweep-Line Delaunay Triangulation durchgeführt. Die Schritte aus dem Algorithmus \ref{lst:planeReconstruction} werden in den folgenden Kapiteln näher beschreiben.

\begin{lstlisting}[mathescape,caption=Planaren Echtzeit Rekonstruktion, label=lst:planeReconstruction]

Eingabe: Octree $O$
Ausgabe: Polygonpunkte $T_{Gesamt}$

für jedes Cluster $C$ aus $O$
    bestimme Ebene [$\vec{n}$, $d$, $P$] mit RANSAC aus $C_{Punkte}$
    wenn keine Ebene mit genügend $P$ in $C_{Punkte}$ gefunden wurde
    		nächstes Cluster (continue)
    wenn Ebene mit [$\vec{n}$, $d$, $H_{alt}$] in $C_{Ebenen}$ existiert	
        füge die konvexe Hülle $H_{alt}$ zu $P$ hinzu	
    bestimme die konvexe Hülle $H_{neu}$
    bestimme die Tringulation $T_{Ebene}$ aus $H_{neu}$
    $T_{Gesamt}$ += $T_{Ebene}$
    $C_{Ebenen}$ += [$\vec{n}$, $d$, $H_{neu}$]
    $C_{Punkte}$ - $P$
		

\end{lstlisting}

\subsection{RANSAC zur Ebenendetektion}

Der \enquote{RAndom SAmple Consensus} Algorithmus (RANSAC), vorgestellt von \citet{fischler1981random}, ist in der Lage, aus einer Menge von Daten mit vielen Ausreißern, die Parameter für ein passendes Modell zu schätzen. Anders als andere Schätzverfahren wie \enquote{Least-Median} oder \enquote{M-Schätzer}, welche aus der Statistik Literatur entnommen und entsprechend angepasst wurden, wurde RANSAC speziell für die Anwendung in der Computer Graphik entwickelt. Der Kern dieses Algorithmus ist das wiederholte Bestimmen eines Modells aus zufälligen und für das Modell ausreichenden Stichproben. Listing \ref{lst:ransac} zeigt den Verlauf des RANSAC Algorithmus. Die Anzahl der Iterationen \(N\) hängt dabei allein von dem Anteil der Ausreißer in den Messwerten ab. Daher sollte sie entsprechend gewählt werden, um die Wahrscheinlichkeit zu verringern, dass Ausreißer in den Stichproben enthalten sind. \citep{derpanis2010overview} \\

\begin{lstlisting}[mathescape,caption=Der RANSAC Algorithmus, label=lst:ransac]
Eingabe: Messwerte $P$, Modelltoleranz $e$, maximale Iterationen $N$
Ausgabe: Modell $m$, Unterstützende Messwerte $P_m$

1. Wähle zufällig so viele Stichproben aus den Messwerten $P$,
   wie nötig sind, um das Modell zu bestimmen
2. Bestimme aus den gewählten Stichproben das Modell $m$
3. Ermittle die Anzahl der Messwerte $P$, die mit einer 
   entsprechenden Toleranz $e$ das ermittelte Modell $m$ 
   unterstützen
4. Wenn prozentual genügend Messwerte aus $P$ das Modell $m$ 
   unterstützen, ermittle aus den unterstützenden Messwerten 
   $P_m$ durch lineare Regression erneut das finale Modell 
   $m$ und terminiere
5. Wiederhole die Schritte 1-4 $N$ mal
\end{lstlisting} 

Um mit dem RANSAC Algorithmus Ebenen in einer Punktewolke bestimmen zu können, werden pro Iteration drei Stichproben \(A\), \(B\) und \(C\) gewählt. Das Ebenenmodell, hier in der Hesse Normalform mit dem Normalenvektor \(\vec{n}\) und dem Abstand zum Koordinatenursprung \(d\), lässt sich dabei durch die Gleichung \ref{eq:normalform} bestimmen.

\begin{equation}\label{eq:normalform}
\vec{n} =\left|\left| \vec{AB} \times \vec{AC}\right|\right|
\qquad
\vec{D} = \vec{A} \cdot \vec{n}
\qquad
d = D_1 + D_2 + D_3
\end{equation}

Um zu ermitteln ob ein Punkt \(P\) aus einer Messreihe die gefundene Ebene \(\left[\vec{n_E}, d_E\right]\) unterstützt, wird die kürzeste Distanz \(d_P\) zwischen Punkt und Ebene wie in Gleichung \ref{eq:plane-distance} ermittelt.  Ein entsprechender Toleranzwert für die Distanz \(d_{min}\), im gezeigten RANSAC Algorithmus \(e\) genannt, wird später bei der Umsetzung abhängig vom Rauschen des Tiefensensors gewählt. 

\begin{equation} \label{eq:plane-distance}
d_P = n_1*P_1+n_2*P_2+n_3*P_3-d_E \qquad support_{d_P} = d_P < d_{min}
\end{equation}

Um das finale Modell der Ebene zu ermitteln, und somit die Varianz des Abstands der Punkte zur Ebene zu minimieren, wird mit Hilfe der unterstützenden Punkte \(P_{support}=\left[x,y,z\right]\) eine lineare Regression durchgeführt. Diese Mittelt ein Ebenenmodell \(E=\left[a,b,c\right]\) aus den zuvor ermittelten Punkten mit Hilfe des \enquote{least squares} Schätzverfahren. Für eine Ebene versucht man die Funktion \(G(a,b,c)\) aus Gleichung \ref{eq:least-squares} zu minimieren. Hierzu muss das lineare Gleichungssystem in Gleichung \ref{eq:least-squares-solution}  gelöst werden. \citep{Regre94:online}

\begin{equation} \label{eq:least-squares}
z = ax + by + c   \qquad G(a, b, c) = \sum {\left(z_i - ax_i - by_i - c\right)}^2
\end{equation}

\begin{equation} \label{eq:least-squares-solution}
\begin{bmatrix}
\sum x_i^2 & \sum x_iy_i & \sum x_i\\ 
\sum x_iy_i  & \sum y_i^2  & \sum y_i \\ 
\sum x_i & \sum y_i  & n
\end{bmatrix}
*
\begin{bmatrix}
a\\ 
b\\ 
c
\end{bmatrix}
=
\begin{bmatrix}
\sum x_iz_i\\ 
\sum y_iz_i\\ 
\sum z_i
\end{bmatrix}
\end{equation}

\subsection{Bestimmung der Ebenenausbreitung}

Nachdem die Ebene und die korrespondierenden Punkte zur Ebene gefunden wurden, muss noch die Ausbreitung der Fläche bestimmt werden, da die Ebene in Hesse Normalform lediglich die Position \(\vec{n} * d\) und Ausrichtung \(\vec{n}\) festhält. \citet{PlanarSurfaceMapping} nutzt hierfür die konvexe Hülle der korrespondierenden Punkte und trianguliert diese. Um das performant umzusetzen, kann man sich hier die Eigenschaft der Ebene zu Nutzen machen und die dreidimensionalen Punkte durch Parallelprojektion als zweidimensionale Punkte auf die Ebene projizieren. Denn die Algorithmen für die Triangulation haben im zweidimensionalen eine deutlich besseres Laufzeitverhalten. \\

Nach der Triangulation können die Ecken der gefundenen Polygone jeweils zurück projiziert werden. Die Gleichungen \ref{eq:projection2d} und \ref{eq:projection3d} bilden die Projektion der Punkte wobei \(R_{\vec{n}to\vec{z}}\) der Rotationsmatrix zwischen dem Normalenvektor \(\vec{n}\) und der Z-Achse \(\vec{z}\) entspricht.\\

\begin{equation} \label{eq:projection2d}
p_{2d} = (p_{3d} - (\vec{n}*d)) * R_{\vec{n}to\vec{z}}
\end{equation}
\begin{equation} \label{eq:projection3d}
p_{3d} = (p_{2d} * R_{\vec{n}to\vec{z}}^{-1}) + (\vec{n}*d)
\end{equation}

\subsubsection{Convex Hull Algorithmus}

Für die Berechnung der konvexen Hülle wird der Graham Scan nach \citet{graham1972efficient} genutzt. Dieser Algorithmus besitzt eine Laufzeit von \(O(n \log n)\) und gilt als einer der populärsten Algorithmen für die Berechnung der konvexen Hülle. Andere Ansätze besitzen dabei ein ähnliches oder schlechteres Laufzeitverhalten. Gestartet wird der Algorithmus mit der Menge aller Punkte \(P\) der Ebene und mit einem Startpunkt \(P_0\), der Bestandteil der konvexen Hülle ist. Hierzu wird meist der Punkt mit dem niedrigsten \(y\) Faktor gewählt. (\(P_0=P_{min(y)}\)) Listing \ref{lst:graham-scan} zeigt den Verlauf des Algorithmus des Graham Scans. Dabei wird in Abbildung \ref{fig:convexhull} nochmals die erste Sortierung und das Unterscheidungskriterium für die Sortierung als auch für die Aussortierung der Punkte verdeutlicht. \citep{convexHull} \\

\begin{figure}
  \centering
	\includegraphics[width=0.9\textwidth]{content/images/methods/convexhull.png} 
  \caption{Sortierung der Punkte nach Winkel zum Startpunkt (links) mit dem Unterscheidungskriterium (rechts). Übernommen von \citet{convexHull}}
  \label{fig:convexhull}
\end{figure}

\begin{lstlisting}[mathescape,caption=Graham Scan Algorithmus, label=lst:graham-scan]
Eingabe: Menge der Punkte $P$, außen liegender Punkt $P_0$
Ausgabe: Punkte der konvexen Hülle

$i$ = 0
sortiere nach dem Winkel zu $P_0$
solange $i$ <= $|P|$
    wenn $\measuredangle P_{i-1} P_{i}$ > $\measuredangle P_{i-1} P_{i+1}$, also $P_i$ rechts von  $\vec{P_{i-1} P_{i+1}}$ liegt
        inkrementiere $i$
    ansonsten
        entferne $P_i$ aus $P$
        dekrementiere $i$
    
\end{lstlisting} 


\subsubsection{Triangulation}

Nachdem die konvexe Hülle bestimmt wurde, müssen die Punkte dieses Pfades noch trianguliert werden. Hierfür wird eine Delaunay-Triangulation mit Hilfe des Sweep-Line Algorithmus von \citet{domiter2008sweep} verwendet. Unter einer Delaunay-Triangulation versteht man zunächst einmal eine Art Constraint, der sogenannten Umkreisbedingung, die besagt, dass ein Kreis, der alle drei Punkte eines gefundenen Polygons durchzieht, keine weiteren Punkte beinhalten darf. \\

Die Sweep-Line Methode ist ein generelles geometrisches Vorgehen, bei dem eine vertikale imaginäre Linie eine Menge geometrischer Objekte von links nach rechts passiert. 

 ... to be continued

\subsection{Clustering der aufgenommenen Punkte}

Wie im Algorithmus \ref{lst:planeReconstruction} zu erkennen wird das zuvor beschriebene Vorgehen für die planare Rekonstruktion immer pro Cluster eines Cluster-Pools durchgeführt. Dadurch werden pro Durchgang des Algorithmus nur ein Bruchteil der gesammelten Punkte rekonstruiert, was wiederum eine Rekonstruktion in Echtzeit möglich macht. Außerdem verhindert das Clustering das Bilden von konvexen Hüllen über Ebenen, die in Zwischenbereichen nicht mit genügend Punkten unterstützt werden. Dieses Problem ist in Abbildung \ref{fig:clustering} links zu sehen.\\

\begin{figure}
  \centering
	\includegraphics[width=1.0\textwidth]{content/images/methods/clustering.png} 
  \caption{Links: Ebenenrekonstruktion ohne Clustering. Rechts: Rekonstruktion mit K-Mean Clustering.}
  \label{fig:clustering}
\end{figure}

Getestet wurde hier das K-Mean Clustering, Agglomeratives Clustering und einfaches räumliches Clustern mit Hilfe eines Octrees. Das K-Mean Clustering hat, wie in Abbildung \ref{fig:clustering} rechts zu erkennen, gute Ergebnisse für die Aufteilung einer Ebenen geliefert, benötigt aber zuvor eine feste Anzahl von Clustern. Agglomeratives Clustering, getestet mit dem euklidischen Distanzmaß, würde zwar die Anzahl der Cluster dynamisch bestimmen, ist jedoch zu aufwändig für eine echtzeit Rekonstruktion. Gute Ergebnisse liefert wiederum ein einfaches räumliches Clustern mit einem Octree. Das bietet zudem den Vorteil, dass diese Datenstruktur direkt als Speicherort der Aufgenommenen Punkte und Ebenen dienen kann. \\

Ein Octree ist zunächst eine Datenstruktur, die wie ein Baum mit beliebiger Tiefe aufgebaut ist und pro Knoten Acht Kinder besitzt. Dabei repräsentiert ein Knoten ein Würfel, der durch seine Kinder in Acht Kind-Würfel aufgeteilt wird. Durch diese räumliche Aufteilung ergeben sich verschiedene Vorteile gegenüber linearen Datenstrukturen. So müssen Bereiche zum festhalten räumlicher Informationen im Octree nur dann allokiert werden, wenn diese Bereiche auch verwendet werden. Speichert man Punkte in den untersten Knoten eines Octrees kann man durch eine Tiefenbegrenzung beim Zugriff auf den Baum ein sehr effektives Downsampling der Punkte vornehmen. Zuletzt entstehen durch die Knoten einer bestimmten Tiefe ein Cluster, zu denen in diesem Fall Punkte bei einer Aufnahme hinzugefügt werden und für eine weitere Verarbeitung extrahiert werden können.\\




% Tests
\chapter{Tests} \label{sec:evaluation}

In diesem Kapitel sollen die beschriebenen und prototypisch implementierten Verfahren zur Überlagerung gegenübergestellt werden, um anhand eines direkten Vergleichs eine objektive Aussage über die Qualität der Ergebnisse treffen zu können. Hierzu wird im ersten Teil das Vorgehen zum Testen vorgestellt, welches darauf folgend mit allen Verfahren umgesetzt wird. Hiernach werden die daraus resultierenden Ergebnisse gegenübergestellt.

\section{Statische Testszenen}

Zum Vergleich der Verfahren wurden zwei statische Szenen gewählt, in denen das Project Tango Gerät nicht bewegt wird und dadurch allen Kandidaten den selben sensorischen Inhalt bietet. Diese Wahl wurde getroffen, um eine zuverlässige und reproduzierbare Informationsquelle für das Gerät zu schaffen. Denn die Reproduktion eines bewegten und dynamischen Szenarios ist für alle zu vergleichenden Verfahren nur sehr schwer möglich. \\

Eine Idee für ein dynamisches Testszenario war es, alle sensorischen Informationen der Hardware einmal aufzunehmen und eine reproduzierbare simulierte Umgebung dieser Daten zu schaffen. Technologien wie das Robot Operating System (ROS) würden dies ermöglichen, jedoch übersteigt der Aufwand den zeitlichen Rahmen dieser Arbeit. Auch wenn die Firma Bosch eine exemplarische Implementation\footnote{Tango Output to Rosbag Files - https://goo.gl/hhnciZ} für die Aufnahme aller Daten in ROS demonstriert, sind die implementierten Verfahren zu sehr in den API Zyklen der Project Tango Schnittstelle involviert, um diese in kurzer Zeit auf eine Desktop Umgebung zu portieren.\\

Die erste gewählte Szene, welche in Abbildung \ref{fig:static-scene} links zu sehen ist, beinhaltet einen Hocker, in Form eines einfachen  Würfels, und einen Sitzball. Der Sitzball wurde gewählt, um auch runde Formen zur Tiefenaufnahme zu testen, welche gegebenenfalls für die Verfahren schwerer zu rekonstruieren sind. Das Project Tango Gerät ist etwas höher in einem Stativ plaziert. Das virtuelle Objekt wird, wie in Abbildung \ref{fig:static-scene} rechts, zwischen die beiden realen Objekte plaziert, sodass es von beiden Seiten durch die realen Objekte überdeckt wird. \\

\begin{figure}[h]
  \centering
	\includegraphics[width=1.0\textwidth]{content/images/evaluation/static-scene.png} 
  \caption{Links: Erste statische Szene mit einem Hocker und einem Sitzball. Rechts: Platzierung des virtuellen Objekts. }
  \label{fig:static-scene}
\end{figure}

Die zweite gewählte Szene, welche in Abbildung \ref{fig:plant-scene} links zu sehen ist, soll als Herausforderung die Überdeckung von komplexeren Strukturen testen. Sie besteht daher aus einer Pflanze, die sich, wie rechts im Bild zu sehen, vor dem virtuellen Objekt befindet. Auch hier befindet sich das Project Tango Gerät in einem Stativ, damit sich die Position während der Tests durch Motion Tracking nicht ändert. \\

\begin{figure}[h]
  \centering
	\includegraphics[width=1.0\textwidth]{content/images/evaluation/plant-scene.png} 
  \caption{Links: Zweite statische Szene mit einer Pflanze im Vordergrund. Rechts: Platzierung des virtuellen Objekts hinter der Pflanze. }
  \label{fig:plant-scene}
\end{figure}

Für beide Szenen sollen alle Kombinationen der Verfahren getestet werden. Somit ergeben sich sechs verschiedene Kombinationen, in denen die Pointcloud Projektion, die TSDF Rekonstruktion und die Ebenen Rekonstruktion jeweils mit und ohne der Anwendung des Guided Filter auf das Tiefenbild getestet werden. Für alle Kombinationen soll ein gerendertes Bild und ein Tiefenbild mit dem virtuellen Objekt festgehalten werden. Zur Auswertung werden die jeweils gerenderten Ergebnisbilder \(p\) mit einem manuell zugeschnittenem Ergebnisbild  \(q\) für jeden Pixel \(i\) verglichen. Für diese Gegenüberstellung wird die Summe der absoluten Bilddifferenz wie in Gleichung \ref{eq:diff} bestimmt.

\begin{equation} \label{eq:diff}
d = \sum_i |p_i-q_i|
\end{equation}

\section{Durchführung der Tests}

Die beiden Testszenen konnten wie beschrieben aufgebaut und mit allen Verfahren durchgetestet werden. Hierzu wurde mit der \enquote{Android Debug Bridge} (adb)\footnote{Android Debug Bridge - http://goo.gl/ffH51x (01.03.16)} eine Videoaufnahme gestartet, in der im Prototypen für jede Szene alle Verfahren durchgeschaltet wurden. Die Verfahren mussten dabei sehr schnell gewechselt werden, um einen potentiellen Drift von Project Tangos Motion Tracking so minimal wie möglich zu halten. Denn diese Bewegungen würden die Ergebnisse stark beeinträchtigen. \\

Bei der Anwendung des Guided Filters wurde immer der Radius \(r = 25\) und der Einflussfaktor \(\epsilon = 0.8\) gewählt. Diese Werte versprachen nach einigen manuellen Tests erfolgreiche Ergebnisse. Der Radius wurde so groß gewählt, damit die bei den Rekonstruktionen üblichen Fehler auch weitreichend in den Gewichtungskernel berücksichtigt werden können. Abbildung \ref{fig:static_occlusion} und \ref{fig:plant_occlusion} im Anhang zeigen jeweils die aus dem Video extrahierten Bildausschnitte. Die obere Reihe zeigt die drei tiefengenerierenden Verfahren ohne den Guided Filter und die untere Reihe jeweils mit dem Filter. In der untersten Reihe sind jeweils die Projektion der generierten Primitiven in der Szene zu sehen, um sich eine Vorstellung der Rekonstruktion machen zu können.

\section{Auswertung der Ergebnisse}

Der Vergleich der Ergebnisse, welcher mit dem bereits beschriebenen Bilddifferenz Ansatz aus der Gleichung \ref{eq:diff} durchgeführt werden soll, wurde mit Hilfe der OpenCV Bibliothek durchgeführt und ist in Listing \ref{lst:compare} im Anhang zu finden. Für jede Szene wurde ein Referenzbild manuell konstruiert, welches dem idealen Ergebnis entsprechen soll. Zur Konstruktion der Vergleichsbilder wurden Ausschnitte  mit leichten Anpassungen aus einigen Ergebnisbildern gewählt und zusammengeschnitten. Diese Referenzbilder sind in Abbildung \ref{fig:reference} zu finden. Mit dem Referenzbild wurden alle zuvor passend zugeschnitten Grafiken einer Szene verglichen. \\

\begin{figure}[h]
  \centering
	\includegraphics[width=0.7\textwidth]{content/images/evaluation/reference.png} 
  \caption{Manuell konstruierte Referenzbilder der idealen Überlagerung mit Filtermaske in Szene 1 (links) und 2 (rechts).}
  \label{fig:reference}
\end{figure}

Der gezeigte Skript zum Verglich der Ergebnisbilder mit den Referenzbildern ergibt neben den Differenzwerten, welche in Tabelle \ref{tab:results} zu finden sind, auch die absoluten Differenzbilder für jedes Verfahren. Diese Ergebnisbilder sind auch im Anhang in Abbildung \ref{fig:static_occlusion_results} zu finden. In der Abbildung werden beide Szenen mit je zwölf Bildern dargestellt, wobei jedes der drei Überlagerungsverfahren (horizontal) mit und ohne Anwendung des Guided Filter (vertikal) festgehalten sind. Sie sind also genau wie die Ergebnisse in Tabelle \ref{tab:results} angeordnet.\\


\begin{table}[h]
\centering
\begin{tabular}{@{}rrrr@{}}
\toprule
                      & \textbf{\begin{tabular}[c]{@{}r@{}}Pointcloud \\ Projektion\end{tabular}} & \textbf{\begin{tabular}[c]{@{}r@{}}TSDF \\ Rekonstruktion\end{tabular}} & \textbf{\begin{tabular}[c]{@{}r@{}}Ebenen \\ Rekonstruktion\end{tabular}} \\ \midrule
\textbf{Szene 1}      & 1400 & 3165 & 2193   \\
\textbf{Szene 1 + GF} & 1506 & 85 & 1545 \\
\textbf{Szene 2}      & 2478 & 4007 & 2180 \\
\textbf{Szene 2 + GF} & 67 & 1876 & 1529 \\
\bottomrule
\end{tabular}
\caption{Distanzwert zwischen dem Referenzbild und den Ergebnisbildern in der jeweiligen Szene}
\label{tab:results}
\end{table}

3165
Neben diesen resultierenden Werten werden im Fazit Kapitel \ref{sec:conclusion} auch anhand einiger dynamischer Tests, mit mehr Bewegung, die implementierten und getesteten Verfahren eingeschätzt. Diese, für die Verfahren nicht reproduzierbaren und deshalb nicht im Detail dokumentierten, Tests beinhalteten eine ähnliche statische Szene, um die sich mit einem Seitenschritt bewegt wurde, um die Genauigkeit der Kantenüberdeckung bei Bewegung gegenüberstellen zu können.

% Fazit
\chapter{Fazit} \label{sec:conclusion}

\section{Evaluation}

Die implementierten Verfahren haben gezeigt, dass mit dem Ansatz der Tiefenbild Überdeckung von \citet{wloka1995resolving} eine Echtzeit Überdeckung virtueller Objekte auf der mobilen Project Tango Hardware erfolgreich umgesetzt werden kann. Dabei wird im Folgenden auf jedes Verfahren sowie ihrer Vor und Nachteile im Kontext der anderen Verfahren und auf Basis der durchgeführten Tests eingegangen. 

\subsection*{Pointcloud Projektion}

Die Überlagerung durch die Pointcloud Projektion bietet gegenüber den anderen Verfahren den Vorteil, dass sie zu jeder Zeit eine dynamische und aktualisierte Repräsentation der Tiefe der Szene liefert und somit auch Änderungen in der Szene sofort berücksichtigt. Außerdem ist das Verfahren nicht auf Clustergrößen beschränkt und kann dadurch auch auch komplexe Strukturen abbilden. Zu erkennen ist dies bei der Testszene zwei, bei der die Bilddifferenz zum optimalen Ergebnis am geringsten ist, obwohl die Pflanze eine komplexe Struktur besitzt.

Dadurch, dass die Pointcloud von Project Tango Fehler in Form von Ausreißern und einem gewissem Rauschen enthalten kann, spiegeln sich diese Fehler auch in der berechneten Projektion wieder. Das führt dazu, dass zum Beispiel die Kante einer realen Überlagerung durchgehend in Bewegung ist und Ihre Struktur mit jedem neuen Tiefenbild variiert. Ein weiteres Problem dieser Technik ist, dass sie, dadurch dass sie sich nur auf einen Datensatz pro Aufnahme bezieht, die Tiefe nur innerhalb des Messbereichs des Tiefensensors repräsentieren kann. Dadurch können Überlagerungen von realen Objekten innerhalb der ersten 50 Zentimeter und ab vier Metern nicht mehr bestimmt werden.

\subsection*{Ebenen Rekonstruktion}

Die Ebenen Rekonstruktion löst die Schwächen der Pointcloud Projektion in dem Sinne, dass sie die Ungenauigkeit der Tiefeninformation als eine Oberflächen Approximation mit Hilfe von RANSAC in Form von Ebenen abbildet. Hierdurch werden Ausreißer ignoriert und auch das Rauschen wird durch eine lineare Regression gemittelt. Zusätzlich ermöglicht das Vorgehen der Ebenen Rekonstruktion eine kontinuierliche Verbesserung, indem alle bereits aufgenommenen Pointclouds in die aktuelle Rekonstruktion einfließen. Auch wenn dieser Rekonstruktionsansatz durch den Octree eine Rekonstruktion in einer groben Struktur, den Clustern des Octrees, durchführt, erhält das Verfahren durch das Ermitteln der konvexen Hülle pro gefundener Ebene einen gewissen Detailgrad, um auch schwierige planare Strukturen abbilden zu können. Die gemessenen Ergebnisse der Szenen eins und zwei spiegeln diese positive Eigenschaft wieder und zeigen, dass diese Art der Rekonstruktion auch komplexe Szenen für dieses Testszenario gut abbilden kann.

In einem manuellen dynamischeren Test mit Bewegungen weißt dieses Verfahren jedoch einige Schwächen auf. So sind durch die begrenzte Dichte der Pointcloud Lücken zwischen den Ebenen zu sehen, die zwar von Aufnahme zu Aufnahme kleiner werden aber üblicherweise nicht komplett schließen. Das führt dazu, dass zum Beispiel große Oberflächen, die ein virtuelles Objekt überlagern, das Objekt vereinzelt nicht aussparen, da keine Tiefe an den Stellen durch Lücken zwischen den Ebenen vorhanden sind. Außerdem ist das Verfahren nur bedingt in der Lage runde Strukturen wie den Sitzball aus Szene eins zu rekonstruieren. Diese Fehler werden besonders dann sichtbar, wenn man sich um diesen Ball dreht und er eine Überlagerung mit den Ecken und Kanten der Ebenen aus der Rekonstruktion auf ein virtuelles Objekt ausübt. Neben der fehlenden Unterstützung für runde Konturen besitzt dieses Verfahren keine Möglichkeit, Messungen zu revidieren wenn reale Objekte in der Szene verändert wurden oder ein Drift Fehler von Project Tango auftritt.

\subsection*{TSDF Rekonstruktion}

Wie zu erwarten liefert Chisel als eine TSDF Implementierung, aufgrund der großen Voxelgröße, nicht die Qualität, die zum Beispiel ein KinectFusion liefern kann. Dafür ist es performant genug, um als eine CPU Implementierung eine Echtzeit Rekonstruktion auf der mobilen Project Tango Hardware zu ermöglichen. Genau wie die Ebenen Rekonstruktion bietet Chisel den Vorteil eine Rekonstruktion pro Tiefenbild anzureichern und stetig zu verbessern. Dadurch können Überlagerungen auch außerhalb des Messbereichs des Tiefensensors ermöglicht werden. Anders als bei der Ebenen Rekonstruktion generiert die TSDF Rekonstruktion stets eine geschlossene Oberfläche. Außerdem können runde Strukturen festgehalten werden, wodurch der in Szene eins stehende Sitzball abgebildet werden kann. 

Die große Voxelgöße führt jedoch dazu, dass wie in beiden getesteten Szenen zu erkennen, die Strukturen der Rekonstrutkion sehr grob ausfallen und die Differenzergebnisse ohne eine Filterung auf einen hohen Fehler hinweisen. Auch wenn Chisel nicht in der Lage ist so detailierte Kantenabbildungen wie die Ebenen Rekonstruktion zu generieren, besitzt Chisel einen Vorteil: Durch den Space Carving Mechanismus können Rekonstruktionen wieder revidiert werden. Das hilft dabei den Problematiken des Drift Effekts von Project Tango entgegenzuwirken. Außerdem könnte durch eine GPU Implementierung auch eine Echtzeit Rekonstruktion mit deutlich kleinerer Voxelgröße und dadurch höherem Detailgrad realisiert werden.


\subsection*{Guided Filter}

Der Guided Filter war in den Tests häufig in der Lage selbst grobe Fehler im Tiefenbild an die Kanten der Farbbilder anzugleichen und somit auch, wie in den Messergebnissen zu erkennen, den Differenzwert zum Optimum zu reduzieren. Jedoch führte der Einsatz des Filters zu deutlichen Performanceeinbußen, denn der Filterprozess selbst benötigt im Durchschnitt \(220\) ms. Der Einsatz von OpenCV erschwert zudem den Einsatz des Filters für die Echtzeit Umsetzung, da die Bildebene pro Bild aus dem OpenGL Framebuffer raus und wieder rein geladen werden muss. Dieser Prozess benötigt zusätzliche \(80\) ms, was die Wiederholrate der prototypischen Implementierung auf 3 Herz reduziert. 

Zusätzlich sind unter gewissen Umständen, bei denen ein virtuelles Objekt nah an der Oberfläche eines realen Objekts, aber immer noch räumlich hinter dem realen Objekt liegt, variable Artefakte aufgefallen, an denen das eigentlich überlagerte virtuelle Objekt durchschimmert. Dieser Effekt ist in Abbildung \ref{fig:artifacts} zu erkennen. Neben den eigentlichen Kanten für die Überlagerung im Farbbildes werden auch Kanten von flachen Strukturen berücksichtigt. Im Bild zu sehen, beeinflusst das Muster vom Würfel das resultierende Tiefenbild soweit, dass eine fehlerhafter Überlagerung nach der Filterung stattfindet. 
 
\begin{figure}[h]
  \centering
	\includegraphics[width=1.0\textwidth]{content/images/artifacts.png} 
  \caption{Fehlerhafte Überdeckung bei der Anwendung des Guided Filters. Links: Reales Objekt mit Rekonstruktion. Mitte: Tiefenbild. Rechts: Sichtbare Fehler nach Guided Filter.}
  \label{fig:artifacts}
\end{figure}

Angewendet auf die Tiefeninformationen der Pointcloud Projektion konnte der Filter in der komplexeren zweiten Szene nahezu das Optimum der Überlagerung erreichen. Schwieriger war jedoch der Einsatz bei der runden Kontur in der ersten Szene, wo der Filter nicht in der Lage war, den initialen groben Fehler vom Tiefensensor zu revidieren. Das selbe Verhalten ist auch bei der Ebenen Rekonstruktion in Szene eins zu beobachten. Besonders beachtenswert ist die Tatsache, dass bei den zu weit reichenden Tiefeninformationen der TSDF Rekonstruktion, in der Mitte der Messergebnisse von Szene eins, ein etwa gleichgroßer Fehler durch die Filterung behoben werden konnte. Eine weitere Besonderheit, die während der Tests beobachtet werden konnte, ist dass der Filter in der Lage war bei der Anwendung auf die Ebenen Rekonstruktion, die Lücken zwischen den Ebenen unkenntlich zu machen. Somit würde dieser Filter ein Nachteil dieses Ansatzes lösen.

\section{Einsatz der Verfahren}

Grundsätzlich ist festzuhalten, dass sich die Pointcloud Projektion ohne den Guided Filter, trotz des erreichbaren Detailgrads, höchstens in einzelnen Bildaufnahmen für eine Überlagerung in Augmented Reality eignet, da das Rauschen der Eingangsdaten durchgehend sichtbar ist. Außerdem ist die Sichtweite auf die Erreichbarkeit des Tiefensensors des aktuellen Ausschnitts begrenzt. Auch die Ebenen Rekonstruktion ist bedingt geeignet, da Lücken zwischen den Ebenen zu erkennen sind, die die Illusion von AR zerstören würde. Auch wenn die TSDF Rekonstruktion durch Chisel nach den statischen Testszenen oft mit Fehler behaftet sind, existieren entschiedene Vorteile gegenüber den anderen Vorgehensweisen. Denn durch Chisel werden geschlossene Flächen gebildet, welche sich dynamisch der Szene anpassen können. Betrachtet man zudem den Einsatz von Chisel in größeren Flächen, wie in Räumen oder sogar im ganzen Gebäude, fällt der gemessene Fehler weniger erkennbar aus.

Angenommen es gäbe die Möglichkeit den Einsatz des Guided Filters für jedes Verfahren in Echtzeit zu ermöglichen, so würde die Pointcloud Projektion durchaus Anwendung finden. Es könnte zum Beispiel in einer AR Applikation genutzt werden, die sich nur in einem bestimmten Sichtbereich bewegt, und die eine gewisse komplexe und dynamische Szene bedienen muss. Ausgehend von den Testergebnissen als Entscheidung zwischen der Ebenen Rekonstruktion und der TSDF Rekonstruktion mit dem Guided Filter würde, wie auch ohne Filter, Chisel die bessere Alternative sein.


\section{Ausblick}



Technologie
* Vorteile von Polygonen als Ausgangsbasis (Schatten, Interaktion)
* Filter Umsetzung in im Vertex/Fragment Shader
* 

Project Tango
* Bilateral Filter in API während Arbeit erschienen => Guided Filter wäre der
* API Änderungen - Zugang zu optimierten Chisel Umsetzungen
* Consumer Phase mit Lenovo Deployment
* Alternative Tiefensensoren LFC


	
	

\setlength{\parskip}{0em}

\appendix

\let\LaTeXStandardClearpage\clearpage
\let\clearpage\relax

\listoffigures
\lstlistoflistings  

\let\clearpage\LaTeXStandardClearpage

\chapter{Ergebnisaufnahmen}
\begin{sidewaysfigure}[h]
  \centering
	\includegraphics[width=1.0\textwidth]{content/images/evaluation/static_occlusion.png} 
  \caption{Ergebnisaufnahmen aus der ersten statischen Szene}
  \label{fig:static_occlusion}
\end{sidewaysfigure}

\begin{sidewaysfigure}[h]
  \centering
	\includegraphics[width=1.0\textwidth]{content/images/evaluation/plant_occlusion.png} 
  \caption{Ergebnisaufnahmen aus der zweiten statischen Szene}
  \label{fig:plant_occlusion}
\end{sidewaysfigure}

\begin{sidewaysfigure}[h]
  \centering
	\includegraphics[width=1.0\textwidth]{content/images/evaluation/static_occlusion_results.png} 
	\includegraphics[width=1.0\textwidth]{content/images/evaluation/spacer.png} 
	\includegraphics[width=1.0\textwidth]{content/images/evaluation/plant_occlusion_results.png} 
  \caption{Differenzbilder der Verfahren in ersten (oben) und zweiten Szene (unten)}
  \label{fig:static_occlusion_results}
\end{sidewaysfigure}


\addcontentsline{toc}{chapter}{Literaturverzeichnis}

\bibliography{main}
\bibliographystyle{natdin} 

\end{document}

