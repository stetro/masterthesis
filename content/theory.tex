\chapter{Theoretische Vorbemerkung}

\section{Augmented Reality}

Augmented Reality (AR) ist eine Klasse aus dem Realitäts-Virtualitäts-Kontinuum von \cite{milgram1995augmented}, zu finden in Abbildung \ref{fig:virtual-continuum}, welche reale und virtuelle Objekte in einer realen Umgebung kombiniert. Diese virtuellen Objekte sind in der realen Umgebung idealerweise fest lokalisiert und fügen sich somit in das reale Erscheinunsgbild ein. Typischerweise sind AR Anwendungen interaktiv, und stellen die virtuellen Objekte in Echtzeit und 3D in der realen Welt dar. Für die Definition von AR Anwendungen gibt es zudem keine Limitierung für die Darstellungstechnologie, wie zum Beispiel das Projekt Tango Tablet oder einem Head-Mounted Display. AR beschränkt sich zudem nicht auf den angesprochenen Sinn - so sind zum Beispiel AR Anwendungen mit visueller, olfaktorischer oder taktiler Umsetzung möglich. \citep{van2010survey}

\begin{figure}
  \centering
	\includegraphics[width=0.85\textwidth]{content/images/virtual-continuum.png} 
  \caption{Vereinfachte Darstellung des Realitäts-Virtualitäts-Kontinuum \citep{milgram1995augmented}}
  \label{fig:virtual-continuum}
\end{figure}

\subsection{Technische Anforderungen}

Virtuel Reality (VR) kapselt sich von der realen Umgebung ab und bindet reale Objekte in eine virtuelle Umgebung ein. VR konnte sich im Gegensatz zu AR deutlich schneller Entwickeln, da die technologischen Anforderungen an AR deutlich höher sind.

\subsubsection{Display Technologie}

Der erste wichtige Teil der technologischen Anforderungen an AR sind visuelle Anzeigen (visual displays), welche sich in diesem Anwendungsfall zunächst in drei Arten der Darstellung unterteilen und zudem unterschiedlich positioniert werden können. Die einfachste und günstigste Art der visuellen Darstellung in AR ist “video see-through”, wodurch die reale Umgebung durch eine Video Aufnahme ersetzt wird und die virtuellen Objekte digital in die Video Aufnahme gerendert werden. Das bietet außerdem die Möglichkeit Objekte aus der realen Umgebung zu entfernen oder zu ändern oder anhand der Luminanz Information vom Video das Rendering der virtuellen Objekte entsprechend an die Realität anzupassen. \\

Die nächste Möglichkeit zur Darstellung ist “optical see-through”, wodurch die virtuellen Objekte durch transparente Spiegel in das Sichtfeld des Betrachters gebracht werden. Anders als bei “video see-throught” bleibt die reale Auflösung für die visuelle Aufnahme des Betrachtes gleich und es können zudem keine Latenzprobleme beim Ändern des Betrachtungswinkels auftreten (parallax-effect).\\

Die dritte Möglichkeit ist die projizierte Darstellung, in der die Augmented Reality Überlagerung auf die realen Objekte projiziert werden. Diese Darstellung ermöglicht die Abdeckung vom gesamten Sichtfeld, benötigt aber eine entsprechende Kalibrierung bei Umgebungsänderungen.\\

Neben der Art der Darstellung können die Darstellungsvarianten anhand Ihrer Positionierung klassifiziert werden. Man unterscheidet zwischen am Kopf befestigten Displays (head-mounted), tragbare Displays  (hand-held) und räumlich positionierten Displays. Zu jeder dieser Displayarten gibt es wiederum unterschiedliche technische Umsetzungen, mit Ihren spezifischen Vor- und Nachteilen bezüglich Ihrer Anwendungsszenarien.

\subsubsection{Six degreees of freedom tracking}

Um eine virtuelle Projektion im realen Raum zu realisieren muss zunächst die Position und die relative Positonsänderung des Displays bestimmt werden, auch “augmented reality registration” genannt. Man spricht dabei üblichweise von “six degrees of freedom (6DOF)”, der Position im Raum (x, y, z) und der Orientierung (yaw, pitch, roll). \\

Frühe Techniken für die Registrierung benötigten eine speziell vorbereitete Räumlichkeit, denn sie basierten auf mechanischen, magnetischen oder Ultraschall Sensoren um die Position zu bestimmen. Diese Sensoren sind zwar immer noch im Einsatz und bilden auch den Grundstein für AR und VR Forschung, sind aber praktisch gesehen zu komplex und aufwändig für die meisten Anwendungsfälle.\\

Für ein grobes Positions-Tracking, vor allem auch außerhalb von Gebäuden wird GPS genutzt. Für großräumliche Anwendung ist GPS mit einer Varianz von 10-15 Metern durchaus praktikabel. Zum Beispiel um sichtbare Flugzeuge oder Sterne zu identifizieren. Innerhalb von Gebäuden basiert die grobe Positionierung oft auf verfügbaren Wifi Access Points oder RFID Markern. [An dieser Stelle sind sicherlich auch Bluetooth LE Beacons zu erwähnen.]\\

Optische Tracking Verfahren basierend auf Bildverarbeitung bieten … (Seite 7)
Optische Sensoren
Hybride mit Trägheits Snesoren!


