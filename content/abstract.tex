\begin{abstract}
\setlength{\parskip}{1em}

Project Tango ist eine neue mobile Plattform des Google Advanced Technology and Projects (ATAP) Teams, die in der Lage ist, Bewegungsverfolgung, Tiefenwahrnehmung und Umgebungswiedererkennung auf mobilen Endgeräten anbieten zu können. Google stellt mit diesem Ansatz ein System für Tablets und Smartphones zur Verfügung, die sich für viele Einsatzmöglichkeiten im Bereich Virtual Reality (VR), Indoor Navigation, Umgebungsvermessung und Augmented Reality (AR) Anwendungen eignet. Der Fokus dieser Arbeit liegt dabei auf die Anwendbarkeit für Augmented Reality Anwendungen. Auch wenn sich durch die kontinuierliche Bestimmung der relativen Geräteposition im Raum die Illusion einer dreidimensionalen AR Anwendung erfolgreich umsetzen lässt, gibt es dennoch das Problem, dass die virtuellen Objekte in einer Szene nicht von realen Objekten im Vordergrund überlagert werden können. 

Dadurch dass die Project Tango Platform den Entwicklern den Zugriff auf kontinuierliche Tiefeninformationen für den aktuellen Sichtbereich bietet, können diese Informationen genutzt werden, um das Problem dieser Überlagerung lösen zu können. Diese Arbeit stellt somit drei verschiedene Verfahren vor, mit denen diese Überlagerung der virtuellen Objekte umgesetzt werden können. Außerdem wird auf einen zusätzlichen Ansatz eingegangen, der zur Verbesserung dieser Verfahren die Bildinformationen der Farbkamera berücksichtigt. Die Verfahren werden im Laufe dieser Arbeit auf der Entwicklerhardware umgesetzt, anhand von Tests gegenübergestellt und hinsichtlich ihrer Anwendbarkeit evaluiert.

\setlength{\parskip}{0em}
\end{abstract}
\selectlanguage{english}
\begin{abstract}
\setlength{\parskip}{1em}

Project Tango is a new mobile platform by Google’s Advanced Technology and Projects (ATAP) Teams, which brings Motion Tracking, Depth Perception, and Area Learning to mobile devices. With Project Tango, Google is providing a technology to tablets and smartphones for building virtual reality (VR), indoor navigation, precise measurement and augmented reality applications. The focus of this document lies mainly upon augmented reality applications. Although you can build an effective 3D AR illusion with the continuous device motion tracking, there is still a problem, that the scenes virtual object cannot be occluded by real objects when they are in foreground.

Since the Project Tango platform is offering a continues depth perception of the current viewport, this depth information can be used to solve the missing occlusion issue. This work is introducing three approaches to enable an augmented reality occlusion by real objects. Additionally an approach will be discussed to optimize the depth occlusion by taking the color information by the device’s camera into account. These methods will also get implemented with the development kit, tested, compared and evaluated concerning their applicability.

\setlength{\parskip}{0em}
\end{abstract}
\selectlanguage{ngerman}


