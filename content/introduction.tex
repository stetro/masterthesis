\chapter{Einleitung}

Project Tango ist eine neue mobile Plattform des Google Advanced Technology and Projects (ATAP) Teams, welche Bewegungsverfolgung, Tiefenwahrnehmung und Umgebungswiedererkennung auf mobilen Endgeräten realisiert.

\begin{quotation}
\enquote{Project Tango combines 3D motion tracking with depth sensing to give your mobile device the ability to know where it is and how it moves through space.}  \citep{Proje19:online}
\end{quotation}

Diese Verfügbarkeit dieser Echtzeitdaten ermöglicht viele verschiedene neue Einsatzmöglichkeiten auf mobilen Endgeräten wie Smartphones und Tablets. Typische Einsatzszenarien dieser Plattform sind die Indoor Navigation, die Vermessung der Umgebung sowie andere Anwendungen im Bereich Virtual und Augmented Reality. Der Fokus dieser Forschungsarbeit liegt hier in dem Anwendungsbereich der dreidimensionalen Augmented Reality (AR). 

Die Anwendungsgebiete für Augmented Reality (dt. Erweiterte Realität) sind sehr vielseitig und liegen in der Medizin, der Unterhaltungsindustrie, Bildung und in vielen weiteren Industriezweigen. Eine barrierefreie Navigationshilfe, Einblendungen für die persönliche Assistenz, kontextsensitive Projektionen und Computerspiele sind Beispiele für typische Anwendungen, die durch AR umgesetzt werden können. 

Für die erfolgreiche Umsetzen einer Augmented Reality Anwendung, müssen die Kameraeigenschaften, wie Brennweite, Verzerrung und die Position der Kamera zu jeder Zeit und idealerweise in Echtzeit bekannt sein. Sensoren wie Kompass, INS (Trägheits\-navigations\-system) oder GPS können zwar eine grobe Lokalisierung ohne bekannte Merkmale im Raum ermöglichen, führen aber langfristig zu Fehlern, wenn keine optischen Referenzen gegeben sind. Mit Hilfe von der Bewegungsverfolgung durch Project Tango kann diese Lokalisierung der Kamera und somit die korrekte Positionierung von virtuellen Objekten im Raum deutlich zuverlässiger, in Echtzeit und ohne vordefinierte Merkmale im Raum realisiert werden. Project Tango eignet sich daher sehr gut für die Umsetzung und den Einsatz von AR Anwendungen.

\section{Augmented Reality Optimierung}

Um eine für den Betrachter effektive und optimierte Augmented Reality Anwendung umsetzen zu können, benötigt man laut \citet{azuma2001recent} die Möglichkeit mehr Informationen über relevante Objekte im realen Raum ermitteln zu können. Durch diese Informationen könnte dem Nutzer zum Beispiel eine Interaktion mit realen Objekten ermöglicht werden oder anhand optischer und semantischer Einordnung der Umgebung passende Funktionen angeboten werden. 

Hinsichtlich der zuletzt erwähnten Kontextsensitivität existieren viele Ansätze, basierend auf optischen Merkmalen der Umgebung. So kann zum Beispiel ein optisches Tracking von realen Objekten, wie von \citet{lee2008hybrid} beschrieben, umgesetzt werden. Project Tango nutzt bereits optische Merkmale, um eine Positionsverfolgung oder das Lernen der Umgebung umzusetzen. Wären diese Merkmale für den Entwickler als Schnittstelle verfügbar, könnte man mit diesen Informationen solche kontextsensitiven Anwendungen umsetzen. Der Fokus soll in dieser Arbeit jedoch nicht auf den optischen Merkmalen sondern auf den Tiefeninformationen, die Project Tango durch den eingebauten Tiefensensor in Form einer Pointcloud liefern kann, liegen.

Ein sinnvoller Einsatz von Augmented Reality Anwendungen besteht darin, virtuelle Objekte in eine echte Szene zu projizieren. Dabei überlagert die Projektion des virtuellen Objekts das aktuelle Kamerabild oder den aktuellen Sichtbereich und erwirkt dadurch den Anschein, als ob sich das virtuelle Objekt wirklich in der Szene befindet. Dieser Effekt funktioniert solange erfolgreich, bis ein reales Objekt sich räumlich vor das virtuelle Objekt bewegt und die zu erwartende Überlagerung des virtuellen Objekts nicht erfolgt. Diese fehlerhafte Darstellung durch eine fehlende Überdeckung ist in Abbildung \ref{fig:occlusion-problem} zu erkennen. 

\begin{figure}[h]
  \centering
	\includegraphics[width=1.0\textwidth]{content/images/occlusion-problem.png} 

  \caption{AR Projektion mit Project Tango - Links: Erfolgreiche Projektion. Rechts: Fehlerhafte Darstellung ohne Überdeckung.}
  \label{fig:occlusion-problem}
\end{figure}

Die Verfügbarkeit der Tiefeninformationen bei Project Tango könnte die Interaktionen oder Darstellungen in einer Augmented Reality Anwendung präziser an die echten räumlichen Gegebenheiten anpassen. Es existieren zum Beispiel prototypische Anwendungen, in denen virtuelle Markierungen passend an echten Objekten im virtuellen Raum positioniert werden können, indem sie auf die aktuellen Tiefeninformation des Sichtbereichs zurückgreifen. Eine weitere Idee ist es, Überlagerungen virtueller Objekte ermitteln zu können, an denen sich reale Objekte im Vordergrund befinden.

\section{Zielsetzung und Vorgehen}

Diese Arbeit will die Fragestellung beantworten, durch welche Verfahren mithilfe der Tiefeninformationen von Project Tango, automatisch und in Echtzeit die Überdeckung virtueller Objekte mit realen Objekten in einer Augmented Reality Szene realisiert werden kann. Dabei soll Project Tango als autonomes System betrachtet werden, welches diese Problemstellung selbstständig und mit den eingeschränkten Ressourcen dieser mobilen Plattform lösen soll.

Hierzu sollen zunächst bestehende Verfahren zur Bestimmung einer Augmented Reality Überdeckung durch eine Literaturrecherche gefunden werden. Diese Verfahren sollen dabei auf Ihre Anwendbarkeit mit der Project Tango Hardware überprüft werden. Sollten sich aus der Recherche weitere Ideen ergeben, wie speziell auf der Project Tango Hardware eine Überdeckung umgesetzt oder verbessert werden kann, sollen diese mit in die Arbeit eingebunden werden. Eine Idee könnte zum Beispiel sein, auch das Farbbild der normalen Kamera von Project Tango in die Optimierung der virtuellen Überlagerung einfließen zu lassen. Die identifizierten Verfahren sollen hiernach entsprechend implementiert werden, um sie darauf folgend in einer Testumgebung gegenüber zu stellen.

Strukturell wird in dieser Arbeit in Kapitel \ref{sec:thema} erst einmal auf die thematischen Grundlagen zu Augmented Reality und Project Tango eingegangen. Hier werden auch die existierenden Verfahren zur AR Überdeckung angesprochen. Unter Kapitel \ref{sec:algorithms} sind theoretische Grundlagen zu finden, die bei der späteren Umsetzung verschiedener Verfahren angewendet werden. Kapitel \ref{sec:optimization} beinhaltet die Argumentation und Beschreibung der gewählten Verfahren, welche unter Kapitel \ref{sec:implementation} auf der Project Tango Hardware umgesetzt werden. In Kapitel \ref{sec:evaluation} werden die vorliegenden Umsetzung in einem Testszenario gegenübergestellt, um eine Aussage treffen zu können, welcher Ansatz auf der Hardware oder für einen bestimmten Einsatz gut funktionieren könnte.


