\chapter{Verfahren zur Realisierung von Überdeckungen in Augmented Reality durch Tiefen\-informationen} \label{sec:optimization}


Die folgenden Abschnitte widmen sich den Verfahren zur möglichen Realisierung von Überlagerungen durch Tiefen- und Bildinformationen. Nach der Recherche zu möglichen Verfahren soll erst einmal der Grundlegende Ansatz von \citet{wloka1995resolving} zur einfachen Überlagerung durch Tiefeninformationen mit Hilfe der Projektion der von Project Tango gelieferten Pointcloud realisiert werden, da dieses Ausschlussverfahren das grundlegende Vorgehensmodell zur Überlagerung darstellt. \\

Die Kanten und Modell basierenden Verfahren zur AR Überdeckung aus Kapitel \ref{sec:ar-occlusion} werden hier nicht weiter berücksichtigt, da sie offensichtliche Nachteile gegenüber anderen Ansätzen bergen. So muss bei Modell basierten Verfahren bereits ein Modell der echten Umgebung existieren und die Kanten basierte Verfahren schränkt den Einsatz auf eine weniger komplexe Szene ein. Außerdem sind beide Verfahrensarten so konzipiert, dass sie keine direkten Tiefeninformationen benötigen, die aber von Project Tango generiert werden können. Aus diesem Grund widmen sich die darauf folgend beschriebenen Umsetzung der Rekonstuktions basierten Verfahren.\\

Während dieser Arbeit wurde zunächst versucht ein eigenes Echtzeit Rekonstuktionsverfahren, basierend auf einer Ebenenerkennung zu entwickeln, welches auch auf der beschriebenen mobilen Project Tango Hardware realisierbar ist. Daraufhin wurde nach weiteren Recherchen ein neuer möglicher Rekonstruktionsmechanismus gefunden, der für den Einsatz auf mobiler Hardware konzipiert wurde. Auch dieser wird hier näher beschrieben, um ihn später zu implementieren und zu testen. Zuletzt soll näher auf die Möglichkeit eingegangen werden, die resultierenden Tiefeninformationen aus der Pointcloud oder aus dem Rendering der Rekonstruktion mit Hilfe der Bildaufnahmen der Farbkamera, zu verbessern. \\


\section{Verdeckung durch Pointcloud Projektion} \label{sec:pc-projection}

Die erste und weniger aufwändige Idee eine Überlagerung in Augmented Reality zu realisieren, ist die Überführung der Pointcloud in eine Depthmap, die wiederum in den Renderingprozess mit eingebracht wird. Das Verfahren von \citet{kanbara2000stereoscopic} verfolgt einen ähnlichen Weg mit einer Stereokamera und einer video see-through Displaytechnologie in Form eines Head-Mounted Display. Wie in Abbildung \ref{fig:stereo-depth-map} zu erkennen, bestimmen sie mit Hilfe der Stereokamera Tiefeninformationen, die das gerenderte virtuelle Objekt an den Positionen ausspart, an denen Tiefeninformationen im Vordergrund vorliegen. Aufgrund von weiteren Forschungsergebnissen aus dem Bereich des Stereomatchings, erreichen sie bessere Ergebnisse als \citet{wloka1995resolving}, die das Vorgehen als Erste vorgestellt haben.

\begin{figure}[h]
  \centering
	\includegraphics[width=1.0\textwidth]{content/images/methods/stereo-depth-map.png} 
  \caption{Visualisierung der Methode zur Vedeckung durch Depth Maps. Übernommen von \citet{kanbara2000stereoscopic}}
  \label{fig:stereo-depth-map}
\end{figure}

Anders als im Verfahren von \citet{kanbara2000stereoscopic} wird hier eine Depthmap anhand der vorhandenen Pointcloud aus dem Infrarotsensor von Project Tango gewonnen. Dadurch, dass die intrinsischen Kameraeigenschaften der Farbkamera zur Verfügung stehen, welche auch die Aufnahmequelle der Infrarotpunkte ist, kann ein Punkt \(P = [X, Y, Z]\) der Pointcloud mit der Gleichung \ref{eq:projection} auf die Bildebene überführt werden. Hier stehen die Variablen \(f_{x/y}\) für die Brennweite und \(c_{x/y}\) für den Bildmittelpunkt auf der Bildebene. \citep{Tango90:online}

\begin{equation}\label{eq:projection}
x = \frac{X* f_x * \frac{r_d}{r_u}}{Z}  + c_x
\qquad
y = \frac{Y* f_y * \frac{r_d}{r_u}}{Z}  + c_y
\end{equation}

Da die Linsen einer Kamera nie perfekte Brechungseigenschaften besitzen, muss an dieser Stelle auch die Verzerrung der Linse mit berücksichtigt werden. In den Gleichungen aus \ref{eq:projection} ist diese Verzerrung in den Parametern \(\frac{r_d}{r_u}\) enthalten. Zur Kalibrierung der Linse wird bei Project Tango für die normale Farbkamera und somit auch für die Aufnahme der Infrarotpunkte die Technik von \citet{tsai1987versatile} verwendet  \citep{Tango90:online}. Das hier beschriebene Verzerrungsmodell basiert dabei auf den drei Parametern \(k_{1} \downarrow  k_{3}\), welche die radiale Verzerrung ausgehend vom Linsenmittelpunkt modellieren kann. Hierzu wird für jeden Punkt, wie in Gleichung \ref{eq:distortion} gezeigt, die radiale Distanz \(r_u\) zum Linsenmittelpunkt und die durch die Linseneigenschaft verzerrte radiale Distanz \(r_d\) ermittelt. Dieses Verhältnis streckt oder staucht die Position radial auf der Bildebene.


\begin{equation} \label{eq:distortion}
r_u = \sqrt{\frac{X^2 + Y^2}{ Z^2}} 
\qquad
r_d = r_u + k_1 * r_u^3 + k_2 * r_u^5 + k_3 * r_u^7
\end{equation}

An dieser berechneten Position auf der Bildebene bzw. Depthmap wird nun ein Punkt mit einem Graustufenwert, entsprechend der Entfernung \(|\overrightarrow{PO_{cam}}|\) vom Punkt \(P\) zum Kameraursprung \(O_{cam}\) gezeichnet. Der Farbwert richtet sich dabei nach der Konvention des Renderingframeworks und den Informationen über die vordere und hintere Clippingebene.

Die Auflösung des Tiefensensors der Project Tango Hardware ist mit \(320 \times 180\) Pixeln gegenüber der Auflösung der Farbkamera mit \(1280 \times 720\) vier mal kleiner. Zusätzlich ist die Dichte der Pointcloud zum eigentlichen Sensor geringer als ihre eigene Auflösung. So würde man bei einer Auflösung von \(320 \times 180 = 57600\) Tiefenpunkte bei idealen Verhältnissen erwarten. Project Tango liefert jedoch unter guten Bedingungen durchschnittlich \(17000\) Tiefenpunkte. Aufgrund der kleineren Auflösung und der geringeren Informationsdichte werden die gezeichneten Punkte auf der Depthmap hier mit einem Radius von 4 Bildpunkten gezeichnet. 

Nachdem die Depthmap generiert wurde, kann zum Ausschluss der Pixel der virtuellen Objekte, welches sich hinter einem realen Objekt befinden, der Z-Buffer Algorithmus aus Kapitel \ref{sec:z-buffer} wie von \citet{wloka1995resolving} beschrieben, angewendet werden. Hierzu wird vor dem virtuellen Rendering der Z-Buffer mit den generierten Informationen aus der Depthmap gefüllt. Pixel der virtuellen Objekte werden somit nicht gerendert, wenn an dieser Position eine geringere Distanz zu realen Objekten in den Tiefeninformationen vorliegen. 




\section{Planare Rekonstruktion}

\subsection{Verfahren zur Ebenendetektion}

\subsubsection{RANSAC}

Der \enquote{RAndom SAmple Consensus} Algorithmus (RANSAC), vorgestellt von \citet{fischler1981random}, ist in der Lage, aus einer Menge von Daten mit vielen Ausreißern, die Parameter für ein passendes Modell zu schätzen. Anders als andere Schätzverfahren wie \enquote{Least-Median} oder \enquote{M-Schätzer}, welche aus der Statistik Literatur entnommen und entsprechend angepasst wurden, wurde RANSAC speziell für die Anwendung in der Computer Graphik entwickelt. Der Kern dieses Algorithmus ist das wiederholte Bestimmen eines Modells aus zufälligen und für das Modell ausreichenden Stichproben. Listing \ref{lst:ransac} zeigt den Verlauf des RANSAC Algorithmus. Die Anzahl der Iterationen \(N\) hängt dabei allein von dem Anteil der Ausreißer in den Messwerten ab. Daher sollte sie entsprechend gewählt werden, um die Wahrscheinlichkeit zu verringern, dass Ausreißer in den Stichproben enthalten sind. \citep{derpanis2010overview} \\

\begin{lstlisting}[caption=Der RANSAC Algorithmus, label=lst:ransac]
Eingabe: Messwerte P, Modelltoleranz e, maximale Iterationen N
Ausgabe: Modell m, Unterstützende Messwerte Pm

1. Wähle zufällig so viele Stichproben aus den Messwerten P,
   wie nötig sind, um das Modell zu bestimmen
2. Bestimme aus den gewählten Stichproben das Modell m
3. Ermittle die Anzahl der Messwerte P, die mit einer 
   entsprechenden Toleranz e das ermittelte Modell m unterstützen
4. Wenn prozentual genügend Messwerte aus P das Modell m unterstützen,
   ermittle aus den unterstützenden Messwerten Pm durch lineare 
   Regression erneut das finale Modell m und terminiere
5. Wiederhole die Schritte 1-4 N mal
\end{lstlisting} 

Um mit dem RANSAC Algorithmus Ebenen in einer Punktewolke bestimmen zu können, werden pro Iteration drei Stichproben \(A\), \(B\) und \(C\) gewählt. Das Ebenenmodell, hier in der Hesse Normalform mit dem Normalenvektor \(\vec{n}\) und dem Abstand zum Koordinatenursprung \(d\), lässt sich dabei durch die Gleichung \ref{eq:normalform} bestimmen.

\begin{equation}\label{eq:normalform}
\vec{n} =\left|\left| \vec{AB} \times \vec{AC}\right|\right|
\qquad
\vec{D} = \vec{A} \cdot \vec{n}
\qquad
d = \vec{D}\left[x\right] + \vec{D}\left[y\right] + \vec{D}\left[z\right]
\end{equation}


\subsubsection{3D Hugh Transformation}

\subsubsection{Agglomeratives Clustering}

\subsubsection{Region Growing}

\subsection{Bestimmung der Ebenenausbreitung}

\subsubsection{Convex Hull Algorithmus}

\subsubsection{Triangulation}

\subsection{Planare Rekonstruktion als Echtzeit Prozess}

\subsubsection{Clusteringverfahren}

\subsubsection{...}

\subsection{Kantenverbesserung durch Einbeziehung von Bildmaterial}

\subsubsection{...}

\section{Echtzeit Polygon Rekonstruktion} \label{sec:polygon_reconstruction}

Das zweite rekonstuktionsbasierte Überlagerungsverfahren bezieht sich in diesem Kapitel auf dem aktuellen Forschungsstand der Echtzeitrekonstruktion und soll in den folgenden Absätzen näher beschrieben werden. Die Echtzeit 3D Rekonstruktion ist bereits ein etabliertes Forschungsgebiet in der Computergrafik und gewinnt, aufgrund von kostengünstigen Consumer Tiefensensoren, wie die Microsoft Kinect\footnote{Microsoft Kinect - \url{https://dev.windows.com/en-us/kinect} (13.03.16)}, Xtion\footnote{Ausus Xtion - \url{https://www.asus.com/de/3D-Sensor/Xtion_PRO_LIVE/} (13.03.16)} oder Structure\footnote{Occipital Structure - \url{http://structure.io/} (13.03.16)} , zunehmend an Bedeutung. Dabei liegt der Fokus zunehmend auf der Echtzeitrekonstruktion, da diese Geräte in der Lage sind, Tiefeninformationen, zwar mit leichten Messfehlern, aber in Echtzeit, zu liefern. \citet{niessner2013real} erwähnen an dieser Stelle zudem den möglichen Einsatz für Augmented Reality:

\begin{quote}
\enquote{The ability to obtain reconstructions
in real-time opens up various interactive applications including:
augmented reality (AR) where real-world geometry can be fused
with 3D graphics and rendered live to the user; ...} \citep{niessner2013real}
\end{quote}

Die Herausforderung in der Echtzeitrekonstruktion liegt dabei in der möglichst performanten Fusion von mehreren überlagernden Tiefenbildern, welche aus verschiedenen Betrachtungswinkeln aufgenommen werden. Hieraus soll eine möglichst detaillierte Repräsentation der echten Umgebung generiert werden, welche sich im Idealfall stetig verbessert. Diese Problemstellung unterscheidet sich dabei von herkömmlichen Rekonstruktionsverfahren wie dem von \citet{hoppe1992surface} und der Poission Rekonstruktion von \citet{kazhdan2006poisson}. Aktuelle Verfahren nutzen verschiedenste optimierte Datenstrukturen, welche zudem durch den Einsatz von entsprechenden GPU Implementierungen beschleunigt und in Echtzeit angewendet werden können. Hier spielt die Gegenüberstellung von Detailgrad, der Skalierung und Geschwindigkeit stets eine große Rolle. \citep{niessner2013real} 

Bekannte Verfahren wie KinectFusion \citep{newcombe2011kinectfusion}, ein SLAM Verfahren von \citet{bylow2013real} oder DynamicFusion \citep{newcombe2015dynamicfusion} nutzen die \enquote{Truncated Signed Distance Function}, kurz TSDF, zur Speicherung und Migration der Oberflächeninformation mehrerer Depth Maps. Das Verfahren von \citet{niessner2013real} erweitert diesen Ansatz mit einem effizienten räumlichen Hashingverfahren, um die Zugriffszeiten und Speicherverbrauch zu minimieren. Darüber hinaus nimmt das Verfahren Chisel von \citep{Klingensmith_2015_7924} diese Vorzüge auf und kombiniert TSDF mit \enquote{visual-inertial odometry}, der Trackingtechnologie von Project Tango. In den folgenden Absätzen werden die Mechanismen hinter TSDF, dem räumlichen Hashing und den Vorzügen von Chisel näher erläutert. Außerdem soll auch noch auf das Rendering der TSDF Oberfläche durch Marching Cubes eingegangen werden, welches in Chisel verwendet wird. 


\subsection{Truncated Signed Distance Function}

Bei der von \citet{curless1996volumetric} vorgestellten räumlichen Repräsentation von Oberflächen, Truncated Signed Distance Function (TSDF), wird der Raum in Voxel einer gewünschten Auflösung unterteilt. Anders als bei Occupancy Maps, in denen die Voxel als sichtbar oder unsichtbar binär markiert werden, werden bei TSDF in den Voxeln die jeweiligen Entfernungen zur nächst gelegenen Oberfläche angegeben. Wichtig dabei ist das Vorzeichen, welches angibt, ob sich ein Voxel innerhalb oder außerhalb eines Objektes befindet. Abbildung \ref{fig:tsdf} zeigt unter a) die Ergebnisse mit Occupancy Maps und in b) die Voxel von TSDF. \citep{curless1996volumetric} 

Gefüllt wird die Repräsentation durch die Depth Maps und der entsprechenden Kameraposition, die im Fall von Project Tango durch Motion Tracking bereits gegeben ist. So wird für jede Tiefeninformation ein Strahl ausgehend von der Kameraposition generiert, der die durchgeschnittenen Voxel aktualisiert. Der Stahl ist dabei von der Länge begrenzt, um die zu aktualisierenden Voxel klein zu halten und zudem keine Oberflächen zu aktualisieren, die sich weiter hinter der gefundenen Oberfläche befindet. Dieses Vorgehen ist in Abbildung \ref{fig:tsdf} c) zu erkennen. \citep{Compu66:online} 

\begin{figure}
  \centering
	\includegraphics[width=1.0\textwidth]{content/images/methods/tsdf.png} 
  \caption{a) Beispielhafte Voxel Füllung von Occupancy Maps; b) Beispielhafte Voxel Füllung durch TSDF; c) Exemplarische 2D Darstellung der Oberfläche mit entsprechenden Strahlensatz für die TSDF. Übernommen von \citet{Compu66:online}}
  \label{fig:tsdf}
\end{figure}

Der Vorteil dieser Repräsentation liegt darin, dass die konkreten Oberflächeninformationen, anders als bei der Diskretisierung von Occupancy Maps, nicht verloren gehen. Das heißt, dass trotz einer gröberen Voxel Struktur stets der Nulldurchgang rekonstruiert werden kann. Neben der Entfernung zur nächsten Oberfläche wird zusätzlich noch ein Gewichtungswert in jedem Voxel gespeichert. Das ermöglicht es leichtes Rauschen durch eine einfache Mittelung mehrerer Messerergebnisse der Oberfläche zu unterdrücken. Außerdem kann somit eine stetige Optimierung der Oberfläche vorgenommen werden. \citep{Compu66:online}

\citet{hoppe1992surface} nutzen in Ihrem offline Rekonstruktionsverfahren auch die hier beschriebene TSDF. Jedoch bestimmen sie für jeden festgehaltenen Punkt der Pointcloud die umliegenden Nachbarn, um eine Tangentialebene zu ermitteln, von der aus die auf der Normalen liegenden Voxel mit der entsprechenden Distanz aktualisiert werden. Für die Echtzeitrekonstruktion ist dieses Vorgehen jedoch zu aufwändig. Hier werden die Voxel nicht anhand der exakten euklidischen Distanz aktualisiert, sondern es wird mithilfe des Raycastings, ausgehend von der Tiefenkamera, eine projizierte Distanz als Approximation verwendet. \citep{Compu66:online} Später bei der detaillierten Beschreibung von Chisel (Kapitel \ref{sec:chisel}) wird das Vorgehen auch nochmal Grafisch in Abbildung \ref{fig:tsdf-sketch} erläutert.

\subsection{Spatial Hashing}

Das Problem der Echtzeitrekonstruktion ist wie bereits angesprochen der Kompromiss zwischen dem Detailgrad, der Skalierung der zu rekonstruierenden Szene und der Performance der Rekonstruktion. Auch die TSDF-Repräsentation ist sehr speicherintensiv und benötigt für die zu scannende Szene reservierten Speicher, der auf mobilen Endgeräten nur begrenzt verfügbar ist. Daher muss auch für größere Rekonstruktionen oder Rekonstruktionen unbekannter Größe ein dynamischer Ansatz gefunden werden, Speicher zu verwalten. \citet{Klingensmith_2015_7924} erwähnt dazu, dass einige Verfahren Octrees einsetzen, die zwar äußerst dynamisch sind, jedoch einen deutlichen Nachteil hinsichtlich der Zugriffszeiten auf die Voxel bergen. 

\begin{figure}[h]
  \centering
	\includegraphics[width=1.0\textwidth]{content/images/methods/hashing.png} 
  \caption{a) Voxel Hashing Datenstruktur. Übernommen von \citet{niessner2013real} b) Darstellung der relevanten Voxel Chunks für die Aktualisierung. Übernommen von \citet{Klingensmith_2015_7924}}
  \label{fig:hashing}
\end{figure}

\citet{niessner2013real} führen daher eine zwei-Ebenen Struktur ein, die auf der zweiten Ebene eine Menge von Voxel räumlich zusammenfassen. Diese werden hier Chunks genannt. Auf der ersten Ebene können diese Chunks in einer Hash Tabelle räumlich mit einer Hashfunktion identifiziert werden. Das ermöglicht somit einen nahezu direkten Zugriff auf räumliche Voxel und bietet zudem Chunks dynamisch zu allokieren. Als Hash der Chunkposition \(x\), \(y\) und \(z\) wird die folgende Hashfunktion aus Gleichung \ref{eq:spatial_hash} verwendet. Bei den Variablen \(p_1\), \(p_2\) und \(p_3\) handelt es sich um willkürlich hohe Primzahlen und \(n\) entspricht der Größe der Hash Tabelle. Da Kollisionen in der Hash Tabelle nicht vollkommen ausgeschlossen werden können, werden die Chunks gegebenenfalls in der Hashtabelle verkettet.

\begin{equation}\label{eq:spatial_hash}
H(x,y,z) = (x * p_1 \oplus y * p_2 \oplus z * p_3) \mod n
\end{equation}

\subsection{Marching Cubes}

Die meisten Echtzeitrekonstruktionen durch TSDF wie KinectFusion sind GPU-Umsetzungen. Diese haben daher die Möglichkeit ein hardwarebeschleunigtes Rendering durch Raycasting durchzuführen. Das Verfahren Chisel von \citet{Klingensmith_2015_7924}, welches eine reine CPU Umsetzung ist, nutzt hingegen einen indirekten Weg zum Rendering durch die Marching Cubes Triangulation. 

\begin{figure}[h]
  \centering
	\includegraphics[width=1.0\textwidth]{content/images/methods/marchingcubes.png} 
  \caption{a) Marching Cubes Voxel Repräsentation mit den Ecken und den Kantenschnittpunkten b) Die 15 möglichen 3D Polygon Varianten. Übernommen von \citet{MarchingCubes:online}}
  \label{fig:marchingcubes}
\end{figure}

Marching Cubes nach \citet{lorensen1987marching} ist ein Algorithmus um aus einer, als Voxel repräsentierten, Isofläche Polygone zu bestimmen, die dieser Fläche möglichst nahe kommt. Hierzu werden zu jedem Voxel die Ecken \(v1\) bis \(v8\) anhand der Nachbarvoxel und des Distanzwertes untersucht, ob sie innerhalb oder außerhalb eines Objektes liegen. Zusätzlich werden zu jeder Kante auf dem Voxel, wenn ein Schnitt der Isofläche existiert, die Schnittpunkte auf den Kanten \(e1\) bis \(e12\) bestimmt. Abbildung \ref{fig:marchingcubes} a) zeigt die Ecken und Kantenschnittpunkte eines Voxels. 

Je nach binärer Gewichtung der Ecken können hiernach aus einem Katalog von 255 Varianten die Polygone nachgeschlagen werden. Inhalt des Katalogs sind die Indizes der Kantenschnittpunkte \(p_{e1}\) bis \(p_{e12}\), aus denen Polygone generiert werden können. Alle diese 255 Varianten können auf 15 verschiedene Fälle zurückgeführt werden, die sich nur in Rotation oder Symmetrie unterscheiden. Die 15 Varianten sind in Abbildung \ref{fig:marchingcubes} b) zu finden. \citep{MarchingCubes:online} 

\subsection{Chisel mit Space Carving} \label{sec:chisel}

Das bereits erwähnte Verfahren Chisel von \citet{Klingensmith_2015_7924} verwendet alle zuvor erwähnten Techniken der TSDF, Spatial Hashing und der Marching Cubes Überführung. Sie sprechen dabei von einem \enquote{dynamic spatial-hashed truncated distance field}. Das für den mobilen Einsatz optimierte Verfahren ist in der Lage eine Echtzeitrekonstruktion von Räumen von bis zu \(300 m^2\) mit einem Detailgrad von zwei bis drei Zentimetern zu erstellen. Zudem können neben Tiefeninformationen auch gefärbte Pointclouds verarbeitet werden, wodurch ein gefärbtes Mesh durch Marching Cubes generiert werden kann. Die Farbinformationen sind jedoch auf die Voxel Auflösung des Verfahrens beschränkt. \citep{Klingensmith_2015_7924}

Zusätzlich erweitern sie den TSDF Algorithmus um die \enquote{space carving} Funktionalität. Sie betrachten dabei den Strahl von der Kamera zur Oberfläche als eine Art Bedingung, in der alle durchstoßenden Voxel bis zur Oberfläche eine negativen Wert beinhalten müssen. Ist das nicht der Fall, so wird ein Voxel außerhalb der inneren Begrenzung auf den leeren Ursprungswert gesetzt. Der Pseudocode in Listing \ref{lst:chisel} sowie die Abbildung \ref{fig:tsdf-sketch} erläutern dieses Verhalten. Diese Verbesserung führt dazu, dass die Rekonstruktion bei stark rauschenden Tiefeninformationen, besonders an Objektkanten, deutlich verbessert wird. Außerdem ist das Verfahren hiermit in der Lage dynamische Änderungen in der Umgebung zu detektieren und neue Oberflächen entsprechend zu aktualisieren. So beeinflusst, als Beispiel, sich eine im Bild bewegende Personen nur kurz die Voxel der TSDF. \citep{Klingensmith_2015_7924}


\begin{lstlisting}[mathescape,caption=Chisel TSDF Algorithmus, label=lst:chisel, float=htbp]
Eingabe: Pointcloud $C$, Kameratransformation $P_{cam}$, 
         Strahlenbegrenzung $t$

für jeden Tiefenwert $\vec{p}$ aus $C$
    bestimme die Oberflächenposition $\vec{z}$ aus $\vec{p}$ und $P_{cam}$
    bestimme einen Strahl $\vec{r}$ aus $\vec{z}$ und $P_{cam}$
    bestimme den Begrenzungsbereich $t_{vor}$, $t_{nach}$ mit $t$ um $\vec{z}$ auf $\vec{r}$
    # space carving
    für jeden Voxel $v$ zwischen Kamera und $t_{vor}$
        wenn die Distanz im Voxel negativ ist
            setze Voxel zurück
    # normale TSDF Bestimmung
    für jeden Voxel $v$ zwischen $t_{vor}$ und $t_{nach}$
        bestimme die Voxeldistanz zu $z$
        setze das Gewicht w des Voxels $v$
\end{lstlisting}

\begin{figure}[h]
  \centering
	\includegraphics[width=1.0\textwidth]{content/images/methods/tsdf-sketch.png} 
  \caption{Exemplarische Visualisierung des Strahls $\protect\vec{r}$ und der zu aktualisierenden Voxel $\protect v$ im TSDF Algorithmus. Links ist hier das Project Tango Gerät zu erkennen, welches Tiefeninformationen der Oberfläche eines Objekts rechts aufgenommen hat.}
  \label{fig:tsdf-sketch}
\end{figure}

Neben space carving wurden zudem eine variable Strahlenbegrenzungen und Gewichtungen der Voxel, abhänig von der jeweils aufgenommenen Tiefe, implementiert. Diese Funktion berücksichtigt die Messungenauigkeiten des Tiefensensors, die bei größerer Entfernung der Oberfläche zum Tiefensonsor zunehmen können. \citep{Klingensmith_2015_7924}


\section{Tiefenanpassungen durch Farbbilder}

Aus allen zuvor beschriebenen Verfahren werden letztendlich Tiefeninformationen, in Form von geometrischen Primitiven oder Punkten im Raum gewonnen. Diese werden passend zur aktuellen Kameraposition als Tiefenbild gerendert und füllen den Z-Buffer für entsprechende Aussparungen bei der Überdeckung virtueller Objekte. Auf Grund von Sensorungenauigkeiten und größeren Auflösungen der Rekonstruktionsverfahren können dabei fehlerhafte Tiefeninformationen in den Z-Buffer gelangen, die zu Fehlern bei der Bestimmung der Überdeckung führen können. Dieses Problem ist am Beispiel der Pointcloud Projektion aus Kapitel \ref{sec:pc-projection} in Abbildung \ref{fig:pc-noise} zu erkennen. 

\begin{figure}[h]
  \centering
	\includegraphics[width=1.0\textwidth]{content/images/methods/pc-noise.png} 
  \caption{Überdeckung mit einfacher Pointcloud Projektion. Links: Resultat der Überdeckung. Mitte: Darstellung des Z-Buffers mit dem Ausschluss des virtuellen Objekts. Rechts: Darstellung der Pointcloud.}
  \label{fig:pc-noise}
\end{figure}

Die Reduktion von Ungenauigkeiten im Tiefenbild könnte durch einen einfachen Weichzeichner erreicht werden. Dieser würde jedoch die Kanten im Farbbild nicht berücksichtigen und somit fehlerhafte Tiefengradienten an den Kanten erzeugen und einen durchaus größeren Fehler generieren. \citet{newcombe2011kinectfusion} wenden einen sogenannten \enquote{Bilateralen Filter} in ihrem KinectFusion Rekonstruktionsverfahren an, bevor sie die Tiefeninformationen in die TSDF Repräsentation einfließen lassen. Dieser Filter von \citet{tomasi1998bilateral} ermöglicht das Weichzeichnen ohne dabei die Kanten im Bild zu übergehen, bezieht sich jedoch nur auf das selbe Bild, auf dem der Filter angewendet wird. 

\citet{liu2012guided} hingegen wenden einen sogenannten \enquote{Guided Filter} in ihrem Verfahren zur Optimierung der Tiefeninformationen für Kinect ähnliche Sensoren auf das Tiefenbild an. Dieser Filter von \citet{he2010guided} ist in der Lage, auf Grundlage eines anderen Leitbildes ein Weichzeichnen durchzuführen, ohne dabei die Kanten des Leitbildes zu überschreiten. Auch wenn \citet{petschnigg2004digital} eine Erweiterung, den Joint Bilateral Filter, vorstellen, der auf Basis eines anderen Leitbildes eine Weichzeichnung ohne Kantenüberschreitung ermöglicht, bietet der Guided Filter eine deutlich bessere Performance. Außerdem verhindert der Guided Filter Fehlerartefakte im Resultat, die bei dem Bilateralen Filter an den Kanten auftreten können. \citep{he2010guided} 

Ausgehend von der Eingangsgrafik \(p\), einem Leitbild \(I\) und dem Ergebnisbild \(q\) wird das grundlegende Modell von dieser Art Filter mit der Gleichung \ref{eq:gf-model} beschrieben. Diese Gleichung findet für jeden Pixel \(i\) in \(q\) eine gewichtete Summe über jeden Pixel \(j\) einer vordefinierten Ausschnittgröße. \(W_{ij}\) entspricht dabei dem Gewicht, welches für die jeweiligen Pixel \(p_j\) gilt. Bei dieser Faltung gilt üblicherweise  \(\sum_{j} W_{ij}(I)=1  \text{ für } \forall i \in [1\ldots |p|]\). \citep{he2010guided}

\begin{equation} \label{eq:gf-model}
q_{i} = \sum_j W_{ij}(I)p_j
\end{equation}

Der Filterkern \(W_{ij}(I)\) des Guided Filter, zu finden in Gleichung \ref{eq:gf-W}, ist, wie auch beim bilateralen Filter, abhängig von einem Leitbild \(I\), um die Gewichte entsprechend den Kanten des Leitbildes an der Position ermitteln zu können. Die Variablen \(\mu_k\) und \(\sigma^2_k\) beschreiben jeweils den Mittelwert und die Abweichung des Leitbildes im Bildausschnitt \(w_k\). \(|w|\) entspricht der Pixelgröße des Ausschnitts. \citep{he2010guided}

\begin{equation} \label{eq:gf-W}
W_{ij}(I) = \frac{1}{|w|^2} \sum_{k:(i,j) \in w_k} (1+\frac{(I_i-\mu_k)(I_j-\mu_k)}{\sigma^2_k + \epsilon})
\end{equation}

Dieser Filterprozess wird auch als eine translationsabhängige Faltung bezeichnet, die üblicherweise aufwändig ist und dessen Berechnungsaufwand abhängig zur Filterkern Größe (\(|w|\)) ist. \citet{he2010guided} stellen jedoch noch eine andere Definition des Filters zur Verfügung, in denen alle Summen der Form \(\sum_i\in w_k f_i\) entsprechen und dadurch mit der Bildintegrationstechnik von \citet{crow1984summed} in \(O(N)\) gelöst werden können. Der Guided Filter wird in der letztendlichen Implementierung nach Gleichung \ref{eq:gf-final} implementiert, in der die Koeffizienten \(\overline{a}_i\) und \(\overline{b}_i\) dem Mittelwert über \(a_k\) aus Gleichung \ref{eq:gf-a} und \(b_k\) aus Gleichung \ref{eq:gf-b} für jedes Fenster \(w_k\) entspricht. So wird auch \(\overline{p}_k\) durch \(\frac{1}{|w|} \sum_{i \in w_k} p_i\) berechnet.

\begin{equation} \label{eq:gf-final}
q_i = \overline{a}_iI_i+\overline{b_i}
\end{equation}

\begin{equation} \label{eq:gf-a}
a_k = \frac{\frac{1}{w} \sum_{i \in w_k} I_i p_I - \mu_k \overline{p}_k}{\sigma_k^2+\epsilon}
\end{equation}

\begin{equation} \label{eq:gf-b}
b_k = \overline{p}_k - a_k\mu_k
\end{equation}

Der Parameter \(\epsilon\) reguliert im beschriebenen Filter von \citet{he2010guided} welcher Bildanteil als beizubehaltende Kante im resultierenden Bild gewertet werden soll und somit stärker oder schwächer in der Gewichtung \(W_{ij}\) Einfluss nimmt. Neben diesem Regulierungsfaktor ist auch die Wahl des Radius \(r\) für den Ausschnitt \(w_k\) als Eingabe für diesen Filter wichtig. Der Radius wirkt sich laut \citet{he2010guided} jedoch nicht wie beim bilateralen Filter auf die Laufzeit des Filters aus. 

\begin{quote}
\enquote{One more advantage of the guided filter over the bilateral filter is that it automatically has an \(O(N)\) time exact algorithm. \(O(N)\) time implies that the time complexity is independent of the window radius \(r\), so we are free to use arbitrary kernel sizes in the applications.} \citep{he2010guided}
\end{quote}

\begin{figure}[h]
  \centering
	\includegraphics[width=1.0\textwidth]{content/images/methods/gf-result.png} 
  \caption{Guided Filter Anwendungsbeispiel. Das Tiefenbild links ergibt durch den Guided Filter mit dem Leitbild in der Mitte das Ergebnis im rechten Bild.}
  \label{fig:gf-result}
\end{figure}

Mit einer Komplexität von \(O(N)\) findet dieser Filter erfolgreich Anwendung in verschiedensten Bereichen. Er wird zum Beispiel zur Rauschunterdrückung, dem Weichzeichnen oder Verstärken von Details, zur HDR Kompression, dem Entfernen von matten Bildeigenschaften oder, wie in diesem Fall, zum zusammengeführten Anreichern von Bildinformationen verwendet \citep{he2010guided}. Angewendet auf das ermittelte Tiefenbild kann dieser Guided Filter, mit dem jeweiligen RGB Bild als Leitbild, ein Rauschen eliminieren und die Kanten der Tiefeninformationen durch einen entsprechend groß gewählten Fensterradius \(r\) und Regulierungsfaktor \(\epsilon\), an die Kanten der Kameraaufnahme angleichen \citep{liu2012guided}. Ein Beispiel für eine erfolgreiche Anwendung dieses Filters ist in Abbildung \ref{fig:gf-result} zu sehen.



