\section{Octree}

Ein Octree ist zunächst eine Datenstruktur, die wie ein Baum mit beliebiger Tiefe aufgebaut ist und pro Knoten Acht Kinder besitzt. Die Funktion ist dabei in \(\mathbb{R}^3\) gleich zu einem Binärbaum in \(\mathbb{R}\) oder einem Quadtree in \(\mathbb{R}^2\). Ein Knoten repräsentiert im Octree einen Würfel, der durch seine Kinder in Acht Kind-Würfel aufgeteilt wird. Durch diese räumliche Aufteilung ergeben sich verschiedene Vorteile gegenüber linearen Datenstrukturen. So müssen Bereiche zum festhalten räumlicher Informationen im Octree nur dann allokiert werden, wenn diese Bereiche auch verwendet werden. Speichert man Punkte in den untersten Knoten eines Octrees kann man durch eine Tiefenbegrenzung beim Zugriff auf den Baum ein sehr effektives Downsampling der Punkte vornehmen. Zuletzt entstehen durch die Knoten einer bestimmten Tiefe ein Cluster, zu denen in diesem Fall Punkte bei einer Aufnahme hinzugefügt werden und für eine weitere Verarbeitung extrahiert werden können.\\